
% Default to the notebook output style

    


% Inherit from the specified cell style.




    
\documentclass[11pt]{article}

    
    
    \usepackage[T1]{fontenc}
    % Nicer default font (+ math font) than Computer Modern for most use cases
    \usepackage{mathpazo}

    % Basic figure setup, for now with no caption control since it's done
    % automatically by Pandoc (which extracts ![](path) syntax from Markdown).
    \usepackage{graphicx}
    % We will generate all images so they have a width \maxwidth. This means
    % that they will get their normal width if they fit onto the page, but
    % are scaled down if they would overflow the margins.
    \makeatletter
    \def\maxwidth{\ifdim\Gin@nat@width>\linewidth\linewidth
    \else\Gin@nat@width\fi}
    \makeatother
    \let\Oldincludegraphics\includegraphics
    % Set max figure width to be 80% of text width, for now hardcoded.
    \renewcommand{\includegraphics}[1]{\Oldincludegraphics[width=.8\maxwidth]{#1}}
    % Ensure that by default, figures have no caption (until we provide a
    % proper Figure object with a Caption API and a way to capture that
    % in the conversion process - todo).
    \usepackage{caption}
    \DeclareCaptionLabelFormat{nolabel}{}
    \captionsetup{labelformat=nolabel}

    \usepackage{adjustbox} % Used to constrain images to a maximum size 
    \usepackage{xcolor} % Allow colors to be defined
    \usepackage{enumerate} % Needed for markdown enumerations to work
    \usepackage{geometry} % Used to adjust the document margins
    \usepackage{amsmath} % Equations
    \usepackage{amssymb} % Equations
    \usepackage{textcomp} % defines textquotesingle
    % Hack from http://tex.stackexchange.com/a/47451/13684:
    \AtBeginDocument{%
        \def\PYZsq{\textquotesingle}% Upright quotes in Pygmentized code
    }
    \usepackage{upquote} % Upright quotes for verbatim code
    \usepackage{eurosym} % defines \euro
    \usepackage[mathletters]{ucs} % Extended unicode (utf-8) support
    \usepackage[utf8x]{inputenc} % Allow utf-8 characters in the tex document
    \usepackage{fancyvrb} % verbatim replacement that allows latex
    \usepackage{grffile} % extends the file name processing of package graphics 
                         % to support a larger range 
    % The hyperref package gives us a pdf with properly built
    % internal navigation ('pdf bookmarks' for the table of contents,
    % internal cross-reference links, web links for URLs, etc.)
    \usepackage{hyperref}
    \usepackage{longtable} % longtable support required by pandoc >1.10
    \usepackage{booktabs}  % table support for pandoc > 1.12.2
    \usepackage[inline]{enumitem} % IRkernel/repr support (it uses the enumerate* environment)
    \usepackage[normalem]{ulem} % ulem is needed to support strikethroughs (\sout)
                                % normalem makes italics be italics, not underlines
    

    
    
    % Colors for the hyperref package
    \definecolor{urlcolor}{rgb}{0,.145,.698}
    \definecolor{linkcolor}{rgb}{.71,0.21,0.01}
    \definecolor{citecolor}{rgb}{.12,.54,.11}

    % ANSI colors
    \definecolor{ansi-black}{HTML}{3E424D}
    \definecolor{ansi-black-intense}{HTML}{282C36}
    \definecolor{ansi-red}{HTML}{E75C58}
    \definecolor{ansi-red-intense}{HTML}{B22B31}
    \definecolor{ansi-green}{HTML}{00A250}
    \definecolor{ansi-green-intense}{HTML}{007427}
    \definecolor{ansi-yellow}{HTML}{DDB62B}
    \definecolor{ansi-yellow-intense}{HTML}{B27D12}
    \definecolor{ansi-blue}{HTML}{208FFB}
    \definecolor{ansi-blue-intense}{HTML}{0065CA}
    \definecolor{ansi-magenta}{HTML}{D160C4}
    \definecolor{ansi-magenta-intense}{HTML}{A03196}
    \definecolor{ansi-cyan}{HTML}{60C6C8}
    \definecolor{ansi-cyan-intense}{HTML}{258F8F}
    \definecolor{ansi-white}{HTML}{C5C1B4}
    \definecolor{ansi-white-intense}{HTML}{A1A6B2}

    % commands and environments needed by pandoc snippets
    % extracted from the output of `pandoc -s`
    \providecommand{\tightlist}{%
      \setlength{\itemsep}{0pt}\setlength{\parskip}{0pt}}
    \DefineVerbatimEnvironment{Highlighting}{Verbatim}{commandchars=\\\{\}}
    % Add ',fontsize=\small' for more characters per line
    \newenvironment{Shaded}{}{}
    \newcommand{\KeywordTok}[1]{\textcolor[rgb]{0.00,0.44,0.13}{\textbf{{#1}}}}
    \newcommand{\DataTypeTok}[1]{\textcolor[rgb]{0.56,0.13,0.00}{{#1}}}
    \newcommand{\DecValTok}[1]{\textcolor[rgb]{0.25,0.63,0.44}{{#1}}}
    \newcommand{\BaseNTok}[1]{\textcolor[rgb]{0.25,0.63,0.44}{{#1}}}
    \newcommand{\FloatTok}[1]{\textcolor[rgb]{0.25,0.63,0.44}{{#1}}}
    \newcommand{\CharTok}[1]{\textcolor[rgb]{0.25,0.44,0.63}{{#1}}}
    \newcommand{\StringTok}[1]{\textcolor[rgb]{0.25,0.44,0.63}{{#1}}}
    \newcommand{\CommentTok}[1]{\textcolor[rgb]{0.38,0.63,0.69}{\textit{{#1}}}}
    \newcommand{\OtherTok}[1]{\textcolor[rgb]{0.00,0.44,0.13}{{#1}}}
    \newcommand{\AlertTok}[1]{\textcolor[rgb]{1.00,0.00,0.00}{\textbf{{#1}}}}
    \newcommand{\FunctionTok}[1]{\textcolor[rgb]{0.02,0.16,0.49}{{#1}}}
    \newcommand{\RegionMarkerTok}[1]{{#1}}
    \newcommand{\ErrorTok}[1]{\textcolor[rgb]{1.00,0.00,0.00}{\textbf{{#1}}}}
    \newcommand{\NormalTok}[1]{{#1}}
    
    % Additional commands for more recent versions of Pandoc
    \newcommand{\ConstantTok}[1]{\textcolor[rgb]{0.53,0.00,0.00}{{#1}}}
    \newcommand{\SpecialCharTok}[1]{\textcolor[rgb]{0.25,0.44,0.63}{{#1}}}
    \newcommand{\VerbatimStringTok}[1]{\textcolor[rgb]{0.25,0.44,0.63}{{#1}}}
    \newcommand{\SpecialStringTok}[1]{\textcolor[rgb]{0.73,0.40,0.53}{{#1}}}
    \newcommand{\ImportTok}[1]{{#1}}
    \newcommand{\DocumentationTok}[1]{\textcolor[rgb]{0.73,0.13,0.13}{\textit{{#1}}}}
    \newcommand{\AnnotationTok}[1]{\textcolor[rgb]{0.38,0.63,0.69}{\textbf{\textit{{#1}}}}}
    \newcommand{\CommentVarTok}[1]{\textcolor[rgb]{0.38,0.63,0.69}{\textbf{\textit{{#1}}}}}
    \newcommand{\VariableTok}[1]{\textcolor[rgb]{0.10,0.09,0.49}{{#1}}}
    \newcommand{\ControlFlowTok}[1]{\textcolor[rgb]{0.00,0.44,0.13}{\textbf{{#1}}}}
    \newcommand{\OperatorTok}[1]{\textcolor[rgb]{0.40,0.40,0.40}{{#1}}}
    \newcommand{\BuiltInTok}[1]{{#1}}
    \newcommand{\ExtensionTok}[1]{{#1}}
    \newcommand{\PreprocessorTok}[1]{\textcolor[rgb]{0.74,0.48,0.00}{{#1}}}
    \newcommand{\AttributeTok}[1]{\textcolor[rgb]{0.49,0.56,0.16}{{#1}}}
    \newcommand{\InformationTok}[1]{\textcolor[rgb]{0.38,0.63,0.69}{\textbf{\textit{{#1}}}}}
    \newcommand{\WarningTok}[1]{\textcolor[rgb]{0.38,0.63,0.69}{\textbf{\textit{{#1}}}}}
    
    
    % Define a nice break command that doesn't care if a line doesn't already
    % exist.
    \def\br{\hspace*{\fill} \\* }
    % Math Jax compatability definitions
    \def\gt{>}
    \def\lt{<}
    % Document parameters
    \title{INFORME\_TP1\_REENTREGA}
    
    
    

    % Pygments definitions
    
\makeatletter
\def\PY@reset{\let\PY@it=\relax \let\PY@bf=\relax%
    \let\PY@ul=\relax \let\PY@tc=\relax%
    \let\PY@bc=\relax \let\PY@ff=\relax}
\def\PY@tok#1{\csname PY@tok@#1\endcsname}
\def\PY@toks#1+{\ifx\relax#1\empty\else%
    \PY@tok{#1}\expandafter\PY@toks\fi}
\def\PY@do#1{\PY@bc{\PY@tc{\PY@ul{%
    \PY@it{\PY@bf{\PY@ff{#1}}}}}}}
\def\PY#1#2{\PY@reset\PY@toks#1+\relax+\PY@do{#2}}

\expandafter\def\csname PY@tok@gd\endcsname{\def\PY@tc##1{\textcolor[rgb]{0.63,0.00,0.00}{##1}}}
\expandafter\def\csname PY@tok@gu\endcsname{\let\PY@bf=\textbf\def\PY@tc##1{\textcolor[rgb]{0.50,0.00,0.50}{##1}}}
\expandafter\def\csname PY@tok@gt\endcsname{\def\PY@tc##1{\textcolor[rgb]{0.00,0.27,0.87}{##1}}}
\expandafter\def\csname PY@tok@gs\endcsname{\let\PY@bf=\textbf}
\expandafter\def\csname PY@tok@gr\endcsname{\def\PY@tc##1{\textcolor[rgb]{1.00,0.00,0.00}{##1}}}
\expandafter\def\csname PY@tok@cm\endcsname{\let\PY@it=\textit\def\PY@tc##1{\textcolor[rgb]{0.25,0.50,0.50}{##1}}}
\expandafter\def\csname PY@tok@vg\endcsname{\def\PY@tc##1{\textcolor[rgb]{0.10,0.09,0.49}{##1}}}
\expandafter\def\csname PY@tok@vi\endcsname{\def\PY@tc##1{\textcolor[rgb]{0.10,0.09,0.49}{##1}}}
\expandafter\def\csname PY@tok@vm\endcsname{\def\PY@tc##1{\textcolor[rgb]{0.10,0.09,0.49}{##1}}}
\expandafter\def\csname PY@tok@mh\endcsname{\def\PY@tc##1{\textcolor[rgb]{0.40,0.40,0.40}{##1}}}
\expandafter\def\csname PY@tok@cs\endcsname{\let\PY@it=\textit\def\PY@tc##1{\textcolor[rgb]{0.25,0.50,0.50}{##1}}}
\expandafter\def\csname PY@tok@ge\endcsname{\let\PY@it=\textit}
\expandafter\def\csname PY@tok@vc\endcsname{\def\PY@tc##1{\textcolor[rgb]{0.10,0.09,0.49}{##1}}}
\expandafter\def\csname PY@tok@il\endcsname{\def\PY@tc##1{\textcolor[rgb]{0.40,0.40,0.40}{##1}}}
\expandafter\def\csname PY@tok@go\endcsname{\def\PY@tc##1{\textcolor[rgb]{0.53,0.53,0.53}{##1}}}
\expandafter\def\csname PY@tok@cp\endcsname{\def\PY@tc##1{\textcolor[rgb]{0.74,0.48,0.00}{##1}}}
\expandafter\def\csname PY@tok@gi\endcsname{\def\PY@tc##1{\textcolor[rgb]{0.00,0.63,0.00}{##1}}}
\expandafter\def\csname PY@tok@gh\endcsname{\let\PY@bf=\textbf\def\PY@tc##1{\textcolor[rgb]{0.00,0.00,0.50}{##1}}}
\expandafter\def\csname PY@tok@ni\endcsname{\let\PY@bf=\textbf\def\PY@tc##1{\textcolor[rgb]{0.60,0.60,0.60}{##1}}}
\expandafter\def\csname PY@tok@nl\endcsname{\def\PY@tc##1{\textcolor[rgb]{0.63,0.63,0.00}{##1}}}
\expandafter\def\csname PY@tok@nn\endcsname{\let\PY@bf=\textbf\def\PY@tc##1{\textcolor[rgb]{0.00,0.00,1.00}{##1}}}
\expandafter\def\csname PY@tok@no\endcsname{\def\PY@tc##1{\textcolor[rgb]{0.53,0.00,0.00}{##1}}}
\expandafter\def\csname PY@tok@na\endcsname{\def\PY@tc##1{\textcolor[rgb]{0.49,0.56,0.16}{##1}}}
\expandafter\def\csname PY@tok@nb\endcsname{\def\PY@tc##1{\textcolor[rgb]{0.00,0.50,0.00}{##1}}}
\expandafter\def\csname PY@tok@nc\endcsname{\let\PY@bf=\textbf\def\PY@tc##1{\textcolor[rgb]{0.00,0.00,1.00}{##1}}}
\expandafter\def\csname PY@tok@nd\endcsname{\def\PY@tc##1{\textcolor[rgb]{0.67,0.13,1.00}{##1}}}
\expandafter\def\csname PY@tok@ne\endcsname{\let\PY@bf=\textbf\def\PY@tc##1{\textcolor[rgb]{0.82,0.25,0.23}{##1}}}
\expandafter\def\csname PY@tok@nf\endcsname{\def\PY@tc##1{\textcolor[rgb]{0.00,0.00,1.00}{##1}}}
\expandafter\def\csname PY@tok@si\endcsname{\let\PY@bf=\textbf\def\PY@tc##1{\textcolor[rgb]{0.73,0.40,0.53}{##1}}}
\expandafter\def\csname PY@tok@s2\endcsname{\def\PY@tc##1{\textcolor[rgb]{0.73,0.13,0.13}{##1}}}
\expandafter\def\csname PY@tok@nt\endcsname{\let\PY@bf=\textbf\def\PY@tc##1{\textcolor[rgb]{0.00,0.50,0.00}{##1}}}
\expandafter\def\csname PY@tok@nv\endcsname{\def\PY@tc##1{\textcolor[rgb]{0.10,0.09,0.49}{##1}}}
\expandafter\def\csname PY@tok@s1\endcsname{\def\PY@tc##1{\textcolor[rgb]{0.73,0.13,0.13}{##1}}}
\expandafter\def\csname PY@tok@dl\endcsname{\def\PY@tc##1{\textcolor[rgb]{0.73,0.13,0.13}{##1}}}
\expandafter\def\csname PY@tok@ch\endcsname{\let\PY@it=\textit\def\PY@tc##1{\textcolor[rgb]{0.25,0.50,0.50}{##1}}}
\expandafter\def\csname PY@tok@m\endcsname{\def\PY@tc##1{\textcolor[rgb]{0.40,0.40,0.40}{##1}}}
\expandafter\def\csname PY@tok@gp\endcsname{\let\PY@bf=\textbf\def\PY@tc##1{\textcolor[rgb]{0.00,0.00,0.50}{##1}}}
\expandafter\def\csname PY@tok@sh\endcsname{\def\PY@tc##1{\textcolor[rgb]{0.73,0.13,0.13}{##1}}}
\expandafter\def\csname PY@tok@ow\endcsname{\let\PY@bf=\textbf\def\PY@tc##1{\textcolor[rgb]{0.67,0.13,1.00}{##1}}}
\expandafter\def\csname PY@tok@sx\endcsname{\def\PY@tc##1{\textcolor[rgb]{0.00,0.50,0.00}{##1}}}
\expandafter\def\csname PY@tok@bp\endcsname{\def\PY@tc##1{\textcolor[rgb]{0.00,0.50,0.00}{##1}}}
\expandafter\def\csname PY@tok@c1\endcsname{\let\PY@it=\textit\def\PY@tc##1{\textcolor[rgb]{0.25,0.50,0.50}{##1}}}
\expandafter\def\csname PY@tok@fm\endcsname{\def\PY@tc##1{\textcolor[rgb]{0.00,0.00,1.00}{##1}}}
\expandafter\def\csname PY@tok@o\endcsname{\def\PY@tc##1{\textcolor[rgb]{0.40,0.40,0.40}{##1}}}
\expandafter\def\csname PY@tok@kc\endcsname{\let\PY@bf=\textbf\def\PY@tc##1{\textcolor[rgb]{0.00,0.50,0.00}{##1}}}
\expandafter\def\csname PY@tok@c\endcsname{\let\PY@it=\textit\def\PY@tc##1{\textcolor[rgb]{0.25,0.50,0.50}{##1}}}
\expandafter\def\csname PY@tok@mf\endcsname{\def\PY@tc##1{\textcolor[rgb]{0.40,0.40,0.40}{##1}}}
\expandafter\def\csname PY@tok@err\endcsname{\def\PY@bc##1{\setlength{\fboxsep}{0pt}\fcolorbox[rgb]{1.00,0.00,0.00}{1,1,1}{\strut ##1}}}
\expandafter\def\csname PY@tok@mb\endcsname{\def\PY@tc##1{\textcolor[rgb]{0.40,0.40,0.40}{##1}}}
\expandafter\def\csname PY@tok@ss\endcsname{\def\PY@tc##1{\textcolor[rgb]{0.10,0.09,0.49}{##1}}}
\expandafter\def\csname PY@tok@sr\endcsname{\def\PY@tc##1{\textcolor[rgb]{0.73,0.40,0.53}{##1}}}
\expandafter\def\csname PY@tok@mo\endcsname{\def\PY@tc##1{\textcolor[rgb]{0.40,0.40,0.40}{##1}}}
\expandafter\def\csname PY@tok@kd\endcsname{\let\PY@bf=\textbf\def\PY@tc##1{\textcolor[rgb]{0.00,0.50,0.00}{##1}}}
\expandafter\def\csname PY@tok@mi\endcsname{\def\PY@tc##1{\textcolor[rgb]{0.40,0.40,0.40}{##1}}}
\expandafter\def\csname PY@tok@kn\endcsname{\let\PY@bf=\textbf\def\PY@tc##1{\textcolor[rgb]{0.00,0.50,0.00}{##1}}}
\expandafter\def\csname PY@tok@cpf\endcsname{\let\PY@it=\textit\def\PY@tc##1{\textcolor[rgb]{0.25,0.50,0.50}{##1}}}
\expandafter\def\csname PY@tok@kr\endcsname{\let\PY@bf=\textbf\def\PY@tc##1{\textcolor[rgb]{0.00,0.50,0.00}{##1}}}
\expandafter\def\csname PY@tok@s\endcsname{\def\PY@tc##1{\textcolor[rgb]{0.73,0.13,0.13}{##1}}}
\expandafter\def\csname PY@tok@kp\endcsname{\def\PY@tc##1{\textcolor[rgb]{0.00,0.50,0.00}{##1}}}
\expandafter\def\csname PY@tok@w\endcsname{\def\PY@tc##1{\textcolor[rgb]{0.73,0.73,0.73}{##1}}}
\expandafter\def\csname PY@tok@kt\endcsname{\def\PY@tc##1{\textcolor[rgb]{0.69,0.00,0.25}{##1}}}
\expandafter\def\csname PY@tok@sc\endcsname{\def\PY@tc##1{\textcolor[rgb]{0.73,0.13,0.13}{##1}}}
\expandafter\def\csname PY@tok@sb\endcsname{\def\PY@tc##1{\textcolor[rgb]{0.73,0.13,0.13}{##1}}}
\expandafter\def\csname PY@tok@sa\endcsname{\def\PY@tc##1{\textcolor[rgb]{0.73,0.13,0.13}{##1}}}
\expandafter\def\csname PY@tok@k\endcsname{\let\PY@bf=\textbf\def\PY@tc##1{\textcolor[rgb]{0.00,0.50,0.00}{##1}}}
\expandafter\def\csname PY@tok@se\endcsname{\let\PY@bf=\textbf\def\PY@tc##1{\textcolor[rgb]{0.73,0.40,0.13}{##1}}}
\expandafter\def\csname PY@tok@sd\endcsname{\let\PY@it=\textit\def\PY@tc##1{\textcolor[rgb]{0.73,0.13,0.13}{##1}}}

\def\PYZbs{\char`\\}
\def\PYZus{\char`\_}
\def\PYZob{\char`\{}
\def\PYZcb{\char`\}}
\def\PYZca{\char`\^}
\def\PYZam{\char`\&}
\def\PYZlt{\char`\<}
\def\PYZgt{\char`\>}
\def\PYZsh{\char`\#}
\def\PYZpc{\char`\%}
\def\PYZdl{\char`\$}
\def\PYZhy{\char`\-}
\def\PYZsq{\char`\'}
\def\PYZdq{\char`\"}
\def\PYZti{\char`\~}
% for compatibility with earlier versions
\def\PYZat{@}
\def\PYZlb{[}
\def\PYZrb{]}
\makeatother


    % Exact colors from NB
    \definecolor{incolor}{rgb}{0.0, 0.0, 0.5}
    \definecolor{outcolor}{rgb}{0.545, 0.0, 0.0}



    
    % Prevent overflowing lines due to hard-to-break entities
    \sloppy 
    % Setup hyperref package
    \hypersetup{
      breaklinks=true,  % so long urls are correctly broken across lines
      colorlinks=true,
      urlcolor=urlcolor,
      linkcolor=linkcolor,
      citecolor=citecolor,
      }
    % Slightly bigger margins than the latex defaults
    
    \geometry{verbose,tmargin=1in,bmargin=1in,lmargin=1in,rmargin=1in}
    
    

    \begin{document}
    
    
    \maketitle
    
    

    
    \hypertarget{section}{%
\subsection{======================================================================}\label{section}}

\hypertarget{organizaciuxf3n-de-datos}{%
\section{75.06/95.58 Organización de
Datos}\label{organizaciuxf3n-de-datos}}

\hypertarget{primer-cuatrimestre-de-2018}{%
\section{Primer Cuatrimestre de
2018}\label{primer-cuatrimestre-de-2018}}

\hypertarget{trabajo-pruxe1ctico-1-anuxe1lisis-exploratorio}{%
\section{Trabajo Práctico 1: Análisis
Exploratorio}\label{trabajo-pruxe1ctico-1-anuxe1lisis-exploratorio}}

\hypertarget{integrantes}{%
\subsection{Integrantes:}\label{integrantes}}

Marcelo Iannuzzi

Gabriel La Torre

Andrés Silvestri

\hypertarget{section-1}{%
\subsection{======================================================================}\label{section-1}}

\hypertarget{introducciuxf3n}{%
\section{Introducción}\label{introducciuxf3n}}

Este trabajo está enfocado en hacer un primer análisis de los datos
ofrecidos por la empresa Navent, de manera que encontremos
particularidades que puedan ser de interés para dicha entidad.

Siendo este el caso, lo primero que vamos a hacer es interiorizarnos de
uno de los pilares que tiene la ciencia de datos, el negocio.

\hypertarget{de-quuxe9-se-encarga-navent}{%
\section{¿De qué se encarga Navent?}\label{de-quuxe9-se-encarga-navent}}

Navent es una empresa que tiene dos misiones muy bien marcadas que se
traducen en dos negocios distintos: ``Ayudamos a que las personas logren
dos de los anhelos más importantes de la vida. Encontrar un Empleo y un
Hogar.''

Es decir que un negocio son los recursos humanos y el otro negocio es el
negocio de bienes raíces.

Dicho esto, primero haremos un primer análisis de los datos recibidos
para saber con qué información lidiamos y empezaremos a proponernos
preguntas que puedan ser de utilidad para la empresa. Posteriormente
intentaremos buscar respuesta a esas preguntas en base a los datos
obtenidos y propondremos una visualización que ayude a su rápido
ententimiento.

    \hypertarget{section}{%
\subsection{======================================================================}\label{section}}

\hypertarget{organizaciuxf3n-del-informe}{%
\section{Organización del Informe:}\label{organizaciuxf3n-del-informe}}

A continuación enunciaremos en primer instancia como tendremos dividido
el informe que llevaremos a cabo. Lo primero será el armado de los
datos, luego haremos análisis sobre los postulantes, los avisos por
separado y luego junto a la información provista de las visitas y los
postulaciones un análisis que englobe todo esto. Con esta idea tratamo
de abordar en primer instancia la información que tenemos por separado,
ver que se puede obtener de cada parte y luego unificarla para poder
trabajarla unificando criterios.

\hypertarget{importaciuxf3n-y-estructura-de-los-datos}{%
\subsection{1 - Importación y estructura de los
datos:}\label{importaciuxf3n-y-estructura-de-los-datos}}

\begin{verbatim}
1.1 - Configuración básica y obtención de los datasets.
1.2 - Procesamiento general de las información y armado básico.
1.3 - Resumen y cuestiones a tener en cuenta sobre el informe.
\end{verbatim}

\hypertarget{anuxe1lisis-sobre-los-postulantes-registrados}{%
\subsection{2 - Análisis sobre los postulantes
registrados:}\label{anuxe1lisis-sobre-los-postulantes-registrados}}

\begin{verbatim}
2.1 - ¿Cuales son los rangos de edad que más comunes entre los usuarios registrados?
2.2 - ¿Qué nivel de especialización tienen estos postulantes?
2.3 - ¿Cual es la proporción entre los géneros de los usuarios registrados?
2.4 - ¿Hay relevancia por parte de los usuarios que no han registrado su género?
\end{verbatim}

\hypertarget{anuxe1lisis-sobre-los-avisos-publicados}{%
\subsection{3 - Análisis sobre los avisos
publicados:}\label{anuxe1lisis-sobre-los-avisos-publicados}}

\begin{verbatim}
3.1 - ¿Qué proporción tenemos entre las zonas que afectan los avisos?
3.2 - ¿Cuales son los tipo de trabajo más requeridos, full-time, part-time?
3.3 - ¿Qué proporción tenemos entre los avisos que están online y los offline?
3.4 - ¿Qué proporción tenemos avisos para Juniors y para Sr/Ssr?
\end{verbatim}

\hypertarget{anuxe1lisis-sobre-las-visitas}{%
\subsection{4 - Análisis sobre las
visitas:}\label{anuxe1lisis-sobre-las-visitas}}

\begin{verbatim}
4.1 - ¿Qué días de la semana se ven más avisos?
4.2 - ¿Cuales son las areas con mayor cantidad de visitas?
4.3 - ¿Cuales vendrían a ser las ingenierías con mayor cantidad de visitas?
4.4 - ¿Qué porcentaje de los avisos evaluados/visitados está online?
4.5 - ¿Cuales son las empresas que reciben mayor cantidad de visitas en sus avisos?
4.6 - ¿Qué género es el que más utiliza los servicios de la empresa?
\end{verbatim}

\hypertarget{anuxe1lisis-sobre-las-postulaciones}{%
\subsection{5 - Análisis sobre las
postulaciones:}\label{anuxe1lisis-sobre-las-postulaciones}}

\begin{verbatim}
5.1 - ¿Qué días de la semana se postula más gente?
5.2 - ¿Cuales son las areas con mayor cantidad de postulaciones?
5.3 - ¿Cuales vendrían a ser las ingenierías con mayor cantidad de postulaciones?
5.4 - ¿Qué porcentaje de los avisos donde se ha postulado la gente está online?
4.5 - ¿Cuales son las empresas con mayor cantidad de postulaciones en sus avisos?
5.6 - ¿Quienes se postulan más a los avisos? ¿Hombres o mujeres?
\end{verbatim}

\hypertarget{conclusiones-generales}{%
\subsection{6 - Conclusiones generales:}\label{conclusiones-generales}}

\begin{verbatim}
6.1 - Conclusiones sobre lo que hemos obtenido del informe.
6.2 - Repositorio GitHub.
\end{verbatim}

\hypertarget{section-1}{%
\subsection{======================================================================}\label{section-1}}

    \hypertarget{importaciuxf3n-y-estructura-de-los-datos}{%
\section{1 - Importación y estructura de los
datos:}\label{importaciuxf3n-y-estructura-de-los-datos}}

\hypertarget{configuraciuxf3n-buxe1sica-y-obtenciuxf3n-de-los-datasets.}{%
\subsection{1.1 - Configuración básica y obtención de los
datasets.}\label{configuraciuxf3n-buxe1sica-y-obtenciuxf3n-de-los-datasets.}}

    \begin{Verbatim}[commandchars=\\\{\}]
{\color{incolor}In [{\color{incolor}1}]:} \PY{c+c1}{\PYZsh{}\PYZsh{} IMPORTACIÓN GENERAL DE LIBRERIAS Y VISUALIZACIÓN DE DATOS (matplotlib y seaborn)}
        \PY{k+kn}{import} \PY{n+nn}{pandas} \PY{k+kn}{as} \PY{n+nn}{pd}
        \PY{k+kn}{import} \PY{n+nn}{numpy} \PY{k+kn}{as} \PY{n+nn}{np}
        \PY{k+kn}{import} \PY{n+nn}{matplotlib.pyplot} \PY{k+kn}{as} \PY{n+nn}{plt}
        \PY{k+kn}{import} \PY{n+nn}{seaborn} \PY{k+kn}{as} \PY{n+nn}{sns}
        \PY{k+kn}{import} \PY{n+nn}{datetime} \PY{k+kn}{as} \PY{n+nn}{DT}
        \PY{k+kn}{import} \PY{n+nn}{warnings}
        \PY{o}{\PYZpc{}}\PY{k}{matplotlib} inline
        \PY{n}{warnings}\PY{o}{.}\PY{n}{filterwarnings}\PY{p}{(}\PY{l+s+s1}{\PYZsq{}}\PY{l+s+s1}{ignore}\PY{l+s+s1}{\PYZsq{}}\PY{p}{)}
        \PY{n}{plt}\PY{o}{.}\PY{n}{style}\PY{o}{.}\PY{n}{use}\PY{p}{(}\PY{l+s+s1}{\PYZsq{}}\PY{l+s+s1}{default}\PY{l+s+s1}{\PYZsq{}}\PY{p}{)} 
        \PY{n}{sns}\PY{o}{.}\PY{n}{set}\PY{p}{(}\PY{n}{style}\PY{o}{=}\PY{l+s+s2}{\PYZdq{}}\PY{l+s+s2}{whitegrid}\PY{l+s+s2}{\PYZdq{}}\PY{p}{)} 
        \PY{c+c1}{\PYZsh{}plt.rcParams[\PYZsq{}figure.figsize\PYZsq{}] = (20, 10)}
\end{Verbatim}


    \begin{Verbatim}[commandchars=\\\{\}]
{\color{incolor}In [{\color{incolor}2}]:} \PY{c+c1}{\PYZsh{}\PYZsh{} OBTENEMOS TODA LA INFORMACIÓN DE LOS DIFERENTES CSV.}
        \PY{n}{nivel\PYZus{}educativo} \PY{o}{=} \PY{n}{pd}\PY{o}{.}\PY{n}{read\PYZus{}csv}\PY{p}{(}\PY{l+s+s1}{\PYZsq{}}\PY{l+s+s1}{fiuba\PYZus{}1\PYZus{}postulantes\PYZus{}educacion.csv}\PY{l+s+s1}{\PYZsq{}}\PY{p}{)}
        \PY{n}{nacimiento\PYZus{}genero} \PY{o}{=} \PY{n}{pd}\PY{o}{.}\PY{n}{read\PYZus{}csv}\PY{p}{(}\PY{l+s+s1}{\PYZsq{}}\PY{l+s+s1}{fiuba\PYZus{}2\PYZus{}postulantes\PYZus{}genero\PYZus{}y\PYZus{}edad.csv}\PY{l+s+s1}{\PYZsq{}}\PY{p}{)}
        \PY{n}{vistas\PYZus{}avisos} \PY{o}{=} \PY{n}{pd}\PY{o}{.}\PY{n}{read\PYZus{}csv}\PY{p}{(}\PY{l+s+s1}{\PYZsq{}}\PY{l+s+s1}{fiuba\PYZus{}3\PYZus{}vistas.csv}\PY{l+s+s1}{\PYZsq{}}\PY{p}{)}
        \PY{n}{postulaciones} \PY{o}{=} \PY{n}{pd}\PY{o}{.}\PY{n}{read\PYZus{}csv}\PY{p}{(}\PY{l+s+s1}{\PYZsq{}}\PY{l+s+s1}{fiuba\PYZus{}4\PYZus{}postulaciones.csv}\PY{l+s+s1}{\PYZsq{}}\PY{p}{)}
        \PY{n}{avisos\PYZus{}online} \PY{o}{=} \PY{n}{pd}\PY{o}{.}\PY{n}{read\PYZus{}csv}\PY{p}{(}\PY{l+s+s1}{\PYZsq{}}\PY{l+s+s1}{fiuba\PYZus{}5\PYZus{}avisos\PYZus{}online.csv}\PY{l+s+s1}{\PYZsq{}}\PY{p}{)}
        \PY{n}{avisos\PYZus{}detalle} \PY{o}{=} \PY{n}{pd}\PY{o}{.}\PY{n}{read\PYZus{}csv}\PY{p}{(}\PY{l+s+s1}{\PYZsq{}}\PY{l+s+s1}{fiuba\PYZus{}6\PYZus{}avisos\PYZus{}detalle.csv}\PY{l+s+s1}{\PYZsq{}}\PY{p}{)}
\end{Verbatim}


    \hypertarget{section}{%
\subsection{======================================================================}\label{section}}

\hypertarget{procesamiento-general-de-las-informaciuxf3n-y-armado-buxe1sico.}{%
\subsection{1.2 - Procesamiento general de las información y armado
básico.}\label{procesamiento-general-de-las-informaciuxf3n-y-armado-buxe1sico.}}

Para cada una de las diferentes fuentes de datos haremos diversos
tratamientos, tanto para purificar los datos como para filtrar aquellos
que más nos ineresan, se hará una breve explicación sobre lo que se
trata de obtener en cada caso a modo de resumen, como así también se
comentara cada linea de código para que se pueda interpretar qué es lo
que se decidió hacer en cada caso.

\hypertarget{postulantes}{%
\subsubsection{1.2.1 - Postulantes:}\label{postulantes}}

Lo primero que vamos a tratar es ver que hay registros repetidos (debido
a que cada postulante tiene varios niveles educativos registrados), por
otro lado haremos un cálculo de la edad que posee cada postulante y los
filtraremos por aquellos que están en un rango de edad coherente (no al
trabajo infantil y tampoco a las personas que sobrepasan los cien años).
Sobre la edad decidimos agregar una columna que nos represente el rango
al que pertenece la persona, siendo un criterio arbitrario y subjetivo
pero que nos servirá para análisis posteriores. También hacemos una
pequeña limpieza sobre los niveles de educación como así también un
filtrado de duplicados cuando no nos interese el apartado educativo.

    \begin{Verbatim}[commandchars=\\\{\}]
{\color{incolor}In [{\color{incolor}3}]:} \PY{c+c1}{\PYZsh{}\PYZsh{} UNIFICACIÓN BÁSICA SOBRE LA INFORMACIÓN DE LOS POSTULANTES.}
        \PY{n}{postulantes} \PY{o}{=} \PY{n}{nivel\PYZus{}educativo}\PY{o}{.}\PY{n}{merge}\PY{p}{(}\PY{n}{nacimiento\PYZus{}genero}\PY{p}{,}\PY{n}{left\PYZus{}on}\PY{o}{=}\PY{l+s+s1}{\PYZsq{}}\PY{l+s+s1}{idpostulante}\PY{l+s+s1}{\PYZsq{}}\PY{p}{,} \PY{n}{right\PYZus{}on}\PY{o}{=}\PY{l+s+s1}{\PYZsq{}}\PY{l+s+s1}{idpostulante}\PY{l+s+s1}{\PYZsq{}}\PY{p}{,}\PY{n}{how}\PY{o}{=}\PY{l+s+s1}{\PYZsq{}}\PY{l+s+s1}{outer}\PY{l+s+s1}{\PYZsq{}}\PY{p}{)}
        \PY{c+c1}{\PYZsh{}\PYZsh{} TRANSFORMAMOS A DATETIME LA FECHA DE NACIMIENTO DE LOS POSTULANTES.}
        \PY{n}{postulantes}\PY{p}{[}\PY{l+s+s1}{\PYZsq{}}\PY{l+s+s1}{fechanacimiento}\PY{l+s+s1}{\PYZsq{}}\PY{p}{]} \PY{o}{=} \PY{n}{pd}\PY{o}{.}\PY{n}{to\PYZus{}datetime}\PY{p}{(}\PY{n}{postulantes}\PY{p}{[}\PY{l+s+s1}{\PYZsq{}}\PY{l+s+s1}{fechanacimiento}\PY{l+s+s1}{\PYZsq{}}\PY{p}{]}\PY{p}{,} \PY{n}{errors} \PY{o}{=} \PY{l+s+s1}{\PYZsq{}}\PY{l+s+s1}{coerce}\PY{l+s+s1}{\PYZsq{}}\PY{p}{)}
        \PY{c+c1}{\PYZsh{}\PYZsh{} CALCULAMOS LA EDAD DE LOS POSTULANTES CON UN VALOR NUMÉRICO.}
        \PY{n}{now} \PY{o}{=} \PY{n}{pd}\PY{o}{.}\PY{n}{Timestamp}\PY{p}{(}\PY{n}{DT}\PY{o}{.}\PY{n}{datetime}\PY{o}{.}\PY{n}{now}\PY{p}{(}\PY{p}{)}\PY{p}{)}
        \PY{n}{postulantes}\PY{p}{[}\PY{l+s+s1}{\PYZsq{}}\PY{l+s+s1}{edad}\PY{l+s+s1}{\PYZsq{}}\PY{p}{]} \PY{o}{=} \PY{p}{(}\PY{n}{now} \PY{o}{\PYZhy{}} \PY{n}{postulantes}\PY{p}{[}\PY{l+s+s1}{\PYZsq{}}\PY{l+s+s1}{fechanacimiento}\PY{l+s+s1}{\PYZsq{}}\PY{p}{]}\PY{p}{)}\PY{o}{.}\PY{n}{astype}\PY{p}{(}\PY{l+s+s1}{\PYZsq{}}\PY{l+s+s1}{\PYZlt{}m8[Y]}\PY{l+s+s1}{\PYZsq{}}\PY{p}{)}
        \PY{c+c1}{\PYZsh{}\PYZsh{} SOLO CONSIDERAMOS COMO VÁLIDOS AQUELLOS POSTULANTES DE ENTRE 17 Y 100 AÑOS.}
        \PY{n}{postulantes} \PY{o}{=} \PY{n}{postulantes}\PY{p}{[}\PY{p}{(}\PY{n}{postulantes}\PY{p}{[}\PY{l+s+s1}{\PYZsq{}}\PY{l+s+s1}{edad}\PY{l+s+s1}{\PYZsq{}}\PY{p}{]} \PY{o}{\PYZgt{}} \PY{l+m+mi}{17}\PY{p}{)} \PY{o}{\PYZam{}} \PY{p}{(}\PY{n}{postulantes}\PY{p}{[}\PY{l+s+s1}{\PYZsq{}}\PY{l+s+s1}{edad}\PY{l+s+s1}{\PYZsq{}}\PY{p}{]} \PY{o}{\PYZlt{}} \PY{l+m+mi}{100}\PY{p}{)}\PY{p}{]}
        \PY{c+c1}{\PYZsh{}\PYZsh{} ARMAMOS UNA COLUMNA SOBRE EL RANGO DE EDAD QUE TIENEN LOS POSTULANTES.}
        \PY{n}{postulantes}\PY{p}{[}\PY{l+s+s1}{\PYZsq{}}\PY{l+s+s1}{rango\PYZus{}edad}\PY{l+s+s1}{\PYZsq{}}\PY{p}{]} \PY{o}{=} \PY{l+s+s1}{\PYZsq{}}\PY{l+s+s1}{abc}\PY{l+s+s1}{\PYZsq{}}
        \PY{n}{postulantes}\PY{o}{.}\PY{n}{loc}\PY{p}{[}\PY{n}{postulantes}\PY{o}{.}\PY{n}{edad} \PY{o}{\PYZlt{}} \PY{l+m+mi}{20}\PY{p}{,} \PY{l+s+s1}{\PYZsq{}}\PY{l+s+s1}{rango\PYZus{}edad}\PY{l+s+s1}{\PYZsq{}}\PY{p}{]} \PY{o}{=} \PY{l+s+s1}{\PYZsq{}}\PY{l+s+s1}{\PYZlt{} 20}\PY{l+s+s1}{\PYZsq{}}
        \PY{n}{postulantes}\PY{o}{.}\PY{n}{loc}\PY{p}{[}\PY{n}{postulantes}\PY{o}{.}\PY{n}{edad} \PY{o}{\PYZgt{}} \PY{l+m+mi}{65}\PY{p}{,} \PY{l+s+s1}{\PYZsq{}}\PY{l+s+s1}{rango\PYZus{}edad}\PY{l+s+s1}{\PYZsq{}}\PY{p}{]} \PY{o}{=} \PY{l+s+s1}{\PYZsq{}}\PY{l+s+s1}{\PYZgt{} 65}\PY{l+s+s1}{\PYZsq{}}
        \PY{n}{postulantes}\PY{o}{.}\PY{n}{loc}\PY{p}{[}\PY{n}{postulantes}\PY{o}{.}\PY{n}{rango\PYZus{}edad} \PY{o}{==} \PY{l+s+s1}{\PYZsq{}}\PY{l+s+s1}{abc}\PY{l+s+s1}{\PYZsq{}}\PY{p}{,} \PY{l+s+s1}{\PYZsq{}}\PY{l+s+s1}{rango\PYZus{}edad}\PY{l+s+s1}{\PYZsq{}}\PY{p}{]} \PY{o}{=} \PY{l+s+s1}{\PYZsq{}}\PY{l+s+s1}{\PYZgt{} 65}\PY{l+s+s1}{\PYZsq{}}
        \PY{c+c1}{\PYZsh{}\PYZsh{} ARMAMOS LOS DIFERENTES RANGOS DE EDAD QUE SE PUEDEN IR DANDO ENTRE LOS POSTULANTES.}
        \PY{n}{postulantes}\PY{o}{.}\PY{n}{loc}\PY{p}{[}\PY{p}{(}\PY{n}{postulantes}\PY{o}{.}\PY{n}{edad} \PY{o}{\PYZgt{}}\PY{o}{=} \PY{l+m+mi}{20}\PY{p}{)} \PY{o}{\PYZam{}} \PY{p}{(}\PY{n}{postulantes}\PY{o}{.}\PY{n}{edad} \PY{o}{\PYZlt{}}\PY{o}{=} \PY{l+m+mi}{24}\PY{p}{)}\PY{p}{,} \PY{l+s+s1}{\PYZsq{}}\PY{l+s+s1}{rango\PYZus{}edad}\PY{l+s+s1}{\PYZsq{}}\PY{p}{]} \PY{o}{=} \PY{l+s+s1}{\PYZsq{}}\PY{l+s+s1}{20 \PYZti{} 25}\PY{l+s+s1}{\PYZsq{}}
        \PY{n}{postulantes}\PY{o}{.}\PY{n}{loc}\PY{p}{[}\PY{p}{(}\PY{n}{postulantes}\PY{o}{.}\PY{n}{edad} \PY{o}{\PYZgt{}}\PY{o}{=} \PY{l+m+mi}{25}\PY{p}{)} \PY{o}{\PYZam{}} \PY{p}{(}\PY{n}{postulantes}\PY{o}{.}\PY{n}{edad} \PY{o}{\PYZlt{}}\PY{o}{=} \PY{l+m+mi}{29}\PY{p}{)}\PY{p}{,} \PY{l+s+s1}{\PYZsq{}}\PY{l+s+s1}{rango\PYZus{}edad}\PY{l+s+s1}{\PYZsq{}}\PY{p}{]} \PY{o}{=} \PY{l+s+s1}{\PYZsq{}}\PY{l+s+s1}{25 \PYZti{} 30}\PY{l+s+s1}{\PYZsq{}}
        \PY{n}{postulantes}\PY{o}{.}\PY{n}{loc}\PY{p}{[}\PY{p}{(}\PY{n}{postulantes}\PY{o}{.}\PY{n}{edad} \PY{o}{\PYZgt{}}\PY{o}{=} \PY{l+m+mi}{30}\PY{p}{)} \PY{o}{\PYZam{}} \PY{p}{(}\PY{n}{postulantes}\PY{o}{.}\PY{n}{edad} \PY{o}{\PYZlt{}}\PY{o}{=} \PY{l+m+mi}{34}\PY{p}{)}\PY{p}{,} \PY{l+s+s1}{\PYZsq{}}\PY{l+s+s1}{rango\PYZus{}edad}\PY{l+s+s1}{\PYZsq{}}\PY{p}{]} \PY{o}{=} \PY{l+s+s1}{\PYZsq{}}\PY{l+s+s1}{30 \PYZti{} 35}\PY{l+s+s1}{\PYZsq{}}
        \PY{n}{postulantes}\PY{o}{.}\PY{n}{loc}\PY{p}{[}\PY{p}{(}\PY{n}{postulantes}\PY{o}{.}\PY{n}{edad} \PY{o}{\PYZgt{}}\PY{o}{=} \PY{l+m+mi}{35}\PY{p}{)} \PY{o}{\PYZam{}} \PY{p}{(}\PY{n}{postulantes}\PY{o}{.}\PY{n}{edad} \PY{o}{\PYZlt{}}\PY{o}{=} \PY{l+m+mi}{39}\PY{p}{)}\PY{p}{,} \PY{l+s+s1}{\PYZsq{}}\PY{l+s+s1}{rango\PYZus{}edad}\PY{l+s+s1}{\PYZsq{}}\PY{p}{]} \PY{o}{=} \PY{l+s+s1}{\PYZsq{}}\PY{l+s+s1}{35 \PYZti{} 40}\PY{l+s+s1}{\PYZsq{}}
        \PY{n}{postulantes}\PY{o}{.}\PY{n}{loc}\PY{p}{[}\PY{p}{(}\PY{n}{postulantes}\PY{o}{.}\PY{n}{edad} \PY{o}{\PYZgt{}}\PY{o}{=} \PY{l+m+mi}{40}\PY{p}{)} \PY{o}{\PYZam{}} \PY{p}{(}\PY{n}{postulantes}\PY{o}{.}\PY{n}{edad} \PY{o}{\PYZlt{}}\PY{o}{=} \PY{l+m+mi}{44}\PY{p}{)}\PY{p}{,} \PY{l+s+s1}{\PYZsq{}}\PY{l+s+s1}{rango\PYZus{}edad}\PY{l+s+s1}{\PYZsq{}}\PY{p}{]} \PY{o}{=} \PY{l+s+s1}{\PYZsq{}}\PY{l+s+s1}{40 \PYZti{} 45}\PY{l+s+s1}{\PYZsq{}}
        \PY{n}{postulantes}\PY{o}{.}\PY{n}{loc}\PY{p}{[}\PY{p}{(}\PY{n}{postulantes}\PY{o}{.}\PY{n}{edad} \PY{o}{\PYZgt{}}\PY{o}{=} \PY{l+m+mi}{45}\PY{p}{)} \PY{o}{\PYZam{}} \PY{p}{(}\PY{n}{postulantes}\PY{o}{.}\PY{n}{edad} \PY{o}{\PYZlt{}}\PY{o}{=} \PY{l+m+mi}{49}\PY{p}{)}\PY{p}{,} \PY{l+s+s1}{\PYZsq{}}\PY{l+s+s1}{rango\PYZus{}edad}\PY{l+s+s1}{\PYZsq{}}\PY{p}{]} \PY{o}{=} \PY{l+s+s1}{\PYZsq{}}\PY{l+s+s1}{45 \PYZti{} 50}\PY{l+s+s1}{\PYZsq{}}
        \PY{n}{postulantes}\PY{o}{.}\PY{n}{loc}\PY{p}{[}\PY{p}{(}\PY{n}{postulantes}\PY{o}{.}\PY{n}{edad} \PY{o}{\PYZgt{}}\PY{o}{=} \PY{l+m+mi}{50}\PY{p}{)} \PY{o}{\PYZam{}} \PY{p}{(}\PY{n}{postulantes}\PY{o}{.}\PY{n}{edad} \PY{o}{\PYZlt{}}\PY{o}{=} \PY{l+m+mi}{54}\PY{p}{)}\PY{p}{,} \PY{l+s+s1}{\PYZsq{}}\PY{l+s+s1}{rango\PYZus{}edad}\PY{l+s+s1}{\PYZsq{}}\PY{p}{]} \PY{o}{=} \PY{l+s+s1}{\PYZsq{}}\PY{l+s+s1}{50 \PYZti{} 55}\PY{l+s+s1}{\PYZsq{}}
        \PY{n}{postulantes}\PY{o}{.}\PY{n}{loc}\PY{p}{[}\PY{p}{(}\PY{n}{postulantes}\PY{o}{.}\PY{n}{edad} \PY{o}{\PYZgt{}}\PY{o}{=} \PY{l+m+mi}{55}\PY{p}{)} \PY{o}{\PYZam{}} \PY{p}{(}\PY{n}{postulantes}\PY{o}{.}\PY{n}{edad} \PY{o}{\PYZlt{}}\PY{o}{=} \PY{l+m+mi}{59}\PY{p}{)}\PY{p}{,} \PY{l+s+s1}{\PYZsq{}}\PY{l+s+s1}{rango\PYZus{}edad}\PY{l+s+s1}{\PYZsq{}}\PY{p}{]} \PY{o}{=} \PY{l+s+s1}{\PYZsq{}}\PY{l+s+s1}{55 \PYZti{} 60}\PY{l+s+s1}{\PYZsq{}}
        \PY{n}{postulantes}\PY{o}{.}\PY{n}{loc}\PY{p}{[}\PY{p}{(}\PY{n}{postulantes}\PY{o}{.}\PY{n}{edad} \PY{o}{\PYZgt{}}\PY{o}{=} \PY{l+m+mi}{60}\PY{p}{)} \PY{o}{\PYZam{}} \PY{p}{(}\PY{n}{postulantes}\PY{o}{.}\PY{n}{edad} \PY{o}{\PYZlt{}}\PY{o}{=} \PY{l+m+mi}{64}\PY{p}{)}\PY{p}{,} \PY{l+s+s1}{\PYZsq{}}\PY{l+s+s1}{rango\PYZus{}edad}\PY{l+s+s1}{\PYZsq{}}\PY{p}{]} \PY{o}{=} \PY{l+s+s1}{\PYZsq{}}\PY{l+s+s1}{60 \PYZti{} 65}\PY{l+s+s1}{\PYZsq{}}
        \PY{c+c1}{\PYZsh{}\PYZsh{} ANALIZAMOS LOS NIVELES EDUCATIVOS Y UNIFICAMOS CRITERIOS SOBRE ESTOS.}
        \PY{n}{postulantes}\PY{o}{.}\PY{n}{loc}\PY{p}{[}\PY{n}{postulantes}\PY{o}{.}\PY{n}{nombre} \PY{o}{==} \PY{l+s+s1}{\PYZsq{}}\PY{l+s+s1}{Terciario/Técnico}\PY{l+s+s1}{\PYZsq{}}\PY{p}{,} \PY{l+s+s1}{\PYZsq{}}\PY{l+s+s1}{nombre}\PY{l+s+s1}{\PYZsq{}}\PY{p}{]} \PY{o}{=} \PY{l+s+s1}{\PYZsq{}}\PY{l+s+s1}{Terciario}\PY{l+s+s1}{\PYZsq{}}
        \PY{n}{postulantes}\PY{o}{.}\PY{n}{loc}\PY{p}{[}\PY{n}{postulantes}\PY{o}{.}\PY{n}{nombre} \PY{o}{==} \PY{l+s+s1}{\PYZsq{}}\PY{l+s+s1}{Terciario/Tecnico}\PY{l+s+s1}{\PYZsq{}}\PY{p}{,} \PY{l+s+s1}{\PYZsq{}}\PY{l+s+s1}{nombre}\PY{l+s+s1}{\PYZsq{}}\PY{p}{]} \PY{o}{=} \PY{l+s+s1}{\PYZsq{}}\PY{l+s+s1}{Terciario}\PY{l+s+s1}{\PYZsq{}}
        \PY{c+c1}{\PYZsh{}\PYZsh{} PARA LOS ANÁLISIS DONDE EL NIVEL EDUCATIVO NO ES RELEVANTE HACEMOS UNA LIMPIEZA DE DUPLICADOS.}
        \PY{n}{postulantes\PYZus{}unicos} \PY{o}{=} \PY{n}{postulantes}\PY{o}{.}\PY{n}{drop\PYZus{}duplicates}\PY{p}{(}\PY{n}{subset}\PY{o}{=}\PY{l+s+s2}{\PYZdq{}}\PY{l+s+s2}{idpostulante}\PY{l+s+s2}{\PYZdq{}}\PY{p}{,} \PY{n}{keep}\PY{o}{=}\PY{l+s+s1}{\PYZsq{}}\PY{l+s+s1}{first}\PY{l+s+s1}{\PYZsq{}}\PY{p}{,} \PY{n}{inplace}\PY{o}{=}\PY{n+nb+bp}{False}\PY{p}{)}
\end{Verbatim}


    \hypertarget{avisos}{%
\subsubsection{1.2.2 - Avisos:}\label{avisos}}

Lo primero que vamos a tratar con respecto a estos datos es que hay unos
pocos que son fuera de CABA y GBA (exactamente 2) esto nos lleva a
eliminarlos porque pierde sentido su análisis. Por otro lado vemos que
hay varias formas de referirce al Gran Buenos Aires, lo que nos lleva a
unificarlo. Haremos un merge con los avisos online teniendo así toda la
información pertinente en una única fuente de datos, así también
podremos diferenciar cuando un aviso está online y cuando no.

    \begin{Verbatim}[commandchars=\\\{\}]
{\color{incolor}In [{\color{incolor}5}]:} \PY{c+c1}{\PYZsh{}\PYZsh{} LIMPIAMOS AQUELLOS AVISOS QUE NO PERTENECEN NI A CABA NI A GBA.}
        \PY{n}{avisos\PYZus{}detalle} \PY{o}{=} \PY{n}{avisos\PYZus{}detalle}\PY{p}{[}\PY{p}{(}\PY{n}{avisos\PYZus{}detalle}\PY{p}{[}\PY{l+s+s1}{\PYZsq{}}\PY{l+s+s1}{nombre\PYZus{}zona}\PY{l+s+s1}{\PYZsq{}}\PY{p}{]} \PY{o}{!=} \PY{l+s+s1}{\PYZsq{}}\PY{l+s+s1}{Buenos Aires (fuera de GBA)}\PY{l+s+s1}{\PYZsq{}}\PY{p}{)}\PY{p}{]}
        \PY{c+c1}{\PYZsh{}\PYZsh{} UNIFICAMOS CRITERIOS A LA HORA DE NOMBRAR AL GRAN BUENOS AIRES.}
        \PY{n}{avisos\PYZus{}detalle}\PY{o}{.}\PY{n}{loc}\PY{p}{[}\PY{n}{avisos\PYZus{}detalle}\PY{o}{.}\PY{n}{nombre\PYZus{}zona} \PY{o}{==} \PY{l+s+s1}{\PYZsq{}}\PY{l+s+s1}{GBA Oeste}\PY{l+s+s1}{\PYZsq{}}\PY{p}{,} \PY{l+s+s1}{\PYZsq{}}\PY{l+s+s1}{nombre\PYZus{}zona}\PY{l+s+s1}{\PYZsq{}}\PY{p}{]} \PY{o}{=} \PY{l+s+s1}{\PYZsq{}}\PY{l+s+s1}{Gran Buenos Aires}\PY{l+s+s1}{\PYZsq{}}
        \PY{c+c1}{\PYZsh{}\PYZsh{} AGREGAMOS UN DATO PARA DISTINGUIR LOS AVISOS ONLINE.}
        \PY{n}{avisos\PYZus{}online}\PY{p}{[}\PY{l+s+s1}{\PYZsq{}}\PY{l+s+s1}{esta\PYZus{}online}\PY{l+s+s1}{\PYZsq{}}\PY{p}{]} \PY{o}{=} \PY{l+s+s1}{\PYZsq{}}\PY{l+s+s1}{si}\PY{l+s+s1}{\PYZsq{}}
        \PY{c+c1}{\PYZsh{}\PYZsh{} UNIFICAMOS LOS DETALLES DE LOS AVISOS CON LOS AVISOS ONLINE.}
        \PY{n}{avisos\PYZus{}detalle\PYZus{}completo} \PY{o}{=} \PY{n}{avisos\PYZus{}detalle}\PY{o}{.}\PY{n}{merge}\PY{p}{(}\PY{n}{avisos\PYZus{}online}\PY{p}{,}\PY{n}{left\PYZus{}on}\PY{o}{=}\PY{l+s+s1}{\PYZsq{}}\PY{l+s+s1}{idaviso}\PY{l+s+s1}{\PYZsq{}}\PY{p}{,} \PY{n}{right\PYZus{}on}\PY{o}{=}\PY{l+s+s1}{\PYZsq{}}\PY{l+s+s1}{idaviso}\PY{l+s+s1}{\PYZsq{}}\PY{p}{,}\PY{n}{how}\PY{o}{=}\PY{l+s+s1}{\PYZsq{}}\PY{l+s+s1}{left}\PY{l+s+s1}{\PYZsq{}}\PY{p}{)}
        \PY{c+c1}{\PYZsh{}\PYZsh{} PONEMOS EN \PYZsq{}NO\PYZsq{} TODOS AQUELLOS AVISOS QUE NO ESTAN ONLINE.}
        \PY{n}{avisos\PYZus{}detalle\PYZus{}completo}\PY{o}{.}\PY{n}{loc}\PY{p}{[}\PY{n}{avisos\PYZus{}detalle\PYZus{}completo}\PY{o}{.}\PY{n}{esta\PYZus{}online} \PY{o}{!=} \PY{l+s+s1}{\PYZsq{}}\PY{l+s+s1}{si}\PY{l+s+s1}{\PYZsq{}}\PY{p}{,} \PY{l+s+s1}{\PYZsq{}}\PY{l+s+s1}{esta\PYZus{}online}\PY{l+s+s1}{\PYZsq{}}\PY{p}{]} \PY{o}{=} \PY{l+s+s1}{\PYZsq{}}\PY{l+s+s1}{no}\PY{l+s+s1}{\PYZsq{}}
        \PY{n}{avisos\PYZus{}detalle\PYZus{}completo}\PY{o}{.}\PY{n}{head}\PY{p}{(}\PY{p}{)}
        \PY{c+c1}{\PYZsh{}\PYZsh{} LIMPIAMOS LOS ACENTOS PARA EL NIVEL LABORAL QUE NOS TRAEN PROBLEMAS A LA HORA DE GRAFICAR.}
        \PY{n}{avisos\PYZus{}detalle\PYZus{}completo}\PY{p}{[}\PY{l+s+s1}{\PYZsq{}}\PY{l+s+s1}{nivel\PYZus{}laboral}\PY{l+s+s1}{\PYZsq{}}\PY{p}{]} \PY{o}{=} \PY{n}{avisos\PYZus{}detalle\PYZus{}completo}\PY{p}{[}\PY{l+s+s1}{\PYZsq{}}\PY{l+s+s1}{nivel\PYZus{}laboral}\PY{l+s+s1}{\PYZsq{}}\PY{p}{]}\PY{o}{.}\PY{n}{str}\PY{o}{.}\PY{n}{replace}\PY{p}{(}\PY{l+s+s2}{\PYZdq{}}\PY{l+s+s2}{á}\PY{l+s+s2}{\PYZdq{}}\PY{p}{,} \PY{l+s+s2}{\PYZdq{}}\PY{l+s+s2}{a}\PY{l+s+s2}{\PYZdq{}}\PY{p}{)}
        \PY{n}{avisos\PYZus{}detalle\PYZus{}completo}\PY{p}{[}\PY{l+s+s1}{\PYZsq{}}\PY{l+s+s1}{nivel\PYZus{}laboral}\PY{l+s+s1}{\PYZsq{}}\PY{p}{]} \PY{o}{=} \PY{n}{avisos\PYZus{}detalle\PYZus{}completo}\PY{p}{[}\PY{l+s+s1}{\PYZsq{}}\PY{l+s+s1}{nivel\PYZus{}laboral}\PY{l+s+s1}{\PYZsq{}}\PY{p}{]}\PY{o}{.}\PY{n}{str}\PY{o}{.}\PY{n}{replace}\PY{p}{(}\PY{l+s+s2}{\PYZdq{}}\PY{l+s+s2}{é}\PY{l+s+s2}{\PYZdq{}}\PY{p}{,} \PY{l+s+s2}{\PYZdq{}}\PY{l+s+s2}{e}\PY{l+s+s2}{\PYZdq{}}\PY{p}{)}
        \PY{n}{avisos\PYZus{}detalle\PYZus{}completo}\PY{p}{[}\PY{l+s+s1}{\PYZsq{}}\PY{l+s+s1}{nivel\PYZus{}laboral}\PY{l+s+s1}{\PYZsq{}}\PY{p}{]} \PY{o}{=} \PY{n}{avisos\PYZus{}detalle\PYZus{}completo}\PY{p}{[}\PY{l+s+s1}{\PYZsq{}}\PY{l+s+s1}{nivel\PYZus{}laboral}\PY{l+s+s1}{\PYZsq{}}\PY{p}{]}\PY{o}{.}\PY{n}{str}\PY{o}{.}\PY{n}{replace}\PY{p}{(}\PY{l+s+s2}{\PYZdq{}}\PY{l+s+s2}{í}\PY{l+s+s2}{\PYZdq{}}\PY{p}{,} \PY{l+s+s2}{\PYZdq{}}\PY{l+s+s2}{i}\PY{l+s+s2}{\PYZdq{}}\PY{p}{)}
        \PY{n}{avisos\PYZus{}detalle\PYZus{}completo}\PY{p}{[}\PY{l+s+s1}{\PYZsq{}}\PY{l+s+s1}{nivel\PYZus{}laboral}\PY{l+s+s1}{\PYZsq{}}\PY{p}{]} \PY{o}{=} \PY{n}{avisos\PYZus{}detalle\PYZus{}completo}\PY{p}{[}\PY{l+s+s1}{\PYZsq{}}\PY{l+s+s1}{nivel\PYZus{}laboral}\PY{l+s+s1}{\PYZsq{}}\PY{p}{]}\PY{o}{.}\PY{n}{str}\PY{o}{.}\PY{n}{replace}\PY{p}{(}\PY{l+s+s2}{\PYZdq{}}\PY{l+s+s2}{ó}\PY{l+s+s2}{\PYZdq{}}\PY{p}{,} \PY{l+s+s2}{\PYZdq{}}\PY{l+s+s2}{o}\PY{l+s+s2}{\PYZdq{}}\PY{p}{)}
        \PY{n}{avisos\PYZus{}detalle\PYZus{}completo}\PY{p}{[}\PY{l+s+s1}{\PYZsq{}}\PY{l+s+s1}{nivel\PYZus{}laboral}\PY{l+s+s1}{\PYZsq{}}\PY{p}{]} \PY{o}{=} \PY{n}{avisos\PYZus{}detalle\PYZus{}completo}\PY{p}{[}\PY{l+s+s1}{\PYZsq{}}\PY{l+s+s1}{nivel\PYZus{}laboral}\PY{l+s+s1}{\PYZsq{}}\PY{p}{]}\PY{o}{.}\PY{n}{str}\PY{o}{.}\PY{n}{replace}\PY{p}{(}\PY{l+s+s2}{\PYZdq{}}\PY{l+s+s2}{ú}\PY{l+s+s2}{\PYZdq{}}\PY{p}{,} \PY{l+s+s2}{\PYZdq{}}\PY{l+s+s2}{u}\PY{l+s+s2}{\PYZdq{}}\PY{p}{)}
\end{Verbatim}


    \hypertarget{visitas}{%
\subsubsection{1.2.3 - Visitas:}\label{visitas}}

Sobre esta información lo primero que queremos hacer es transformar la
fecha a un formato que podamos manejar con propiedad, en segunda
instancia nos parece interesante obtener el día de la semana donde se
produjo la visita. Luego debemos de ordenar los días para que a futuro
sea más lógica la graficación.

Luego le incorporamos la información de los detalles de cada aviso, como
así también hacemos un filtro sobre las ingenierías que más relevancia
muestran en los avisos, como así también una limpieza sobre los acentos
tanto para la denominación de la empresa como para el area sobre la que
hace referencia el aviso. Unificamos también los datos referidos a
ADECCO, el cual se divercifica en zonas pero para nuestro análisis
termina siendo lo mismo.

Por último y no menos importante es necesario empezar a obtener la
información de quién es la persona que ha decidido visitar la
publicación, con lo que terminamos unificando la información con los
datos de los postulantes previamente adquiridos.

    \begin{Verbatim}[commandchars=\\\{\}]
{\color{incolor}In [{\color{incolor}5}]:} \PY{c+c1}{\PYZsh{}\PYZsh{} PLANTEMOAS LA FECHA EN QUE SE VISITAS EL AVISO COMO UNA NUEVA COLUMNA DATETIME.}
        \PY{n}{vistas\PYZus{}avisos}\PY{p}{[}\PY{l+s+s1}{\PYZsq{}}\PY{l+s+s1}{fecha\PYZus{}vista}\PY{l+s+s1}{\PYZsq{}}\PY{p}{]} \PY{o}{=} \PY{n}{pd}\PY{o}{.}\PY{n}{to\PYZus{}datetime}\PY{p}{(}\PY{n}{vistas\PYZus{}avisos}\PY{p}{[}\PY{l+s+s1}{\PYZsq{}}\PY{l+s+s1}{timestamp}\PY{l+s+s1}{\PYZsq{}}\PY{p}{]}\PY{p}{,} \PY{n}{errors} \PY{o}{=} \PY{l+s+s1}{\PYZsq{}}\PY{l+s+s1}{coerce}\PY{l+s+s1}{\PYZsq{}}\PY{p}{)}
        \PY{c+c1}{\PYZsh{}\PYZsh{} HACEMOS UNA DIFERENCIA ENTRE EL DÍA DE LA SEMANA EN EL QUE SE LLEVÓ A CABO LA VISITA.}
        \PY{n}{vistas\PYZus{}avisos}\PY{p}{[}\PY{l+s+s1}{\PYZsq{}}\PY{l+s+s1}{dia\PYZus{}semana}\PY{l+s+s1}{\PYZsq{}}\PY{p}{]} \PY{o}{=} \PY{n}{vistas\PYZus{}avisos}\PY{p}{[}\PY{l+s+s1}{\PYZsq{}}\PY{l+s+s1}{fecha\PYZus{}vista}\PY{l+s+s1}{\PYZsq{}}\PY{p}{]}\PY{o}{.}\PY{n}{dt}\PY{o}{.}\PY{n}{weekday\PYZus{}name}
        \PY{c+c1}{\PYZsh{}\PYZsh{} CREAMOS UN ORDENAMIENTO PARA LOS DÍAS DE LA SEMANA, COSA DE QUE NO QUEDE ORDENADA POR VALORES ASCENDENTES.}
        \PY{n}{vistas\PYZus{}avisos}\PY{p}{[}\PY{l+s+s1}{\PYZsq{}}\PY{l+s+s1}{dia\PYZus{}semana}\PY{l+s+s1}{\PYZsq{}}\PY{p}{]} \PY{o}{=} \PY{n}{pd}\PY{o}{.}\PY{n}{Categorical}\PY{p}{(}\PY{n}{vistas\PYZus{}avisos}\PY{p}{[}\PY{l+s+s1}{\PYZsq{}}\PY{l+s+s1}{dia\PYZus{}semana}\PY{l+s+s1}{\PYZsq{}}\PY{p}{]}\PY{p}{,} \PY{n}{categories}\PY{o}{=}\PY{p}{[}\PY{l+s+s1}{\PYZsq{}}\PY{l+s+s1}{Monday}\PY{l+s+s1}{\PYZsq{}}\PY{p}{,}\PY{l+s+s1}{\PYZsq{}}\PY{l+s+s1}{Tuesday}\PY{l+s+s1}{\PYZsq{}}\PY{p}{,}\PY{l+s+s1}{\PYZsq{}}\PY{l+s+s1}{Wednesday}\PY{l+s+s1}{\PYZsq{}}\PY{p}{,}\PY{l+s+s1}{\PYZsq{}}\PY{l+s+s1}{Thursday}\PY{l+s+s1}{\PYZsq{}}\PY{p}{,}\PY{l+s+s1}{\PYZsq{}}\PY{l+s+s1}{Friday}\PY{l+s+s1}{\PYZsq{}}\PY{p}{,}\PY{l+s+s1}{\PYZsq{}}\PY{l+s+s1}{Saturday}\PY{l+s+s1}{\PYZsq{}}\PY{p}{,} \PY{l+s+s1}{\PYZsq{}}\PY{l+s+s1}{Sunday}\PY{l+s+s1}{\PYZsq{}}\PY{p}{]}\PY{p}{,} \PY{n}{ordered}\PY{o}{=}\PY{n+nb+bp}{True}\PY{p}{)}
        \PY{n}{vistas\PYZus{}dias} \PY{o}{=} \PY{n}{vistas\PYZus{}avisos}\PY{p}{[}\PY{l+s+s1}{\PYZsq{}}\PY{l+s+s1}{dia\PYZus{}semana}\PY{l+s+s1}{\PYZsq{}}\PY{p}{]}\PY{o}{.}\PY{n}{value\PYZus{}counts}\PY{p}{(}\PY{p}{)}
        \PY{n}{vistas\PYZus{}dias} \PY{o}{=} \PY{n}{vistas\PYZus{}dias}\PY{o}{.}\PY{n}{sort\PYZus{}index}\PY{p}{(}\PY{p}{)}
        \PY{c+c1}{\PYZsh{}\PYZsh{} LE AGREGAMOS EL DETALLE DE LOS AVISOS A LAS VISITAS QUE ESTAMOS TRABAJANDO.}
        \PY{n}{vistas\PYZus{}avisos\PYZus{}completas} \PY{o}{=} \PY{n}{vistas\PYZus{}avisos}\PY{o}{.}\PY{n}{merge}\PY{p}{(}\PY{n}{avisos\PYZus{}detalle\PYZus{}completo}\PY{p}{,}\PY{n}{left\PYZus{}on}\PY{o}{=}\PY{l+s+s1}{\PYZsq{}}\PY{l+s+s1}{idAviso}\PY{l+s+s1}{\PYZsq{}}\PY{p}{,} \PY{n}{right\PYZus{}on}\PY{o}{=}\PY{l+s+s1}{\PYZsq{}}\PY{l+s+s1}{idaviso}\PY{l+s+s1}{\PYZsq{}}\PY{p}{,}\PY{n}{how}\PY{o}{=}\PY{l+s+s1}{\PYZsq{}}\PY{l+s+s1}{left}\PY{l+s+s1}{\PYZsq{}}\PY{p}{)}
        \PY{c+c1}{\PYZsh{}\PYZsh{} HACEMOS UNA LIMPIEZA SOBRE ALGUNOS NOMBRE EN LAS VISITAS.}
        \PY{n}{vistas\PYZus{}avisos\PYZus{}completas}\PY{o}{.}\PY{n}{loc}\PY{p}{[}\PY{n}{vistas\PYZus{}avisos\PYZus{}completas}\PY{o}{.}\PY{n}{nombre\PYZus{}area} \PY{o}{==} \PY{l+s+s1}{\PYZsq{}}\PY{l+s+s1}{Ingeniería  Eléctrica y Electrónica}\PY{l+s+s1}{\PYZsq{}}\PY{p}{,} \PY{l+s+s1}{\PYZsq{}}\PY{l+s+s1}{nombre\PYZus{}area}\PY{l+s+s1}{\PYZsq{}}\PY{p}{]} \PY{o}{=} \PY{l+s+s1}{\PYZsq{}}\PY{l+s+s1}{Ingeniería Eléctrica y Electrónica}\PY{l+s+s1}{\PYZsq{}}
        \PY{n}{vistas\PYZus{}avisos\PYZus{}completas}\PY{o}{.}\PY{n}{loc}\PY{p}{[}\PY{n}{vistas\PYZus{}avisos\PYZus{}completas}\PY{o}{.}\PY{n}{nombre\PYZus{}area} \PY{o}{==} \PY{l+s+s1}{\PYZsq{}}\PY{l+s+s1}{Ingeniería  Mecánica}\PY{l+s+s1}{\PYZsq{}}\PY{p}{,} \PY{l+s+s1}{\PYZsq{}}\PY{l+s+s1}{nombre\PYZus{}area}\PY{l+s+s1}{\PYZsq{}}\PY{p}{]} \PY{o}{=} \PY{l+s+s1}{\PYZsq{}}\PY{l+s+s1}{Ingeniería Mecánica}\PY{l+s+s1}{\PYZsq{}}
        \PY{n}{vistas\PYZus{}avisos\PYZus{}completas}\PY{o}{.}\PY{n}{loc}\PY{p}{[}\PY{n}{vistas\PYZus{}avisos\PYZus{}completas}\PY{o}{.}\PY{n}{nombre\PYZus{}area} \PY{o}{==} \PY{l+s+s1}{\PYZsq{}}\PY{l+s+s1}{Ingeniería  Industrial}\PY{l+s+s1}{\PYZsq{}}\PY{p}{,} \PY{l+s+s1}{\PYZsq{}}\PY{l+s+s1}{nombre\PYZus{}area}\PY{l+s+s1}{\PYZsq{}}\PY{p}{]} \PY{o}{=} \PY{l+s+s1}{\PYZsq{}}\PY{l+s+s1}{Ingeniería Industrial}\PY{l+s+s1}{\PYZsq{}}
        \PY{n}{vistas\PYZus{}avisos\PYZus{}completas}\PY{o}{.}\PY{n}{loc}\PY{p}{[}\PY{n}{vistas\PYZus{}avisos\PYZus{}completas}\PY{o}{.}\PY{n}{nombre\PYZus{}area} \PY{o}{==} \PY{l+s+s1}{\PYZsq{}}\PY{l+s+s1}{Ingeniería  Automotriz}\PY{l+s+s1}{\PYZsq{}}\PY{p}{,} \PY{l+s+s1}{\PYZsq{}}\PY{l+s+s1}{nombre\PYZus{}area}\PY{l+s+s1}{\PYZsq{}}\PY{p}{]} \PY{o}{=} \PY{l+s+s1}{\PYZsq{}}\PY{l+s+s1}{Ingeniería Automotriz}\PY{l+s+s1}{\PYZsq{}}
        \PY{n}{vistas\PYZus{}avisos\PYZus{}completas}\PY{o}{.}\PY{n}{loc}\PY{p}{[}\PY{n}{vistas\PYZus{}avisos\PYZus{}completas}\PY{o}{.}\PY{n}{nombre\PYZus{}area} \PY{o}{==} \PY{l+s+s1}{\PYZsq{}}\PY{l+s+s1}{Ingeniería  Metalurgica}\PY{l+s+s1}{\PYZsq{}}\PY{p}{,} \PY{l+s+s1}{\PYZsq{}}\PY{l+s+s1}{nombre\PYZus{}area}\PY{l+s+s1}{\PYZsq{}}\PY{p}{]} \PY{o}{=} \PY{l+s+s1}{\PYZsq{}}\PY{l+s+s1}{Ingeniería Metalurgica}\PY{l+s+s1}{\PYZsq{}}
        \PY{c+c1}{\PYZsh{}\PYZsh{} LIMPIAMOS LOS ACENTOS PARA EL NOMBRE\PYZus{}AREA QUE NOS TRAEN PROBLEMAS A LA HORA DE GRAFICAR.}
        \PY{n}{vistas\PYZus{}avisos\PYZus{}completas}\PY{p}{[}\PY{l+s+s1}{\PYZsq{}}\PY{l+s+s1}{nombre\PYZus{}area}\PY{l+s+s1}{\PYZsq{}}\PY{p}{]} \PY{o}{=} \PY{n}{vistas\PYZus{}avisos\PYZus{}completas}\PY{p}{[}\PY{l+s+s1}{\PYZsq{}}\PY{l+s+s1}{nombre\PYZus{}area}\PY{l+s+s1}{\PYZsq{}}\PY{p}{]}\PY{o}{.}\PY{n}{str}\PY{o}{.}\PY{n}{replace}\PY{p}{(}\PY{l+s+s2}{\PYZdq{}}\PY{l+s+s2}{á}\PY{l+s+s2}{\PYZdq{}}\PY{p}{,} \PY{l+s+s2}{\PYZdq{}}\PY{l+s+s2}{a}\PY{l+s+s2}{\PYZdq{}}\PY{p}{)}
        \PY{n}{vistas\PYZus{}avisos\PYZus{}completas}\PY{p}{[}\PY{l+s+s1}{\PYZsq{}}\PY{l+s+s1}{nombre\PYZus{}area}\PY{l+s+s1}{\PYZsq{}}\PY{p}{]} \PY{o}{=} \PY{n}{vistas\PYZus{}avisos\PYZus{}completas}\PY{p}{[}\PY{l+s+s1}{\PYZsq{}}\PY{l+s+s1}{nombre\PYZus{}area}\PY{l+s+s1}{\PYZsq{}}\PY{p}{]}\PY{o}{.}\PY{n}{str}\PY{o}{.}\PY{n}{replace}\PY{p}{(}\PY{l+s+s2}{\PYZdq{}}\PY{l+s+s2}{é}\PY{l+s+s2}{\PYZdq{}}\PY{p}{,} \PY{l+s+s2}{\PYZdq{}}\PY{l+s+s2}{e}\PY{l+s+s2}{\PYZdq{}}\PY{p}{)}
        \PY{n}{vistas\PYZus{}avisos\PYZus{}completas}\PY{p}{[}\PY{l+s+s1}{\PYZsq{}}\PY{l+s+s1}{nombre\PYZus{}area}\PY{l+s+s1}{\PYZsq{}}\PY{p}{]} \PY{o}{=} \PY{n}{vistas\PYZus{}avisos\PYZus{}completas}\PY{p}{[}\PY{l+s+s1}{\PYZsq{}}\PY{l+s+s1}{nombre\PYZus{}area}\PY{l+s+s1}{\PYZsq{}}\PY{p}{]}\PY{o}{.}\PY{n}{str}\PY{o}{.}\PY{n}{replace}\PY{p}{(}\PY{l+s+s2}{\PYZdq{}}\PY{l+s+s2}{í}\PY{l+s+s2}{\PYZdq{}}\PY{p}{,} \PY{l+s+s2}{\PYZdq{}}\PY{l+s+s2}{i}\PY{l+s+s2}{\PYZdq{}}\PY{p}{)}
        \PY{n}{vistas\PYZus{}avisos\PYZus{}completas}\PY{p}{[}\PY{l+s+s1}{\PYZsq{}}\PY{l+s+s1}{nombre\PYZus{}area}\PY{l+s+s1}{\PYZsq{}}\PY{p}{]} \PY{o}{=} \PY{n}{vistas\PYZus{}avisos\PYZus{}completas}\PY{p}{[}\PY{l+s+s1}{\PYZsq{}}\PY{l+s+s1}{nombre\PYZus{}area}\PY{l+s+s1}{\PYZsq{}}\PY{p}{]}\PY{o}{.}\PY{n}{str}\PY{o}{.}\PY{n}{replace}\PY{p}{(}\PY{l+s+s2}{\PYZdq{}}\PY{l+s+s2}{ó}\PY{l+s+s2}{\PYZdq{}}\PY{p}{,} \PY{l+s+s2}{\PYZdq{}}\PY{l+s+s2}{o}\PY{l+s+s2}{\PYZdq{}}\PY{p}{)}
        \PY{n}{vistas\PYZus{}avisos\PYZus{}completas}\PY{p}{[}\PY{l+s+s1}{\PYZsq{}}\PY{l+s+s1}{nombre\PYZus{}area}\PY{l+s+s1}{\PYZsq{}}\PY{p}{]} \PY{o}{=} \PY{n}{vistas\PYZus{}avisos\PYZus{}completas}\PY{p}{[}\PY{l+s+s1}{\PYZsq{}}\PY{l+s+s1}{nombre\PYZus{}area}\PY{l+s+s1}{\PYZsq{}}\PY{p}{]}\PY{o}{.}\PY{n}{str}\PY{o}{.}\PY{n}{replace}\PY{p}{(}\PY{l+s+s2}{\PYZdq{}}\PY{l+s+s2}{ú}\PY{l+s+s2}{\PYZdq{}}\PY{p}{,} \PY{l+s+s2}{\PYZdq{}}\PY{l+s+s2}{u}\PY{l+s+s2}{\PYZdq{}}\PY{p}{)}
        \PY{c+c1}{\PYZsh{}\PYZsh{} LIMPIAMOS LOS ACENTOS PARA LA DENOMINACION\PYZus{}EMPRESA QUE NOS TRAEN PROBLEMAS A LA HORA DE GRAFICAR.}
        \PY{n}{vistas\PYZus{}avisos\PYZus{}completas}\PY{p}{[}\PY{l+s+s1}{\PYZsq{}}\PY{l+s+s1}{denominacion\PYZus{}empresa}\PY{l+s+s1}{\PYZsq{}}\PY{p}{]} \PY{o}{=} \PY{n}{vistas\PYZus{}avisos\PYZus{}completas}\PY{p}{[}\PY{l+s+s1}{\PYZsq{}}\PY{l+s+s1}{denominacion\PYZus{}empresa}\PY{l+s+s1}{\PYZsq{}}\PY{p}{]}\PY{o}{.}\PY{n}{str}\PY{o}{.}\PY{n}{replace}\PY{p}{(}\PY{l+s+s2}{\PYZdq{}}\PY{l+s+s2}{á}\PY{l+s+s2}{\PYZdq{}}\PY{p}{,} \PY{l+s+s2}{\PYZdq{}}\PY{l+s+s2}{a}\PY{l+s+s2}{\PYZdq{}}\PY{p}{)}
        \PY{n}{vistas\PYZus{}avisos\PYZus{}completas}\PY{p}{[}\PY{l+s+s1}{\PYZsq{}}\PY{l+s+s1}{denominacion\PYZus{}empresa}\PY{l+s+s1}{\PYZsq{}}\PY{p}{]} \PY{o}{=} \PY{n}{vistas\PYZus{}avisos\PYZus{}completas}\PY{p}{[}\PY{l+s+s1}{\PYZsq{}}\PY{l+s+s1}{denominacion\PYZus{}empresa}\PY{l+s+s1}{\PYZsq{}}\PY{p}{]}\PY{o}{.}\PY{n}{str}\PY{o}{.}\PY{n}{replace}\PY{p}{(}\PY{l+s+s2}{\PYZdq{}}\PY{l+s+s2}{é}\PY{l+s+s2}{\PYZdq{}}\PY{p}{,} \PY{l+s+s2}{\PYZdq{}}\PY{l+s+s2}{e}\PY{l+s+s2}{\PYZdq{}}\PY{p}{)}
        \PY{n}{vistas\PYZus{}avisos\PYZus{}completas}\PY{p}{[}\PY{l+s+s1}{\PYZsq{}}\PY{l+s+s1}{denominacion\PYZus{}empresa}\PY{l+s+s1}{\PYZsq{}}\PY{p}{]} \PY{o}{=} \PY{n}{vistas\PYZus{}avisos\PYZus{}completas}\PY{p}{[}\PY{l+s+s1}{\PYZsq{}}\PY{l+s+s1}{denominacion\PYZus{}empresa}\PY{l+s+s1}{\PYZsq{}}\PY{p}{]}\PY{o}{.}\PY{n}{str}\PY{o}{.}\PY{n}{replace}\PY{p}{(}\PY{l+s+s2}{\PYZdq{}}\PY{l+s+s2}{í}\PY{l+s+s2}{\PYZdq{}}\PY{p}{,} \PY{l+s+s2}{\PYZdq{}}\PY{l+s+s2}{i}\PY{l+s+s2}{\PYZdq{}}\PY{p}{)}
        \PY{n}{vistas\PYZus{}avisos\PYZus{}completas}\PY{p}{[}\PY{l+s+s1}{\PYZsq{}}\PY{l+s+s1}{denominacion\PYZus{}empresa}\PY{l+s+s1}{\PYZsq{}}\PY{p}{]} \PY{o}{=} \PY{n}{vistas\PYZus{}avisos\PYZus{}completas}\PY{p}{[}\PY{l+s+s1}{\PYZsq{}}\PY{l+s+s1}{denominacion\PYZus{}empresa}\PY{l+s+s1}{\PYZsq{}}\PY{p}{]}\PY{o}{.}\PY{n}{str}\PY{o}{.}\PY{n}{replace}\PY{p}{(}\PY{l+s+s2}{\PYZdq{}}\PY{l+s+s2}{ó}\PY{l+s+s2}{\PYZdq{}}\PY{p}{,} \PY{l+s+s2}{\PYZdq{}}\PY{l+s+s2}{o}\PY{l+s+s2}{\PYZdq{}}\PY{p}{)}
        \PY{n}{vistas\PYZus{}avisos\PYZus{}completas}\PY{p}{[}\PY{l+s+s1}{\PYZsq{}}\PY{l+s+s1}{denominacion\PYZus{}empresa}\PY{l+s+s1}{\PYZsq{}}\PY{p}{]} \PY{o}{=} \PY{n}{vistas\PYZus{}avisos\PYZus{}completas}\PY{p}{[}\PY{l+s+s1}{\PYZsq{}}\PY{l+s+s1}{denominacion\PYZus{}empresa}\PY{l+s+s1}{\PYZsq{}}\PY{p}{]}\PY{o}{.}\PY{n}{str}\PY{o}{.}\PY{n}{replace}\PY{p}{(}\PY{l+s+s2}{\PYZdq{}}\PY{l+s+s2}{ú}\PY{l+s+s2}{\PYZdq{}}\PY{p}{,} \PY{l+s+s2}{\PYZdq{}}\PY{l+s+s2}{u}\PY{l+s+s2}{\PYZdq{}}\PY{p}{)}
        \PY{c+c1}{\PYZsh{}\PYZsh{} HACEMOS UN FILTRADO SOBRE LAS AREAS QUE SE RELACIONAN CON INGENIERÍA.}
        \PY{n}{filtrado\PYZus{}ingenieria} \PY{o}{=} \PY{n}{vistas\PYZus{}avisos\PYZus{}completas}\PY{p}{[}\PY{n}{vistas\PYZus{}avisos\PYZus{}completas}\PY{p}{[}\PY{l+s+s1}{\PYZsq{}}\PY{l+s+s1}{nombre\PYZus{}area}\PY{l+s+s1}{\PYZsq{}}\PY{p}{]}\PY{o}{.}\PY{n}{str}\PY{o}{.}\PY{n}{contains}\PY{p}{(}\PY{l+s+s2}{\PYZdq{}}\PY{l+s+s2}{Ingen}\PY{l+s+s2}{\PYZdq{}}\PY{p}{,} \PY{n}{na}\PY{o}{=}\PY{n+nb+bp}{False}\PY{p}{)}\PY{p}{]}
        \PY{n}{filtrado\PYZus{}ingenieria}\PY{p}{[}\PY{l+s+s1}{\PYZsq{}}\PY{l+s+s1}{nombre\PYZus{}area}\PY{l+s+s1}{\PYZsq{}}\PY{p}{]} \PY{o}{=} \PY{n}{filtrado\PYZus{}ingenieria}\PY{p}{[}\PY{l+s+s1}{\PYZsq{}}\PY{l+s+s1}{nombre\PYZus{}area}\PY{l+s+s1}{\PYZsq{}}\PY{p}{]}\PY{o}{.}\PY{n}{str}\PY{o}{.}\PY{n}{replace}\PY{p}{(}\PY{l+s+s2}{\PYZdq{}}\PY{l+s+s2}{Ingenieria}\PY{l+s+s2}{\PYZdq{}}\PY{p}{,} \PY{l+s+s2}{\PYZdq{}}\PY{l+s+s2}{I.}\PY{l+s+s2}{\PYZdq{}}\PY{p}{)}
        \PY{n}{filtrado\PYZus{}ingenieria}\PY{p}{[}\PY{l+s+s1}{\PYZsq{}}\PY{l+s+s1}{nombre\PYZus{}area}\PY{l+s+s1}{\PYZsq{}}\PY{p}{]} \PY{o}{=} \PY{n}{filtrado\PYZus{}ingenieria}\PY{p}{[}\PY{l+s+s1}{\PYZsq{}}\PY{l+s+s1}{nombre\PYZus{}area}\PY{l+s+s1}{\PYZsq{}}\PY{p}{]}\PY{o}{.}\PY{n}{str}\PY{o}{.}\PY{n}{replace}\PY{p}{(}\PY{l+s+s2}{\PYZdq{}}\PY{l+s+s2}{Ingenierias}\PY{l+s+s2}{\PYZdq{}}\PY{p}{,} \PY{l+s+s2}{\PYZdq{}}\PY{l+s+s2}{I.}\PY{l+s+s2}{\PYZdq{}}\PY{p}{)}
        \PY{c+c1}{\PYZsh{}\PYZsh{} COMO LOS NOMBRES SUELEN SER MUY LARGOS LOS LIMPIAMOS PARA QUE SEA MÁS LEGIBLE EL GRÁFICO.}
        \PY{n}{filtrado\PYZus{}ingenieria} \PY{o}{=} \PY{n}{filtrado\PYZus{}ingenieria}\PY{p}{[}\PY{n}{filtrado\PYZus{}ingenieria}\PY{p}{[}\PY{l+s+s1}{\PYZsq{}}\PY{l+s+s1}{nombre\PYZus{}area}\PY{l+s+s1}{\PYZsq{}}\PY{p}{]}\PY{o}{.}\PY{n}{str}\PY{o}{.}\PY{n}{contains}\PY{p}{(}\PY{l+s+s2}{\PYZdq{}}\PY{l+s+s2}{I. }\PY{l+s+s2}{\PYZdq{}}\PY{p}{,} \PY{n}{na}\PY{o}{=}\PY{n+nb+bp}{False}\PY{p}{)}\PY{p}{]}
        \PY{c+c1}{\PYZsh{}\PYZsh{} NOS QUEDAMOS CON LAS DIEZ MÁS RELEVANTES.}
        \PY{n}{diez\PYZus{}ingenierias} \PY{o}{=} \PY{n}{filtrado\PYZus{}ingenieria}\PY{p}{[}\PY{l+s+s1}{\PYZsq{}}\PY{l+s+s1}{nombre\PYZus{}area}\PY{l+s+s1}{\PYZsq{}}\PY{p}{]}\PY{o}{.}\PY{n}{value\PYZus{}counts}\PY{p}{(}\PY{p}{)}\PY{p}{[}\PY{p}{:}\PY{l+m+mi}{10}\PY{p}{]}
        \PY{c+c1}{\PYZsh{}\PYZsh{} PARA COMPARAR TAMBIÉN BUSCAMOS AQUELLLAS AREAS EN GENERAL QUE TIENEN MÁS VISITAS.}
        \PY{n}{top\PYZus{}visitados\PYZus{}area}\PY{o}{=} \PY{n}{vistas\PYZus{}avisos\PYZus{}completas}\PY{p}{[}\PY{l+s+s1}{\PYZsq{}}\PY{l+s+s1}{nombre\PYZus{}area}\PY{l+s+s1}{\PYZsq{}}\PY{p}{]}\PY{o}{.}\PY{n}{value\PYZus{}counts}\PY{p}{(}\PY{p}{)}\PY{p}{[}\PY{p}{:}\PY{l+m+mi}{10}\PY{p}{]}
        \PY{c+c1}{\PYZsh{}\PYZsh{} VIENDO LOS DATOS VEMOS QUE ADECCO TIENE MUCHOS DATOS PERO SEPARADOS SEGÚN ZONAS, LOS UNIFICAMOS PARA MAYOR CLARIDAD.}
        \PY{n}{vistas\PYZus{}avisos\PYZus{}completas}\PY{o}{.}\PY{n}{loc}\PY{p}{[}\PY{n}{vistas\PYZus{}avisos\PYZus{}completas}\PY{o}{.}\PY{n}{denominacion\PYZus{}empresa}\PY{o}{.}\PY{n}{str}\PY{o}{.}\PY{n}{contains}\PY{p}{(}\PY{l+s+s2}{\PYZdq{}}\PY{l+s+s2}{Adecco}\PY{l+s+s2}{\PYZdq{}}\PY{p}{)}\PY{o}{==}\PY{n+nb+bp}{True}\PY{p}{,} \PY{l+s+s1}{\PYZsq{}}\PY{l+s+s1}{denominacion\PYZus{}empresa}\PY{l+s+s1}{\PYZsq{}}\PY{p}{]} \PY{o}{=} \PY{l+s+s1}{\PYZsq{}}\PY{l+s+s1}{Adecco}\PY{l+s+s1}{\PYZsq{}}
        \PY{c+c1}{\PYZsh{}\PYZsh{} VEMOS LAS EMPRESAS QUE TIENEN UNA MAYOR CANTIDAD DE VISITAS.}
        \PY{n}{diez\PYZus{}vistas\PYZus{}empresas} \PY{o}{=} \PY{n}{vistas\PYZus{}avisos\PYZus{}completas}\PY{p}{[}\PY{l+s+s1}{\PYZsq{}}\PY{l+s+s1}{denominacion\PYZus{}empresa}\PY{l+s+s1}{\PYZsq{}}\PY{p}{]}\PY{o}{.}\PY{n}{value\PYZus{}counts}\PY{p}{(}\PY{p}{)}\PY{p}{[}\PY{p}{:}\PY{l+m+mi}{10}\PY{p}{]}
        \PY{c+c1}{\PYZsh{}\PYZsh{} PARAMETRIZAMOS LAS FECHAS TANTO EL DÍA COMO EL MES PARA ESTE FUENTE DE DATOS COMPLETA.}
        \PY{n}{vistas\PYZus{}avisos\PYZus{}completas}\PY{p}{[}\PY{l+s+s1}{\PYZsq{}}\PY{l+s+s1}{fecha\PYZus{}vista\PYZus{}completa}\PY{l+s+s1}{\PYZsq{}}\PY{p}{]} \PY{o}{=} \PY{n}{pd}\PY{o}{.}\PY{n}{to\PYZus{}datetime}\PY{p}{(}\PY{n}{vistas\PYZus{}avisos\PYZus{}completas}\PY{p}{[}\PY{l+s+s1}{\PYZsq{}}\PY{l+s+s1}{timestamp}\PY{l+s+s1}{\PYZsq{}}\PY{p}{]}\PY{p}{,} \PY{n}{errors} \PY{o}{=} \PY{l+s+s1}{\PYZsq{}}\PY{l+s+s1}{coerce}\PY{l+s+s1}{\PYZsq{}}\PY{p}{)}
        \PY{n}{vistas\PYZus{}avisos\PYZus{}completas}\PY{p}{[}\PY{l+s+s1}{\PYZsq{}}\PY{l+s+s1}{dia\PYZus{}semana\PYZus{}completa}\PY{l+s+s1}{\PYZsq{}}\PY{p}{]} \PY{o}{=} \PY{n}{vistas\PYZus{}avisos\PYZus{}completas}\PY{p}{[}\PY{l+s+s1}{\PYZsq{}}\PY{l+s+s1}{fecha\PYZus{}vista\PYZus{}completa}\PY{l+s+s1}{\PYZsq{}}\PY{p}{]}\PY{o}{.}\PY{n}{dt}\PY{o}{.}\PY{n}{weekday\PYZus{}name}
        \PY{n}{vistas\PYZus{}avisos\PYZus{}completas}\PY{p}{[}\PY{l+s+s1}{\PYZsq{}}\PY{l+s+s1}{mes\PYZus{}completa}\PY{l+s+s1}{\PYZsq{}}\PY{p}{]} \PY{o}{=} \PY{n}{vistas\PYZus{}avisos\PYZus{}completas}\PY{p}{[}\PY{l+s+s1}{\PYZsq{}}\PY{l+s+s1}{fecha\PYZus{}vista\PYZus{}completa}\PY{l+s+s1}{\PYZsq{}}\PY{p}{]}\PY{o}{.}\PY{n}{dt}\PY{o}{.}\PY{n}{month}
        \PY{n}{vistas\PYZus{}avisos\PYZus{}completas}\PY{p}{[}\PY{l+s+s1}{\PYZsq{}}\PY{l+s+s1}{dia\PYZus{}completa}\PY{l+s+s1}{\PYZsq{}}\PY{p}{]} \PY{o}{=} \PY{n}{vistas\PYZus{}avisos\PYZus{}completas}\PY{p}{[}\PY{l+s+s1}{\PYZsq{}}\PY{l+s+s1}{fecha\PYZus{}vista\PYZus{}completa}\PY{l+s+s1}{\PYZsq{}}\PY{p}{]}\PY{o}{.}\PY{n}{dt}\PY{o}{.}\PY{n}{day}
        \PY{c+c1}{\PYZsh{}\PYZsh{} OBTENEMOS LA INFORMACION DE QUIEN HA HECHO LA VISITA.}
        \PY{n}{vistas\PYZus{}avisos\PYZus{}completas} \PY{o}{=} \PY{n}{vistas\PYZus{}avisos\PYZus{}completas}\PY{o}{.}\PY{n}{merge}\PY{p}{(}\PY{n}{postulantes\PYZus{}unicos}\PY{p}{,}\PY{n}{left\PYZus{}on}\PY{o}{=}\PY{l+s+s1}{\PYZsq{}}\PY{l+s+s1}{idpostulante}\PY{l+s+s1}{\PYZsq{}}\PY{p}{,} \PY{n}{right\PYZus{}on}\PY{o}{=}\PY{l+s+s1}{\PYZsq{}}\PY{l+s+s1}{idpostulante}\PY{l+s+s1}{\PYZsq{}}\PY{p}{,}\PY{n}{how}\PY{o}{=}\PY{l+s+s1}{\PYZsq{}}\PY{l+s+s1}{left}\PY{l+s+s1}{\PYZsq{}}\PY{p}{)}
\end{Verbatim}


    \hypertarget{postulaciones}{%
\subsubsection{1.2.4 - Postulaciones:}\label{postulaciones}}

Con estos datos vamos a ver las postulaciones a los avisos propiamente
dichos, lo primero sería hacer algo similar a lo que se hizo con las
visitas, transformar la fecha a un formato apropiado para poder
trabajarla, nuevamente como segunda medida obtenemos el día de la semana
al que corresponde cada postulación. Luego debemos de ordenar los días
de la semana para que sea más lógica la graficación, también nificamos
con los detalles que tienen los avisos cosa de poder contrastar mayor
información, como así también aplicamos una limpieza sobre los acentos
tanto para el area como para la denominación de la empresa.

Obtenemos también las diez areas con más postulaciones y las diez
ingenierías con más postulaciones. Unificamos los datos referidos a
ADECCO, el cual se divercifica en zonas pero para nuestro análisis
termina siendo lo mismo. Luego esto nos lleva a tomar un listado de las
diez empresas con mayor cantidad de postulaciones.

Por último y no menos importante es necesario empezar a obtener la
información de quién es la persona que se ha decidido postular, con lo
que terminamos unificando la información con los datos de los
postulantes previamente adquiridos.

    \begin{Verbatim}[commandchars=\\\{\}]
{\color{incolor}In [{\color{incolor}7}]:} \PY{c+c1}{\PYZsh{}\PYZsh{} REDUCIMOS EL STRING QUE CORRESPONDE A LA FECHA SOLO HACIENDO REFERENCIA AL DÍA/MES/AÑO.}
        \PY{n}{postulaciones}\PY{p}{[}\PY{l+s+s1}{\PYZsq{}}\PY{l+s+s1}{fecha\PYZus{}postulado}\PY{l+s+s1}{\PYZsq{}}\PY{p}{]} \PY{o}{=} \PY{n}{postulaciones}\PY{p}{[}\PY{l+s+s1}{\PYZsq{}}\PY{l+s+s1}{fechapostulacion}\PY{l+s+s1}{\PYZsq{}}\PY{p}{]}\PY{o}{.}\PY{n}{str}\PY{o}{.}\PY{n}{slice}\PY{p}{(}\PY{l+m+mi}{0}\PY{p}{,} \PY{l+m+mi}{10}\PY{p}{)}
        \PY{c+c1}{\PYZsh{}\PYZsh{} LO TRANSFORMAMOS EN UN DATETIME COSA DE PODER TRABAJARLO MEJOR.}
        \PY{n}{postulaciones}\PY{p}{[}\PY{l+s+s1}{\PYZsq{}}\PY{l+s+s1}{fecha\PYZus{}postulado}\PY{l+s+s1}{\PYZsq{}}\PY{p}{]} \PY{o}{=} \PY{n}{pd}\PY{o}{.}\PY{n}{to\PYZus{}datetime}\PY{p}{(}\PY{n}{postulaciones}\PY{p}{[}\PY{l+s+s1}{\PYZsq{}}\PY{l+s+s1}{fecha\PYZus{}postulado}\PY{l+s+s1}{\PYZsq{}}\PY{p}{]}\PY{p}{,} \PY{n}{errors} \PY{o}{=} \PY{l+s+s1}{\PYZsq{}}\PY{l+s+s1}{coerce}\PY{l+s+s1}{\PYZsq{}}\PY{p}{)}
        \PY{c+c1}{\PYZsh{}\PYZsh{} OBTENEMOS EL DÍA DE LA SEMANA QUE CORRESPONDE A ESA POSTULACIÓN EN CUESTIÓN.}
        \PY{n}{postulaciones}\PY{p}{[}\PY{l+s+s1}{\PYZsq{}}\PY{l+s+s1}{dia\PYZus{}semana}\PY{l+s+s1}{\PYZsq{}}\PY{p}{]} \PY{o}{=} \PY{n}{postulaciones}\PY{p}{[}\PY{l+s+s1}{\PYZsq{}}\PY{l+s+s1}{fecha\PYZus{}postulado}\PY{l+s+s1}{\PYZsq{}}\PY{p}{]}\PY{o}{.}\PY{n}{dt}\PY{o}{.}\PY{n}{weekday\PYZus{}name}
        \PY{c+c1}{\PYZsh{}\PYZsh{} CREAMOS UN ORDENAMIENTO PARA LOS DÍAS DE LA SEMANA, COSA DE QUE NO QUEDE ORDENADA POR VALORES ASCENDENTES.}
        \PY{n}{postulaciones}\PY{p}{[}\PY{l+s+s1}{\PYZsq{}}\PY{l+s+s1}{dia\PYZus{}semana}\PY{l+s+s1}{\PYZsq{}}\PY{p}{]} \PY{o}{=} \PY{n}{pd}\PY{o}{.}\PY{n}{Categorical}\PY{p}{(}\PY{n}{postulaciones}\PY{p}{[}\PY{l+s+s1}{\PYZsq{}}\PY{l+s+s1}{dia\PYZus{}semana}\PY{l+s+s1}{\PYZsq{}}\PY{p}{]}\PY{p}{,} \PY{n}{categories}\PY{o}{=}\PY{p}{[}\PY{l+s+s1}{\PYZsq{}}\PY{l+s+s1}{Monday}\PY{l+s+s1}{\PYZsq{}}\PY{p}{,}\PY{l+s+s1}{\PYZsq{}}\PY{l+s+s1}{Tuesday}\PY{l+s+s1}{\PYZsq{}}\PY{p}{,}\PY{l+s+s1}{\PYZsq{}}\PY{l+s+s1}{Wednesday}\PY{l+s+s1}{\PYZsq{}}\PY{p}{,}\PY{l+s+s1}{\PYZsq{}}\PY{l+s+s1}{Thursday}\PY{l+s+s1}{\PYZsq{}}\PY{p}{,}\PY{l+s+s1}{\PYZsq{}}\PY{l+s+s1}{Friday}\PY{l+s+s1}{\PYZsq{}}\PY{p}{,}\PY{l+s+s1}{\PYZsq{}}\PY{l+s+s1}{Saturday}\PY{l+s+s1}{\PYZsq{}}\PY{p}{,} \PY{l+s+s1}{\PYZsq{}}\PY{l+s+s1}{Sunday}\PY{l+s+s1}{\PYZsq{}}\PY{p}{]}\PY{p}{,} \PY{n}{ordered}\PY{o}{=}\PY{n+nb+bp}{True}\PY{p}{)}
        \PY{n}{postulaciones\PYZus{}dias} \PY{o}{=} \PY{n}{postulaciones}\PY{p}{[}\PY{l+s+s1}{\PYZsq{}}\PY{l+s+s1}{dia\PYZus{}semana}\PY{l+s+s1}{\PYZsq{}}\PY{p}{]}\PY{o}{.}\PY{n}{value\PYZus{}counts}\PY{p}{(}\PY{p}{)}
        \PY{n}{postulaciones\PYZus{}dias} \PY{o}{=} \PY{n}{postulaciones\PYZus{}dias}\PY{o}{.}\PY{n}{sort\PYZus{}index}\PY{p}{(}\PY{p}{)}
        \PY{c+c1}{\PYZsh{}\PYZsh{} LE AGREGAMOS EL DETALLE DE LOS AVISOS A LAS POSTULACIONES QUE ESTAMOS TRABAJANDO.}
        \PY{n}{postulaciones\PYZus{}avisos\PYZus{}completos} \PY{o}{=} \PY{n}{postulaciones}\PY{o}{.}\PY{n}{merge}\PY{p}{(}\PY{n}{avisos\PYZus{}detalle\PYZus{}completo}\PY{p}{,}\PY{n}{left\PYZus{}on}\PY{o}{=}\PY{l+s+s1}{\PYZsq{}}\PY{l+s+s1}{idaviso}\PY{l+s+s1}{\PYZsq{}}\PY{p}{,} \PY{n}{right\PYZus{}on}\PY{o}{=}\PY{l+s+s1}{\PYZsq{}}\PY{l+s+s1}{idaviso}\PY{l+s+s1}{\PYZsq{}}\PY{p}{,}\PY{n}{how}\PY{o}{=}\PY{l+s+s1}{\PYZsq{}}\PY{l+s+s1}{left}\PY{l+s+s1}{\PYZsq{}}\PY{p}{)}
        \PY{c+c1}{\PYZsh{}\PYZsh{} LIMPIAMOS LOS ACENTOS SOBRE EL NOMBRE\PYZus{}AREA QUE NOS TRAEN PROBLEMAS A LA HORA DE GRAFICAR.}
        \PY{n}{postulaciones\PYZus{}avisos\PYZus{}completos}\PY{p}{[}\PY{l+s+s1}{\PYZsq{}}\PY{l+s+s1}{nombre\PYZus{}area}\PY{l+s+s1}{\PYZsq{}}\PY{p}{]} \PY{o}{=} \PY{n}{postulaciones\PYZus{}avisos\PYZus{}completos}\PY{p}{[}\PY{l+s+s1}{\PYZsq{}}\PY{l+s+s1}{nombre\PYZus{}area}\PY{l+s+s1}{\PYZsq{}}\PY{p}{]}\PY{o}{.}\PY{n}{str}\PY{o}{.}\PY{n}{replace}\PY{p}{(}\PY{l+s+s2}{\PYZdq{}}\PY{l+s+s2}{á}\PY{l+s+s2}{\PYZdq{}}\PY{p}{,} \PY{l+s+s2}{\PYZdq{}}\PY{l+s+s2}{a}\PY{l+s+s2}{\PYZdq{}}\PY{p}{)}
        \PY{n}{postulaciones\PYZus{}avisos\PYZus{}completos}\PY{p}{[}\PY{l+s+s1}{\PYZsq{}}\PY{l+s+s1}{nombre\PYZus{}area}\PY{l+s+s1}{\PYZsq{}}\PY{p}{]} \PY{o}{=} \PY{n}{postulaciones\PYZus{}avisos\PYZus{}completos}\PY{p}{[}\PY{l+s+s1}{\PYZsq{}}\PY{l+s+s1}{nombre\PYZus{}area}\PY{l+s+s1}{\PYZsq{}}\PY{p}{]}\PY{o}{.}\PY{n}{str}\PY{o}{.}\PY{n}{replace}\PY{p}{(}\PY{l+s+s2}{\PYZdq{}}\PY{l+s+s2}{é}\PY{l+s+s2}{\PYZdq{}}\PY{p}{,} \PY{l+s+s2}{\PYZdq{}}\PY{l+s+s2}{e}\PY{l+s+s2}{\PYZdq{}}\PY{p}{)}
        \PY{n}{postulaciones\PYZus{}avisos\PYZus{}completos}\PY{p}{[}\PY{l+s+s1}{\PYZsq{}}\PY{l+s+s1}{nombre\PYZus{}area}\PY{l+s+s1}{\PYZsq{}}\PY{p}{]} \PY{o}{=} \PY{n}{postulaciones\PYZus{}avisos\PYZus{}completos}\PY{p}{[}\PY{l+s+s1}{\PYZsq{}}\PY{l+s+s1}{nombre\PYZus{}area}\PY{l+s+s1}{\PYZsq{}}\PY{p}{]}\PY{o}{.}\PY{n}{str}\PY{o}{.}\PY{n}{replace}\PY{p}{(}\PY{l+s+s2}{\PYZdq{}}\PY{l+s+s2}{í}\PY{l+s+s2}{\PYZdq{}}\PY{p}{,} \PY{l+s+s2}{\PYZdq{}}\PY{l+s+s2}{i}\PY{l+s+s2}{\PYZdq{}}\PY{p}{)}
        \PY{n}{postulaciones\PYZus{}avisos\PYZus{}completos}\PY{p}{[}\PY{l+s+s1}{\PYZsq{}}\PY{l+s+s1}{nombre\PYZus{}area}\PY{l+s+s1}{\PYZsq{}}\PY{p}{]} \PY{o}{=} \PY{n}{postulaciones\PYZus{}avisos\PYZus{}completos}\PY{p}{[}\PY{l+s+s1}{\PYZsq{}}\PY{l+s+s1}{nombre\PYZus{}area}\PY{l+s+s1}{\PYZsq{}}\PY{p}{]}\PY{o}{.}\PY{n}{str}\PY{o}{.}\PY{n}{replace}\PY{p}{(}\PY{l+s+s2}{\PYZdq{}}\PY{l+s+s2}{ó}\PY{l+s+s2}{\PYZdq{}}\PY{p}{,} \PY{l+s+s2}{\PYZdq{}}\PY{l+s+s2}{o}\PY{l+s+s2}{\PYZdq{}}\PY{p}{)}
        \PY{n}{postulaciones\PYZus{}avisos\PYZus{}completos}\PY{p}{[}\PY{l+s+s1}{\PYZsq{}}\PY{l+s+s1}{nombre\PYZus{}area}\PY{l+s+s1}{\PYZsq{}}\PY{p}{]} \PY{o}{=} \PY{n}{postulaciones\PYZus{}avisos\PYZus{}completos}\PY{p}{[}\PY{l+s+s1}{\PYZsq{}}\PY{l+s+s1}{nombre\PYZus{}area}\PY{l+s+s1}{\PYZsq{}}\PY{p}{]}\PY{o}{.}\PY{n}{str}\PY{o}{.}\PY{n}{replace}\PY{p}{(}\PY{l+s+s2}{\PYZdq{}}\PY{l+s+s2}{ú}\PY{l+s+s2}{\PYZdq{}}\PY{p}{,} \PY{l+s+s2}{\PYZdq{}}\PY{l+s+s2}{u}\PY{l+s+s2}{\PYZdq{}}\PY{p}{)}
        \PY{c+c1}{\PYZsh{}\PYZsh{} LIMPIAMOS LOS ACENTOS SOBRE LA DENOMINACIÓN\PYZus{}EMPRESA QUE NOS TRAEN PROBLEMAS A LA HORA DE GRAFICAR.}
        \PY{n}{postulaciones\PYZus{}avisos\PYZus{}completos}\PY{p}{[}\PY{l+s+s1}{\PYZsq{}}\PY{l+s+s1}{denominacion\PYZus{}empresa}\PY{l+s+s1}{\PYZsq{}}\PY{p}{]} \PY{o}{=} \PY{n}{postulaciones\PYZus{}avisos\PYZus{}completos}\PY{p}{[}\PY{l+s+s1}{\PYZsq{}}\PY{l+s+s1}{denominacion\PYZus{}empresa}\PY{l+s+s1}{\PYZsq{}}\PY{p}{]}\PY{o}{.}\PY{n}{str}\PY{o}{.}\PY{n}{replace}\PY{p}{(}\PY{l+s+s2}{\PYZdq{}}\PY{l+s+s2}{á}\PY{l+s+s2}{\PYZdq{}}\PY{p}{,} \PY{l+s+s2}{\PYZdq{}}\PY{l+s+s2}{a}\PY{l+s+s2}{\PYZdq{}}\PY{p}{)}
        \PY{n}{postulaciones\PYZus{}avisos\PYZus{}completos}\PY{p}{[}\PY{l+s+s1}{\PYZsq{}}\PY{l+s+s1}{denominacion\PYZus{}empresa}\PY{l+s+s1}{\PYZsq{}}\PY{p}{]} \PY{o}{=} \PY{n}{postulaciones\PYZus{}avisos\PYZus{}completos}\PY{p}{[}\PY{l+s+s1}{\PYZsq{}}\PY{l+s+s1}{denominacion\PYZus{}empresa}\PY{l+s+s1}{\PYZsq{}}\PY{p}{]}\PY{o}{.}\PY{n}{str}\PY{o}{.}\PY{n}{replace}\PY{p}{(}\PY{l+s+s2}{\PYZdq{}}\PY{l+s+s2}{é}\PY{l+s+s2}{\PYZdq{}}\PY{p}{,} \PY{l+s+s2}{\PYZdq{}}\PY{l+s+s2}{e}\PY{l+s+s2}{\PYZdq{}}\PY{p}{)}
        \PY{n}{postulaciones\PYZus{}avisos\PYZus{}completos}\PY{p}{[}\PY{l+s+s1}{\PYZsq{}}\PY{l+s+s1}{denominacion\PYZus{}empresa}\PY{l+s+s1}{\PYZsq{}}\PY{p}{]} \PY{o}{=} \PY{n}{postulaciones\PYZus{}avisos\PYZus{}completos}\PY{p}{[}\PY{l+s+s1}{\PYZsq{}}\PY{l+s+s1}{denominacion\PYZus{}empresa}\PY{l+s+s1}{\PYZsq{}}\PY{p}{]}\PY{o}{.}\PY{n}{str}\PY{o}{.}\PY{n}{replace}\PY{p}{(}\PY{l+s+s2}{\PYZdq{}}\PY{l+s+s2}{í}\PY{l+s+s2}{\PYZdq{}}\PY{p}{,} \PY{l+s+s2}{\PYZdq{}}\PY{l+s+s2}{i}\PY{l+s+s2}{\PYZdq{}}\PY{p}{)}
        \PY{n}{postulaciones\PYZus{}avisos\PYZus{}completos}\PY{p}{[}\PY{l+s+s1}{\PYZsq{}}\PY{l+s+s1}{denominacion\PYZus{}empresa}\PY{l+s+s1}{\PYZsq{}}\PY{p}{]} \PY{o}{=} \PY{n}{postulaciones\PYZus{}avisos\PYZus{}completos}\PY{p}{[}\PY{l+s+s1}{\PYZsq{}}\PY{l+s+s1}{denominacion\PYZus{}empresa}\PY{l+s+s1}{\PYZsq{}}\PY{p}{]}\PY{o}{.}\PY{n}{str}\PY{o}{.}\PY{n}{replace}\PY{p}{(}\PY{l+s+s2}{\PYZdq{}}\PY{l+s+s2}{ó}\PY{l+s+s2}{\PYZdq{}}\PY{p}{,} \PY{l+s+s2}{\PYZdq{}}\PY{l+s+s2}{o}\PY{l+s+s2}{\PYZdq{}}\PY{p}{)}
        \PY{n}{postulaciones\PYZus{}avisos\PYZus{}completos}\PY{p}{[}\PY{l+s+s1}{\PYZsq{}}\PY{l+s+s1}{denominacion\PYZus{}empresa}\PY{l+s+s1}{\PYZsq{}}\PY{p}{]} \PY{o}{=} \PY{n}{postulaciones\PYZus{}avisos\PYZus{}completos}\PY{p}{[}\PY{l+s+s1}{\PYZsq{}}\PY{l+s+s1}{denominacion\PYZus{}empresa}\PY{l+s+s1}{\PYZsq{}}\PY{p}{]}\PY{o}{.}\PY{n}{str}\PY{o}{.}\PY{n}{replace}\PY{p}{(}\PY{l+s+s2}{\PYZdq{}}\PY{l+s+s2}{ú}\PY{l+s+s2}{\PYZdq{}}\PY{p}{,} \PY{l+s+s2}{\PYZdq{}}\PY{l+s+s2}{u}\PY{l+s+s2}{\PYZdq{}}\PY{p}{)}
        \PY{c+c1}{\PYZsh{}\PYZsh{} OBTENEMOS LAS DIEZ AREAS CON MÁS POSTULACIONES.}
        \PY{n}{top\PYZus{}postulaciones\PYZus{}area}\PY{o}{=} \PY{n}{postulaciones\PYZus{}avisos\PYZus{}completos}\PY{p}{[}\PY{l+s+s1}{\PYZsq{}}\PY{l+s+s1}{nombre\PYZus{}area}\PY{l+s+s1}{\PYZsq{}}\PY{p}{]}\PY{o}{.}\PY{n}{value\PYZus{}counts}\PY{p}{(}\PY{p}{)}\PY{p}{[}\PY{p}{:}\PY{l+m+mi}{10}\PY{p}{]}
        \PY{c+c1}{\PYZsh{}\PYZsh{} LIMPIAMOS LOS NOMBRES DE LAS INGENIERÍAS QUE INTERVIENEN EN LA CUESTIÓN.}
        \PY{n}{postulaciones\PYZus{}avisos\PYZus{}completos}\PY{o}{.}\PY{n}{loc}\PY{p}{[}\PY{n}{postulaciones\PYZus{}avisos\PYZus{}completos}\PY{o}{.}\PY{n}{nombre\PYZus{}area} \PY{o}{==} \PY{l+s+s1}{\PYZsq{}}\PY{l+s+s1}{Ingeniería  Eléctrica y Electrónica}\PY{l+s+s1}{\PYZsq{}}\PY{p}{,} \PY{l+s+s1}{\PYZsq{}}\PY{l+s+s1}{nombre\PYZus{}area}\PY{l+s+s1}{\PYZsq{}}\PY{p}{]} \PY{o}{=} \PY{l+s+s1}{\PYZsq{}}\PY{l+s+s1}{Ingeniería Eléctrica y Electrónica}\PY{l+s+s1}{\PYZsq{}}
        \PY{n}{postulaciones\PYZus{}avisos\PYZus{}completos}\PY{o}{.}\PY{n}{loc}\PY{p}{[}\PY{n}{postulaciones\PYZus{}avisos\PYZus{}completos}\PY{o}{.}\PY{n}{nombre\PYZus{}area} \PY{o}{==} \PY{l+s+s1}{\PYZsq{}}\PY{l+s+s1}{Ingeniería  Mecánica}\PY{l+s+s1}{\PYZsq{}}\PY{p}{,} \PY{l+s+s1}{\PYZsq{}}\PY{l+s+s1}{nombre\PYZus{}area}\PY{l+s+s1}{\PYZsq{}}\PY{p}{]} \PY{o}{=} \PY{l+s+s1}{\PYZsq{}}\PY{l+s+s1}{Ingeniería Mecánica}\PY{l+s+s1}{\PYZsq{}}
        \PY{n}{postulaciones\PYZus{}avisos\PYZus{}completos}\PY{o}{.}\PY{n}{loc}\PY{p}{[}\PY{n}{postulaciones\PYZus{}avisos\PYZus{}completos}\PY{o}{.}\PY{n}{nombre\PYZus{}area} \PY{o}{==} \PY{l+s+s1}{\PYZsq{}}\PY{l+s+s1}{Ingeniería  Industrial}\PY{l+s+s1}{\PYZsq{}}\PY{p}{,} \PY{l+s+s1}{\PYZsq{}}\PY{l+s+s1}{nombre\PYZus{}area}\PY{l+s+s1}{\PYZsq{}}\PY{p}{]} \PY{o}{=} \PY{l+s+s1}{\PYZsq{}}\PY{l+s+s1}{Ingeniería Industrial}\PY{l+s+s1}{\PYZsq{}}
        \PY{n}{postulaciones\PYZus{}avisos\PYZus{}completos}\PY{o}{.}\PY{n}{loc}\PY{p}{[}\PY{n}{postulaciones\PYZus{}avisos\PYZus{}completos}\PY{o}{.}\PY{n}{nombre\PYZus{}area} \PY{o}{==} \PY{l+s+s1}{\PYZsq{}}\PY{l+s+s1}{Ingeniería  Automotriz}\PY{l+s+s1}{\PYZsq{}}\PY{p}{,} \PY{l+s+s1}{\PYZsq{}}\PY{l+s+s1}{nombre\PYZus{}area}\PY{l+s+s1}{\PYZsq{}}\PY{p}{]} \PY{o}{=} \PY{l+s+s1}{\PYZsq{}}\PY{l+s+s1}{Ingeniería Automotriz}\PY{l+s+s1}{\PYZsq{}}
        \PY{n}{postulaciones\PYZus{}avisos\PYZus{}completos}\PY{o}{.}\PY{n}{loc}\PY{p}{[}\PY{n}{postulaciones\PYZus{}avisos\PYZus{}completos}\PY{o}{.}\PY{n}{nombre\PYZus{}area} \PY{o}{==} \PY{l+s+s1}{\PYZsq{}}\PY{l+s+s1}{Ingeniería  Metalurgica}\PY{l+s+s1}{\PYZsq{}}\PY{p}{,} \PY{l+s+s1}{\PYZsq{}}\PY{l+s+s1}{nombre\PYZus{}area}\PY{l+s+s1}{\PYZsq{}}\PY{p}{]} \PY{o}{=} \PY{l+s+s1}{\PYZsq{}}\PY{l+s+s1}{Ingeniería Metalurgica}\PY{l+s+s1}{\PYZsq{}}
        \PY{c+c1}{\PYZsh{}\PYZsh{} HACEMOS UN FILTRADO SOBRE EL NOMBRE DE AREA SEGÚN CORRESPONDA A INGENIERÍA.}
        \PY{n}{filtrado\PYZus{}postu\PYZus{}ingenieria} \PY{o}{=} \PY{n}{postulaciones\PYZus{}avisos\PYZus{}completos}\PY{p}{[}\PY{n}{postulaciones\PYZus{}avisos\PYZus{}completos}\PY{p}{[}\PY{l+s+s1}{\PYZsq{}}\PY{l+s+s1}{nombre\PYZus{}area}\PY{l+s+s1}{\PYZsq{}}\PY{p}{]}\PY{o}{.}\PY{n}{str}\PY{o}{.}\PY{n}{contains}\PY{p}{(}\PY{l+s+s2}{\PYZdq{}}\PY{l+s+s2}{Ingen}\PY{l+s+s2}{\PYZdq{}}\PY{p}{,} \PY{n}{na}\PY{o}{=}\PY{n+nb+bp}{False}\PY{p}{)}\PY{p}{]}
        \PY{n}{filtrado\PYZus{}postu\PYZus{}ingenieria}\PY{p}{[}\PY{l+s+s1}{\PYZsq{}}\PY{l+s+s1}{nombre\PYZus{}area}\PY{l+s+s1}{\PYZsq{}}\PY{p}{]} \PY{o}{=} \PY{n}{filtrado\PYZus{}postu\PYZus{}ingenieria}\PY{p}{[}\PY{l+s+s1}{\PYZsq{}}\PY{l+s+s1}{nombre\PYZus{}area}\PY{l+s+s1}{\PYZsq{}}\PY{p}{]}\PY{o}{.}\PY{n}{str}\PY{o}{.}\PY{n}{replace}\PY{p}{(}\PY{l+s+s2}{\PYZdq{}}\PY{l+s+s2}{Ingenieria}\PY{l+s+s2}{\PYZdq{}}\PY{p}{,} \PY{l+s+s2}{\PYZdq{}}\PY{l+s+s2}{I.}\PY{l+s+s2}{\PYZdq{}}\PY{p}{)}
        \PY{n}{filtrado\PYZus{}postu\PYZus{}ingenieria}\PY{p}{[}\PY{l+s+s1}{\PYZsq{}}\PY{l+s+s1}{nombre\PYZus{}area}\PY{l+s+s1}{\PYZsq{}}\PY{p}{]} \PY{o}{=} \PY{n}{filtrado\PYZus{}postu\PYZus{}ingenieria}\PY{p}{[}\PY{l+s+s1}{\PYZsq{}}\PY{l+s+s1}{nombre\PYZus{}area}\PY{l+s+s1}{\PYZsq{}}\PY{p}{]}\PY{o}{.}\PY{n}{str}\PY{o}{.}\PY{n}{replace}\PY{p}{(}\PY{l+s+s2}{\PYZdq{}}\PY{l+s+s2}{Ingenierias}\PY{l+s+s2}{\PYZdq{}}\PY{p}{,} \PY{l+s+s2}{\PYZdq{}}\PY{l+s+s2}{I.}\PY{l+s+s2}{\PYZdq{}}\PY{p}{)}
        \PY{n}{filtrado\PYZus{}postu\PYZus{}ingenieria} \PY{o}{=} \PY{n}{filtrado\PYZus{}postu\PYZus{}ingenieria}\PY{p}{[}\PY{n}{filtrado\PYZus{}postu\PYZus{}ingenieria}\PY{p}{[}\PY{l+s+s1}{\PYZsq{}}\PY{l+s+s1}{nombre\PYZus{}area}\PY{l+s+s1}{\PYZsq{}}\PY{p}{]}\PY{o}{.}\PY{n}{str}\PY{o}{.}\PY{n}{contains}\PY{p}{(}\PY{l+s+s2}{\PYZdq{}}\PY{l+s+s2}{I. }\PY{l+s+s2}{\PYZdq{}}\PY{p}{,} \PY{n}{na}\PY{o}{=}\PY{n+nb+bp}{False}\PY{p}{)}\PY{p}{]}
        \PY{c+c1}{\PYZsh{}\PYZsh{} TOMAMOS SOLO LAS DIEZ INGENIERÍAS QUE MAS RELEVANCIA MUESTREN.}
        \PY{n}{diez\PYZus{}ingenierias\PYZus{}postu} \PY{o}{=} \PY{n}{filtrado\PYZus{}postu\PYZus{}ingenieria}\PY{p}{[}\PY{l+s+s1}{\PYZsq{}}\PY{l+s+s1}{nombre\PYZus{}area}\PY{l+s+s1}{\PYZsq{}}\PY{p}{]}\PY{o}{.}\PY{n}{value\PYZus{}counts}\PY{p}{(}\PY{p}{)}\PY{p}{[}\PY{p}{:}\PY{l+m+mi}{10}\PY{p}{]}
        \PY{c+c1}{\PYZsh{}\PYZsh{} VIENDO LOS DATOS VEMOS QUE ADECCO TIENE MUCHOS DATOS PERO SEPARADOS SEGÚN ZONAS, LOS UNIFICAMOS PARA MAYOR CLARIDAD.}
        \PY{n}{postulaciones\PYZus{}avisos\PYZus{}completos}\PY{o}{.}\PY{n}{loc}\PY{p}{[}\PY{n}{postulaciones\PYZus{}avisos\PYZus{}completos}\PY{o}{.}\PY{n}{denominacion\PYZus{}empresa}\PY{o}{.}\PY{n}{str}\PY{o}{.}\PY{n}{contains}\PY{p}{(}\PY{l+s+s2}{\PYZdq{}}\PY{l+s+s2}{Adecco}\PY{l+s+s2}{\PYZdq{}}\PY{p}{)}\PY{o}{==}\PY{n+nb+bp}{True}\PY{p}{,} \PY{l+s+s1}{\PYZsq{}}\PY{l+s+s1}{denominacion\PYZus{}empresa}\PY{l+s+s1}{\PYZsq{}}\PY{p}{]} \PY{o}{=} \PY{l+s+s1}{\PYZsq{}}\PY{l+s+s1}{Adecco}\PY{l+s+s1}{\PYZsq{}}
        \PY{c+c1}{\PYZsh{}\PYZsh{} TOMAMOS LAS DIEZ EMPRESAS CON MAYOR CANTIDAD DE POSTULACIONES.}
        \PY{n}{diez\PYZus{}postulaciones\PYZus{}empresas} \PY{o}{=} \PY{n}{postulaciones\PYZus{}avisos\PYZus{}completos}\PY{p}{[}\PY{l+s+s1}{\PYZsq{}}\PY{l+s+s1}{denominacion\PYZus{}empresa}\PY{l+s+s1}{\PYZsq{}}\PY{p}{]}\PY{o}{.}\PY{n}{value\PYZus{}counts}\PY{p}{(}\PY{p}{)}\PY{p}{[}\PY{p}{:}\PY{l+m+mi}{10}\PY{p}{]}
        \PY{c+c1}{\PYZsh{}\PYZsh{} OBTENEMOS LA INFORMACIÓN DE QUIEN HA HECHO LA POSTULACIÓN.}
        \PY{n}{postulaciones\PYZus{}avisos\PYZus{}completos} \PY{o}{=} \PY{n}{postulaciones\PYZus{}avisos\PYZus{}completos}\PY{o}{.}\PY{n}{merge}\PY{p}{(}\PY{n}{postulantes\PYZus{}unicos}\PY{p}{,}\PY{n}{left\PYZus{}on}\PY{o}{=}\PY{l+s+s1}{\PYZsq{}}\PY{l+s+s1}{idpostulante}\PY{l+s+s1}{\PYZsq{}}\PY{p}{,} \PY{n}{right\PYZus{}on}\PY{o}{=}\PY{l+s+s1}{\PYZsq{}}\PY{l+s+s1}{idpostulante}\PY{l+s+s1}{\PYZsq{}}\PY{p}{,}\PY{n}{how}\PY{o}{=}\PY{l+s+s1}{\PYZsq{}}\PY{l+s+s1}{left}\PY{l+s+s1}{\PYZsq{}}\PY{p}{)}
\end{Verbatim}


    \hypertarget{section}{%
\subsection{======================================================================}\label{section}}

\hypertarget{resumen-y-cuestiones-a-tener-en-cuenta-sobre-el-informe.}{%
\subsection{1.3 - Resumen y cuestiones a tener en cuenta sobre el
informe.}\label{resumen-y-cuestiones-a-tener-en-cuenta-sobre-el-informe.}}

En primer instancia contamos con la siguiente información para poder
trabajar, por un lado tenemos los postulantes que son las personas que
se registran, sobre las cuales tenemos tanto su nivel educativo, su
fecha de nacimiento como así también su género, sobre estos datos los
hemos limpiado y armado rangos de edades como así también hemos
mantenido los duplicados en los niveles educativos puesto que para
algunos informes nos pueden ser de utilidad, para aquellos informes que
por el contrario nos resulte redundante o genere algún tipo de ruido
hemos optado por filtrarlos y que solo quede un registro por postulante.

Sobre los avisos vemos que tenemos tanto la información de si este se
encuentra online como no, lo que nos da la pauta para saber si está en
vigencia, por otro lado tenemos mucha información sobre el area de
trabajo, el nivel de experiencia que requiere el puesto, también tenemos
el nombre de la empresa (o grupo de empresas). todo sobre lo cual se
pueden aplicar diversa cantidad de filtros o artilugios para limpiar los
datos y poder generar gráficos más legibles y útiles.

Tenemos datos tanto de las visitas como de las postulaciones en sí,
ambos se pueden trabajar de manera similar puesto que el análisis que
hacemos sobre uno puede hacerse de manera análoga sobre el otro, lo que
nos lleva a tomar medidas similares a la hora de normalizar datos como
ser la fecha de ocurrencia, como así también la información que
obtenemos del detalle referente al aviso que se esta postulando o
viendo.

\hypertarget{section-1}{%
\subsection{======================================================================}\label{section-1}}

    \hypertarget{anuxe1lisis-sobre-los-postulantes-registrados}{%
\section{2 - Análisis sobre los postulantes
registrados:}\label{anuxe1lisis-sobre-los-postulantes-registrados}}

En esta sección nos abocaremos a realizar diferentes tipos de gráficos
relacionados pura y exclusivamente con los datos que tenemos sobre los
postulantes, es decir que tomaremos la información provista por la
fuente de datos referida a los postulantes y llevaremos a cabo
diferentes análisis que nos pueden resultar de interés como de utilidad
para una análisis próximo un poco más exaustivo. También es bueno
considerar que muchas veces denominamos usuarios registrados a los
postulantes, puesto que en definitiva representan lo mismo.

\hypertarget{cuales-son-los-rangos-de-edad-que-muxe1s-comunes-entre-los-usuarios-registrados}{%
\subsection{2.1 - ¿Cuales son los rangos de edad que más comunes entre
los usuarios
registrados?}\label{cuales-son-los-rangos-de-edad-que-muxe1s-comunes-entre-los-usuarios-registrados}}

Vale aclarar en este punto que hemos tomado ciertas consideraciones a
tener en cuenta, en primer instancia hemos decidido que las edades que
consideramos lógicas son aquellas que están dentro de un rango mayor a
los 17 años y menor a los 100 años, considerando que las edades fuera de
ese rango son ilógicas y no aportan nada al informe, por otro lado
tenemos el problema de que a priori hay registros repetidos para los
postulantes por los diferentes niveles educativos, por eso eliminamos
los repetidos, total nos basta con un registro por postulante.

    \begin{Verbatim}[commandchars=\\\{\}]
{\color{incolor}In [{\color{incolor}21}]:} \PY{c+c1}{\PYZsh{}\PYZsh{} MOSTRAMOS LOS RANGOS DE EDAD SIN TENER POSTULANTES REPETIDOS.}
         \PY{n}{rangos\PYZus{}edad} \PY{o}{=} \PY{n}{postulantes\PYZus{}unicos}\PY{p}{[}\PY{l+s+s1}{\PYZsq{}}\PY{l+s+s1}{rango\PYZus{}edad}\PY{l+s+s1}{\PYZsq{}}\PY{p}{]}\PY{o}{.}\PY{n}{value\PYZus{}counts}\PY{p}{(}\PY{p}{)}
         \PY{n}{g} \PY{o}{=} \PY{n}{sns}\PY{o}{.}\PY{n}{barplot}\PY{p}{(}\PY{n}{x}\PY{o}{=}\PY{n}{rangos\PYZus{}edad}\PY{o}{.}\PY{n}{values}\PY{p}{,} \PY{n}{y}\PY{o}{=}\PY{n}{rangos\PYZus{}edad}\PY{o}{.}\PY{n}{index}\PY{p}{,} \PY{n}{orient}\PY{o}{=}\PY{l+s+s1}{\PYZsq{}}\PY{l+s+s1}{h}\PY{l+s+s1}{\PYZsq{}}\PY{p}{)}
         \PY{n}{g}\PY{o}{.}\PY{n}{set\PYZus{}title}\PY{p}{(}\PY{l+s+s2}{\PYZdq{}}\PY{l+s+s2}{Usuarios por Rango de Edad}\PY{l+s+s2}{\PYZdq{}}\PY{p}{,} \PY{n}{fontsize}\PY{o}{=}\PY{l+m+mi}{18}\PY{p}{)}
         \PY{n}{g}\PY{o}{.}\PY{n}{set\PYZus{}xlabel}\PY{p}{(}\PY{l+s+s2}{\PYZdq{}}\PY{l+s+s2}{Cantidad de Usuarios}\PY{l+s+s2}{\PYZdq{}}\PY{p}{,} \PY{n}{fontsize}\PY{o}{=}\PY{l+m+mi}{18}\PY{p}{)}
         \PY{n}{g}\PY{o}{.}\PY{n}{set\PYZus{}ylabel}\PY{p}{(}\PY{l+s+s2}{\PYZdq{}}\PY{l+s+s2}{Rangos de Edad}\PY{l+s+s2}{\PYZdq{}}\PY{p}{,} \PY{n}{fontsize}\PY{o}{=}\PY{l+m+mi}{18}\PY{p}{)}
\end{Verbatim}


\begin{Verbatim}[commandchars=\\\{\}]
{\color{outcolor}Out[{\color{outcolor}21}]:} Text(0,0.5,'Rangos de Edad')
\end{Verbatim}
            
    \begin{center}
    \adjustimage{max size={0.9\linewidth}{0.9\paperheight}}{output_15_1.png}
    \end{center}
    { \hspace*{\fill} \\}
    
    \hypertarget{quuxe9-nivel-de-especializaciuxf3n-tienen-estos-postulantes}{%
\subsection{2.2 - ¿Qué nivel de especialización tienen estos
postulantes?}\label{quuxe9-nivel-de-especializaciuxf3n-tienen-estos-postulantes}}

En este caso lo que nos interesa ver es el tipo de especialización que
registran los diferentes usuarios, vale tener en cuenta que a diferencia
del caso anterior en este caso si nos interesan los registros repetidos
porque una persona puede tener varios tipos de especialización, como así
también diferentes niveles educativos y el hecho de tener alguna
especialización no quita la posibilidad o validez de tener otra. Por una
cuestión de calidad gráfica hemos decidido unificar los terciarios y los
técnicos (compartían un nombre conjunto) en solo terciarios para que
fuese más legible el gráfico.

    \begin{Verbatim}[commandchars=\\\{\}]
{\color{incolor}In [{\color{incolor}22}]:} \PY{c+c1}{\PYZsh{}\PYZsh{} GRAFICAMOS LOS DIFERENTES NIVELES EDUCATIVOS DE LOS USUARIOS.}
         \PY{n}{g} \PY{o}{=} \PY{n}{sns}\PY{o}{.}\PY{n}{countplot}\PY{p}{(}\PY{n}{x}\PY{o}{=}\PY{l+s+s2}{\PYZdq{}}\PY{l+s+s2}{nombre}\PY{l+s+s2}{\PYZdq{}}\PY{p}{,} \PY{n}{hue}\PY{o}{=}\PY{l+s+s2}{\PYZdq{}}\PY{l+s+s2}{estado}\PY{l+s+s2}{\PYZdq{}}\PY{p}{,} \PY{n}{data}\PY{o}{=}\PY{n}{postulantes}\PY{p}{,} \PY{n}{palette}\PY{o}{=}\PY{l+s+s2}{\PYZdq{}}\PY{l+s+s2}{hls}\PY{l+s+s2}{\PYZdq{}}\PY{p}{)}
         \PY{n}{g}\PY{o}{.}\PY{n}{set\PYZus{}title}\PY{p}{(}\PY{l+s+s2}{\PYZdq{}}\PY{l+s+s2}{Niveles educativos de los usuarios}\PY{l+s+s2}{\PYZdq{}}\PY{p}{,} \PY{n}{fontsize}\PY{o}{=}\PY{l+m+mi}{18}\PY{p}{)}
         \PY{n}{g}\PY{o}{.}\PY{n}{set\PYZus{}xlabel}\PY{p}{(}\PY{l+s+s2}{\PYZdq{}}\PY{l+s+s2}{Estudios}\PY{l+s+s2}{\PYZdq{}}\PY{p}{,} \PY{n}{fontsize}\PY{o}{=}\PY{l+m+mi}{18}\PY{p}{)}
         \PY{n}{g}\PY{o}{.}\PY{n}{set\PYZus{}ylabel}\PY{p}{(}\PY{l+s+s2}{\PYZdq{}}\PY{l+s+s2}{Cantidad de Usuarios}\PY{l+s+s2}{\PYZdq{}}\PY{p}{,} \PY{n}{fontsize}\PY{o}{=}\PY{l+m+mi}{18}\PY{p}{)}
\end{Verbatim}


\begin{Verbatim}[commandchars=\\\{\}]
{\color{outcolor}Out[{\color{outcolor}22}]:} Text(0,0.5,'Cantidad de Usuarios')
\end{Verbatim}
            
    \begin{center}
    \adjustimage{max size={0.9\linewidth}{0.9\paperheight}}{output_17_1.png}
    \end{center}
    { \hspace*{\fill} \\}
    
    \hypertarget{cual-es-la-proporciuxf3n-entre-los-guxe9neros-de-los-usuarios-registrados}{%
\subsection{2.3 - ¿Cual es la proporción entre los géneros de los
usuarios
registrados?}\label{cual-es-la-proporciuxf3n-entre-los-guxe9neros-de-los-usuarios-registrados}}

Lo que buscamos ver en este gráfico es la proporción que tenemos entre
los diferentes postulantes, como así también ver el margen que tenemos
de usuarios que no han especificado su género, esto nos permite a futuro
ver por ejemplo que trabajos son más vistos o postulados por los hombres
o por las mujeres, como así también ver si se mantiene un orden parejo
en su proporcionalidad. Por otro lado podemos ver si es relevante la
cantidad de usuarios que no han especificado su género, lo que nos
podría cambiar el enfoque si fuese un valor proporcionalmente alto.

    \begin{Verbatim}[commandchars=\\\{\}]
{\color{incolor}In [{\color{incolor}10}]:} \PY{c+c1}{\PYZsh{}\PYZsh{} GRAFICAMOS LA PROPORCIÓN DE GÉNERO ENTRE LOS POSTULANTES.}
         \PY{n}{postulantes\PYZus{}masculinos} \PY{o}{=} \PY{n}{postulantes\PYZus{}unicos}\PY{p}{[}\PY{n}{postulantes\PYZus{}unicos}\PY{p}{[}\PY{l+s+s1}{\PYZsq{}}\PY{l+s+s1}{sexo}\PY{l+s+s1}{\PYZsq{}}\PY{p}{]} \PY{o}{==} \PY{l+s+s1}{\PYZsq{}}\PY{l+s+s1}{MASC}\PY{l+s+s1}{\PYZsq{}}\PY{p}{]}
         \PY{n}{postulantes\PYZus{}femeninos} \PY{o}{=} \PY{n}{postulantes\PYZus{}unicos}\PY{p}{[}\PY{n}{postulantes\PYZus{}unicos}\PY{p}{[}\PY{l+s+s1}{\PYZsq{}}\PY{l+s+s1}{sexo}\PY{l+s+s1}{\PYZsq{}}\PY{p}{]} \PY{o}{==} \PY{l+s+s1}{\PYZsq{}}\PY{l+s+s1}{FEM}\PY{l+s+s1}{\PYZsq{}}\PY{p}{]}
         \PY{n}{sizes} \PY{o}{=} \PY{p}{[}\PY{n}{postulantes\PYZus{}masculinos}\PY{o}{.}\PY{n}{sexo}\PY{o}{.}\PY{n}{count}\PY{p}{(}\PY{p}{)}\PY{p}{,} \PY{n}{postulantes\PYZus{}femeninos}\PY{o}{.}\PY{n}{sexo}\PY{o}{.}\PY{n}{count}\PY{p}{(}\PY{p}{)}\PY{p}{]}
         \PY{n}{nombres} \PY{o}{=} \PY{p}{[}\PY{l+s+s1}{\PYZsq{}}\PY{l+s+s1}{Masculino}\PY{l+s+s1}{\PYZsq{}}\PY{p}{,} \PY{l+s+s1}{\PYZsq{}}\PY{l+s+s1}{Femenino}\PY{l+s+s1}{\PYZsq{}}\PY{p}{]}
         \PY{n}{plt}\PY{o}{.}\PY{n}{figure}\PY{p}{(}\PY{n}{figsize}\PY{o}{=}\PY{p}{(}\PY{l+m+mi}{8}\PY{p}{,} \PY{l+m+mi}{8}\PY{p}{)}\PY{p}{)}
         \PY{n}{plt}\PY{o}{.}\PY{n}{title}\PY{p}{(}\PY{l+s+s1}{\PYZsq{}}\PY{l+s+s1}{Distribucion sobre el sexo de los postulantes}\PY{l+s+s1}{\PYZsq{}}\PY{p}{,} \PY{n}{fontsize}\PY{o}{=}\PY{l+m+mi}{20}\PY{p}{)}
         \PY{n}{plt}\PY{o}{.}\PY{n}{pie}\PY{p}{(}\PY{n}{sizes}\PY{p}{,} \PY{n}{labels}\PY{o}{=}\PY{n}{nombres}\PY{p}{,} \PY{n}{autopct}\PY{o}{=}\PY{l+s+s1}{\PYZsq{}}\PY{l+s+si}{\PYZpc{}1.1f}\PY{l+s+si}{\PYZpc{}\PYZpc{}}\PY{l+s+s1}{\PYZsq{}}\PY{p}{,} \PY{n}{startangle}\PY{o}{=}\PY{l+m+mi}{20}\PY{p}{,} \PY{n}{colors}\PY{o}{=}\PY{p}{[}\PY{l+s+s1}{\PYZsq{}}\PY{l+s+s1}{lightgreen}\PY{l+s+s1}{\PYZsq{}}\PY{p}{,} \PY{l+s+s1}{\PYZsq{}}\PY{l+s+s1}{lightblue}\PY{l+s+s1}{\PYZsq{}}\PY{p}{]}\PY{p}{,} \PY{n}{explode}\PY{o}{=}\PY{p}{(}\PY{l+m+mf}{0.1}\PY{p}{,} \PY{l+m+mi}{0}\PY{p}{)}\PY{p}{)}
         \PY{n}{plt}\PY{o}{.}\PY{n}{show}\PY{p}{(}\PY{p}{)}
\end{Verbatim}


    \begin{center}
    \adjustimage{max size={0.9\linewidth}{0.9\paperheight}}{output_19_0.png}
    \end{center}
    { \hspace*{\fill} \\}
    
    \hypertarget{hay-relevancia-por-parte-de-los-usuarios-que-no-han-registrado-su-guxe9nero}{%
\subsection{2.4 - ¿Hay relevancia por parte de los usuarios que no han
registrado su
género?}\label{hay-relevancia-por-parte-de-los-usuarios-que-no-han-registrado-su-guxe9nero}}

Buscamos corroborar la relevancia que puedan tener los usuarios que no
han registrado su género, como así también ver si se encuentra algún
factor clave en relación a esto, siempre y cuando su valor sea
verdaderamente apreciable para dedicarle un análisis con mayor
profundidad. Como podemos observar el valor es muy pequeño en relación
al resto de los valores lo que no nos aporta nada de utilidad para un
tratamiento especial sobre el tema.

    \begin{Verbatim}[commandchars=\\\{\}]
{\color{incolor}In [{\color{incolor}11}]:} \PY{c+c1}{\PYZsh{}\PYZsh{} GRAFICAMOS LOS USUARIOS SIN GÉNERO.}
         \PY{n}{postulantes\PYZus{}indefinidos} \PY{o}{=} \PY{n}{postulantes\PYZus{}unicos}\PY{p}{[}\PY{n}{postulantes\PYZus{}unicos}\PY{p}{[}\PY{l+s+s1}{\PYZsq{}}\PY{l+s+s1}{sexo}\PY{l+s+s1}{\PYZsq{}}\PY{p}{]} \PY{o}{==} \PY{l+s+s1}{\PYZsq{}}\PY{l+s+s1}{NO\PYZus{}DECLARA}\PY{l+s+s1}{\PYZsq{}}\PY{p}{]}
         \PY{n}{postulantes\PYZus{}indefinidos}\PY{o}{.}\PY{n}{sexo}\PY{o}{.}\PY{n}{count}\PY{p}{(}\PY{p}{)}
\end{Verbatim}


\begin{Verbatim}[commandchars=\\\{\}]
{\color{outcolor}Out[{\color{outcolor}11}]:} 444
\end{Verbatim}
            
    \hypertarget{section}{%
\subsection{======================================================================}\label{section}}

    \hypertarget{anuxe1lisis-sobre-los-avisos-publicados}{%
\section{3 - Análisis sobre los avisos
publicados:}\label{anuxe1lisis-sobre-los-avisos-publicados}}

    \hypertarget{quuxe9-proporciuxf3n-tenemos-entre-las-zonas-que-afectan-los-avisos}{%
\subsection{3.1 - ¿Qué proporción tenemos entre las zonas que afectan
los
avisos?}\label{quuxe9-proporciuxf3n-tenemos-entre-las-zonas-que-afectan-los-avisos}}

En un primer apartado buscaremos ver que relación hay entre las zonas
afectadas, a pesar de no disponer de mayor detalle, al menos podemos ver
que cuantos avisos están orientados al Gran Buenos Aires y cuales a la
Ciudad Autónoma de Buenos Aires, lo que nos servirá como un primer
acercamiento a la información.

    \begin{Verbatim}[commandchars=\\\{\}]
{\color{incolor}In [{\color{incolor}12}]:} \PY{c+c1}{\PYZsh{}\PYZsh{} GRAFICAMOS LA PROPORCIÓN ENTRE LAS ZONAS.}
         \PY{n}{avisos\PYZus{}caba} \PY{o}{=} \PY{n}{avisos\PYZus{}detalle\PYZus{}completo}\PY{p}{[}\PY{n}{avisos\PYZus{}detalle\PYZus{}completo}\PY{p}{[}\PY{l+s+s1}{\PYZsq{}}\PY{l+s+s1}{nombre\PYZus{}zona}\PY{l+s+s1}{\PYZsq{}}\PY{p}{]} \PY{o}{==} \PY{l+s+s1}{\PYZsq{}}\PY{l+s+s1}{Capital Federal}\PY{l+s+s1}{\PYZsq{}}\PY{p}{]}
         \PY{n}{avisos\PYZus{}bsas} \PY{o}{=} \PY{n}{avisos\PYZus{}detalle\PYZus{}completo}\PY{p}{[}\PY{n}{avisos\PYZus{}detalle\PYZus{}completo}\PY{p}{[}\PY{l+s+s1}{\PYZsq{}}\PY{l+s+s1}{nombre\PYZus{}zona}\PY{l+s+s1}{\PYZsq{}}\PY{p}{]} \PY{o}{==} \PY{l+s+s1}{\PYZsq{}}\PY{l+s+s1}{Gran Buenos Aires}\PY{l+s+s1}{\PYZsq{}}\PY{p}{]}
         \PY{n}{sizes} \PY{o}{=} \PY{p}{[}\PY{n}{avisos\PYZus{}caba}\PY{o}{.}\PY{n}{nombre\PYZus{}zona}\PY{o}{.}\PY{n}{count}\PY{p}{(}\PY{p}{)}\PY{p}{,} \PY{n}{avisos\PYZus{}bsas}\PY{o}{.}\PY{n}{nombre\PYZus{}zona}\PY{o}{.}\PY{n}{count}\PY{p}{(}\PY{p}{)}\PY{p}{]}
         \PY{n}{nombres} \PY{o}{=} \PY{p}{[}\PY{l+s+s1}{\PYZsq{}}\PY{l+s+s1}{CABA}\PY{l+s+s1}{\PYZsq{}}\PY{p}{,} \PY{l+s+s1}{\PYZsq{}}\PY{l+s+s1}{GBA}\PY{l+s+s1}{\PYZsq{}}\PY{p}{]}
         \PY{n}{plt}\PY{o}{.}\PY{n}{figure}\PY{p}{(}\PY{n}{figsize}\PY{o}{=}\PY{p}{(}\PY{l+m+mi}{8}\PY{p}{,} \PY{l+m+mi}{8}\PY{p}{)}\PY{p}{)}
         \PY{n}{plt}\PY{o}{.}\PY{n}{title}\PY{p}{(}\PY{l+s+s1}{\PYZsq{}}\PY{l+s+s1}{Distribucion por las zonas referente a los avisos}\PY{l+s+s1}{\PYZsq{}}\PY{p}{,} \PY{n}{fontsize}\PY{o}{=}\PY{l+m+mi}{20}\PY{p}{)}
         \PY{n}{plt}\PY{o}{.}\PY{n}{pie}\PY{p}{(}\PY{n}{sizes}\PY{p}{,} \PY{n}{labels}\PY{o}{=}\PY{n}{nombres}\PY{p}{,} \PY{n}{autopct}\PY{o}{=}\PY{l+s+s1}{\PYZsq{}}\PY{l+s+si}{\PYZpc{}1.1f}\PY{l+s+si}{\PYZpc{}\PYZpc{}}\PY{l+s+s1}{\PYZsq{}}\PY{p}{,} \PY{n}{startangle}\PY{o}{=}\PY{l+m+mi}{20}\PY{p}{,} \PY{n}{colors}\PY{o}{=}\PY{p}{[}\PY{l+s+s1}{\PYZsq{}}\PY{l+s+s1}{lightgreen}\PY{l+s+s1}{\PYZsq{}}\PY{p}{,} \PY{l+s+s1}{\PYZsq{}}\PY{l+s+s1}{lightblue}\PY{l+s+s1}{\PYZsq{}}\PY{p}{]}\PY{p}{,} \PY{n}{explode}\PY{o}{=}\PY{p}{(}\PY{l+m+mf}{0.1}\PY{p}{,} \PY{l+m+mi}{0}\PY{p}{)}\PY{p}{)}
         \PY{n}{plt}\PY{o}{.}\PY{n}{show}\PY{p}{(}\PY{p}{)}
\end{Verbatim}


    \begin{center}
    \adjustimage{max size={0.9\linewidth}{0.9\paperheight}}{output_25_0.png}
    \end{center}
    { \hspace*{\fill} \\}
    
    \hypertarget{cuales-son-los-tipo-de-trabajo-muxe1s-requeridos-full-time-part-time}{%
\subsection{3.2 - ¿Cuales son los tipo de trabajo más requeridos,
full-time,
part-time?}\label{cuales-son-los-tipo-de-trabajo-muxe1s-requeridos-full-time-part-time}}

Listaremos los tipos de trabajo más buscados por las empresas y sus
avisos, en primer instancia por lógica sabemos que hay dos que son los
que más avisos tienen que son aquellos del tipo full-time y los del tipo
part-time, pero también hay otros tipos y sería interesante ver que
relación hay con el resto, son relevantes o no. Como podremos observar
el trabajo full-time es por lejos el tipo de trabajo que predomina el
mercado.

    \begin{Verbatim}[commandchars=\\\{\}]
{\color{incolor}In [{\color{incolor}13}]:} \PY{c+c1}{\PYZsh{}\PYZsh{} GRAFICAMOS LAS ÁREAS CON MAYOR RELEVANCIA A LA HORA DE VISITAR LOS AVISOS.}
         \PY{n}{avisos\PYZus{}tipo} \PY{o}{=} \PY{n}{avisos\PYZus{}detalle\PYZus{}completo}\PY{p}{[}\PY{l+s+s1}{\PYZsq{}}\PY{l+s+s1}{tipo\PYZus{}de\PYZus{}trabajo}\PY{l+s+s1}{\PYZsq{}}\PY{p}{]}\PY{o}{.}\PY{n}{value\PYZus{}counts}\PY{p}{(}\PY{p}{)}\PY{p}{[}\PY{p}{:}\PY{l+m+mi}{5}\PY{p}{]}
         \PY{n}{g} \PY{o}{=} \PY{n}{sns}\PY{o}{.}\PY{n}{barplot}\PY{p}{(}\PY{n}{x}\PY{o}{=} \PY{n}{avisos\PYZus{}tipo}\PY{o}{.}\PY{n}{values}\PY{p}{,} \PY{n}{y}\PY{o}{=}\PY{n}{avisos\PYZus{}tipo}\PY{o}{.}\PY{n}{index}\PY{p}{)}
         \PY{n}{g}\PY{o}{.}\PY{n}{set\PYZus{}title}\PY{p}{(}\PY{l+s+s2}{\PYZdq{}}\PY{l+s+s2}{TOP 5 tipos de trabajo relevantes.}\PY{l+s+s2}{\PYZdq{}}\PY{p}{,} \PY{n}{fontsize}\PY{o}{=}\PY{l+m+mi}{18}\PY{p}{)}
         \PY{n}{g}\PY{o}{.}\PY{n}{set\PYZus{}xlabel}\PY{p}{(}\PY{l+s+s2}{\PYZdq{}}\PY{l+s+s2}{Cantidad de aviisos}\PY{l+s+s2}{\PYZdq{}}\PY{p}{,} \PY{n}{fontsize}\PY{o}{=}\PY{l+m+mi}{18}\PY{p}{)}
         \PY{n}{g}\PY{o}{.}\PY{n}{set\PYZus{}ylabel}\PY{p}{(}\PY{l+s+s2}{\PYZdq{}}\PY{l+s+s2}{Tipo de trabajo}\PY{l+s+s2}{\PYZdq{}}\PY{p}{,} \PY{n}{fontsize}\PY{o}{=}\PY{l+m+mi}{18}\PY{p}{)}
\end{Verbatim}


\begin{Verbatim}[commandchars=\\\{\}]
{\color{outcolor}Out[{\color{outcolor}13}]:} Text(0,0.5,'Tipo de trabajo')
\end{Verbatim}
            
    \begin{center}
    \adjustimage{max size={0.9\linewidth}{0.9\paperheight}}{output_27_1.png}
    \end{center}
    { \hspace*{\fill} \\}
    
    \hypertarget{quuxe9-proporciuxf3n-tenemos-entre-los-avisos-que-estuxe1n-online-y-los-offline}{%
\subsection{3.3 - ¿Qué proporción tenemos entre los avisos que están
online y los
offline?}\label{quuxe9-proporciuxf3n-tenemos-entre-los-avisos-que-estuxe1n-online-y-los-offline}}

Buscamos una relación/proporción entre los diferentes avisos, ya que a
priori sabemos que hay dos estados posible, tanto el online como el
offline, esto nos podría indicar que sea un valor parejo entre ambos o
que alguna de estas cantidades sea mucho mayor que la otra lo que nos
llevaría a tomar uno u otro enfoque a la hora de proseguir con el
análisis.

    \begin{Verbatim}[commandchars=\\\{\}]
{\color{incolor}In [{\color{incolor}14}]:} \PY{c+c1}{\PYZsh{}\PYZsh{} GRAFICAMOS LA PROPORCIÓN ENTRE EL ONLINE Y EL OFFLINE.}
         \PY{n}{avisos\PYZus{}online} \PY{o}{=} \PY{n}{avisos\PYZus{}detalle\PYZus{}completo}\PY{p}{[}\PY{n}{avisos\PYZus{}detalle\PYZus{}completo}\PY{p}{[}\PY{l+s+s1}{\PYZsq{}}\PY{l+s+s1}{esta\PYZus{}online}\PY{l+s+s1}{\PYZsq{}}\PY{p}{]} \PY{o}{==} \PY{l+s+s1}{\PYZsq{}}\PY{l+s+s1}{si}\PY{l+s+s1}{\PYZsq{}}\PY{p}{]}
         \PY{n}{avisos\PYZus{}offline} \PY{o}{=} \PY{n}{avisos\PYZus{}detalle\PYZus{}completo}\PY{p}{[}\PY{n}{avisos\PYZus{}detalle\PYZus{}completo}\PY{p}{[}\PY{l+s+s1}{\PYZsq{}}\PY{l+s+s1}{esta\PYZus{}online}\PY{l+s+s1}{\PYZsq{}}\PY{p}{]} \PY{o}{!=} \PY{l+s+s1}{\PYZsq{}}\PY{l+s+s1}{si}\PY{l+s+s1}{\PYZsq{}}\PY{p}{]}
         \PY{n}{sizes} \PY{o}{=} \PY{p}{[}\PY{n}{avisos\PYZus{}online}\PY{o}{.}\PY{n}{esta\PYZus{}online}\PY{o}{.}\PY{n}{count}\PY{p}{(}\PY{p}{)}\PY{p}{,} \PY{n}{avisos\PYZus{}offline}\PY{o}{.}\PY{n}{esta\PYZus{}online}\PY{o}{.}\PY{n}{count}\PY{p}{(}\PY{p}{)}\PY{p}{]}
         \PY{n}{nombres} \PY{o}{=} \PY{p}{[}\PY{l+s+s1}{\PYZsq{}}\PY{l+s+s1}{Online}\PY{l+s+s1}{\PYZsq{}}\PY{p}{,} \PY{l+s+s1}{\PYZsq{}}\PY{l+s+s1}{Offline}\PY{l+s+s1}{\PYZsq{}}\PY{p}{]}
         \PY{n}{plt}\PY{o}{.}\PY{n}{figure}\PY{p}{(}\PY{n}{figsize}\PY{o}{=}\PY{p}{(}\PY{l+m+mi}{8}\PY{p}{,} \PY{l+m+mi}{8}\PY{p}{)}\PY{p}{)}
         \PY{n}{plt}\PY{o}{.}\PY{n}{title}\PY{p}{(}\PY{l+s+s1}{\PYZsq{}}\PY{l+s+s1}{Distribucion sobre los avisos online/offline}\PY{l+s+s1}{\PYZsq{}}\PY{p}{,} \PY{n}{fontsize}\PY{o}{=}\PY{l+m+mi}{20}\PY{p}{)}
         \PY{n}{plt}\PY{o}{.}\PY{n}{pie}\PY{p}{(}\PY{n}{sizes}\PY{p}{,} \PY{n}{labels}\PY{o}{=}\PY{n}{nombres}\PY{p}{,} \PY{n}{autopct}\PY{o}{=}\PY{l+s+s1}{\PYZsq{}}\PY{l+s+si}{\PYZpc{}1.1f}\PY{l+s+si}{\PYZpc{}\PYZpc{}}\PY{l+s+s1}{\PYZsq{}}\PY{p}{,} \PY{n}{startangle}\PY{o}{=}\PY{l+m+mi}{20}\PY{p}{,} \PY{n}{colors}\PY{o}{=}\PY{p}{[}\PY{l+s+s1}{\PYZsq{}}\PY{l+s+s1}{lightgreen}\PY{l+s+s1}{\PYZsq{}}\PY{p}{,} \PY{l+s+s1}{\PYZsq{}}\PY{l+s+s1}{lightblue}\PY{l+s+s1}{\PYZsq{}}\PY{p}{]}\PY{p}{,} \PY{n}{explode}\PY{o}{=}\PY{p}{(}\PY{l+m+mf}{0.1}\PY{p}{,} \PY{l+m+mi}{0}\PY{p}{)}\PY{p}{)}
         \PY{n}{plt}\PY{o}{.}\PY{n}{show}\PY{p}{(}\PY{p}{)}
\end{Verbatim}


    \begin{center}
    \adjustimage{max size={0.9\linewidth}{0.9\paperheight}}{output_29_0.png}
    \end{center}
    { \hspace*{\fill} \\}
    
    \hypertarget{quuxe9-proporciuxf3n-tenemos-avisos-para-juniors-y-para-srssr}{%
\subsection{3.4 - ¿Qué proporción tenemos avisos para Juniors y para
Sr/Ssr?}\label{quuxe9-proporciuxf3n-tenemos-avisos-para-juniors-y-para-srssr}}

La idea es ver que relación hay entre los juniors y los senior o
semi-senior como para corroborar si se está buscando mucha más gente con
mayor experiencia y capacitación o si se busca gente con menos
experiencia y/o menor capacitación, esto lo podemos vincular o
relacionar con lo que hemos estudiado para los usuarios registrados y
sus rangos de edad y sus niveles educativos.

    \begin{Verbatim}[commandchars=\\\{\}]
{\color{incolor}In [{\color{incolor}21}]:} \PY{c+c1}{\PYZsh{}\PYZsh{} GRAFICAMOS LA PROPORCIÓN ENTRE JR y SSR/SR}
         \PY{n}{g} \PY{o}{=} \PY{n}{sns}\PY{o}{.}\PY{n}{countplot}\PY{p}{(}\PY{n}{x}\PY{o}{=}\PY{l+s+s1}{\PYZsq{}}\PY{l+s+s1}{nivel\PYZus{}laboral\PYZus{}x}\PY{l+s+s1}{\PYZsq{}}\PY{p}{,} \PY{n}{data}\PY{o}{=}\PY{n}{avisos\PYZus{}detalle\PYZus{}completo}\PY{p}{,} \PY{n}{order}\PY{o}{=}\PY{n}{avisos\PYZus{}detalle\PYZus{}completo}\PY{p}{[}\PY{l+s+s1}{\PYZsq{}}\PY{l+s+s1}{nivel\PYZus{}laboral\PYZus{}x}\PY{l+s+s1}{\PYZsq{}}\PY{p}{]}\PY{o}{.}\PY{n}{value\PYZus{}counts}\PY{p}{(}\PY{p}{)}\PY{o}{.}\PY{n}{index}\PY{p}{,} \PY{n}{orient}\PY{o}{=}\PY{l+s+s1}{\PYZsq{}}\PY{l+s+s1}{v}\PY{l+s+s1}{\PYZsq{}}\PY{p}{)}
         \PY{n}{g}\PY{o}{.}\PY{n}{set\PYZus{}xticklabels}\PY{p}{(}\PY{n}{g}\PY{o}{.}\PY{n}{get\PYZus{}xticklabels}\PY{p}{(}\PY{p}{)}\PY{p}{,}\PY{n}{rotation}\PY{o}{=}\PY{l+m+mi}{90}\PY{p}{)}
         \PY{n}{g}\PY{o}{.}\PY{n}{set\PYZus{}xlabel}\PY{p}{(}\PY{l+s+s2}{\PYZdq{}}\PY{l+s+s2}{NIvel laboral}\PY{l+s+s2}{\PYZdq{}}\PY{p}{,} \PY{n}{fontsize}\PY{o}{=}\PY{l+m+mi}{20}\PY{p}{)}
         \PY{n}{g}\PY{o}{.}\PY{n}{set\PYZus{}ylabel}\PY{p}{(}\PY{l+s+s2}{\PYZdq{}}\PY{l+s+s2}{Cantidad de avisos}\PY{l+s+s2}{\PYZdq{}}\PY{p}{,} \PY{n}{fontsize}\PY{o}{=}\PY{l+m+mi}{18}\PY{p}{)}
         \PY{n}{g}\PY{o}{.}\PY{n}{set\PYZus{}title}\PY{p}{(}\PY{l+s+s2}{\PYZdq{}}\PY{l+s+s2}{Nivel laboral requerido en los avisos}\PY{l+s+s2}{\PYZdq{}}\PY{p}{,} \PY{n}{fontsize}\PY{o}{=}\PY{l+m+mi}{18}\PY{p}{)}
\end{Verbatim}


\begin{Verbatim}[commandchars=\\\{\}]
{\color{outcolor}Out[{\color{outcolor}21}]:} Text(0.5,1,'Nivel laboral requerido en los avisos')
\end{Verbatim}
            
    \begin{center}
    \adjustimage{max size={0.9\linewidth}{0.9\paperheight}}{output_31_1.png}
    \end{center}
    { \hspace*{\fill} \\}
    
    \hypertarget{section}{%
\subsection{======================================================================}\label{section}}

    \hypertarget{anuxe1lisis-sobre-las-visitas}{%
\section{4 - Análisis sobre las
visitas:}\label{anuxe1lisis-sobre-las-visitas}}

\hypertarget{quuxe9-duxedas-de-la-semana-se-ven-muxe1s-avisos}{%
\subsection{4.1 - ¿Qué días de la semana se ven más
avisos?}\label{quuxe9-duxedas-de-la-semana-se-ven-muxe1s-avisos}}

La idea es tener ordenada la semana y ver cómo es que hay días más
influyentes a la hora de visitar anuncios, esto es independiente a los
postulaciones o incluso a los avisos, porque nos da la pauta de que días
la gente ingresa a revisar con mayor frecuencia los diferentes avisos,
sería interesante para las empresas el saber que día en la semana es el
más propicio para iniciar una búsqueda laboral.

    \begin{Verbatim}[commandchars=\\\{\}]
{\color{incolor}In [{\color{incolor}23}]:} \PY{c+c1}{\PYZsh{}\PYZsh{} GRAFICAMOS LOS DÍAS DE LA SEMANA DONDE SE VISITAN MÁS AVISOS.}
         \PY{n}{g} \PY{o}{=} \PY{n}{sns}\PY{o}{.}\PY{n}{countplot}\PY{p}{(}\PY{n}{x}\PY{o}{=}\PY{l+s+s1}{\PYZsq{}}\PY{l+s+s1}{dia\PYZus{}semana}\PY{l+s+s1}{\PYZsq{}}\PY{p}{,} \PY{n}{data}\PY{o}{=}\PY{n}{vistas\PYZus{}avisos}\PY{p}{,} \PY{n}{order}\PY{o}{=}\PY{n}{vistas\PYZus{}avisos}\PY{p}{[}\PY{l+s+s1}{\PYZsq{}}\PY{l+s+s1}{dia\PYZus{}semana}\PY{l+s+s1}{\PYZsq{}}\PY{p}{]}\PY{o}{.}\PY{n}{value\PYZus{}counts}\PY{p}{(}\PY{p}{)}\PY{o}{.}\PY{n}{index}\PY{p}{,} \PY{n}{orient}\PY{o}{=}\PY{l+s+s1}{\PYZsq{}}\PY{l+s+s1}{v}\PY{l+s+s1}{\PYZsq{}}\PY{p}{)}
         \PY{n}{g}\PY{o}{.}\PY{n}{set\PYZus{}xticklabels}\PY{p}{(}\PY{n}{g}\PY{o}{.}\PY{n}{get\PYZus{}xticklabels}\PY{p}{(}\PY{p}{)}\PY{p}{,}\PY{n}{rotation}\PY{o}{=}\PY{l+m+mi}{90}\PY{p}{)}
         \PY{n}{g}\PY{o}{.}\PY{n}{set\PYZus{}xlabel}\PY{p}{(}\PY{l+s+s2}{\PYZdq{}}\PY{l+s+s2}{Dia de la semana}\PY{l+s+s2}{\PYZdq{}}\PY{p}{,} \PY{n}{fontsize}\PY{o}{=}\PY{l+m+mi}{18}\PY{p}{)}
         \PY{n}{g}\PY{o}{.}\PY{n}{set\PYZus{}ylabel}\PY{p}{(}\PY{l+s+s2}{\PYZdq{}}\PY{l+s+s2}{Cantidad de vistas}\PY{l+s+s2}{\PYZdq{}}\PY{p}{,} \PY{n}{fontsize}\PY{o}{=}\PY{l+m+mi}{18}\PY{p}{)}
         \PY{n}{g}\PY{o}{.}\PY{n}{set\PYZus{}title}\PY{p}{(}\PY{l+s+s2}{\PYZdq{}}\PY{l+s+s2}{Vistas por dia de la semana}\PY{l+s+s2}{\PYZdq{}}\PY{p}{,} \PY{n}{fontsize}\PY{o}{=}\PY{l+m+mi}{18}\PY{p}{)}
\end{Verbatim}


\begin{Verbatim}[commandchars=\\\{\}]
{\color{outcolor}Out[{\color{outcolor}23}]:} Text(0.5,1,'Vistas por dia de la semana')
\end{Verbatim}
            
    \begin{center}
    \adjustimage{max size={0.9\linewidth}{0.9\paperheight}}{output_34_1.png}
    \end{center}
    { \hspace*{\fill} \\}
    
    \hypertarget{cuales-son-las-areas-con-mayor-cantidad-de-visitas}{%
\subsection{4.2 - ¿Cuales son las areas con mayor cantidad de
visitas?}\label{cuales-son-las-areas-con-mayor-cantidad-de-visitas}}

Buscamos ver cuales son las areas que mayor interés traen para los
usuarios, aunque esto está vinculado directamente a la cantidad de
avisos que haya para cada área, pero de todas maneras nos puede ser de
utilidad para corroborar cuales son estas áreas y que relación tienen
con los avisos y con las postulaciones realizadas. Como bien podremos
ver las areas de Ventas y Administración son las que mayor relevancia
tienen y las que más visitas reciben, siendo las que predominan en el
mercado.

    \begin{Verbatim}[commandchars=\\\{\}]
{\color{incolor}In [{\color{incolor}24}]:} \PY{c+c1}{\PYZsh{}\PYZsh{} GRAFICAMOS LAS ÁREAS CON MAYOR RELEVANCIA A LA HORA DE VISITAR LOS AVISOS.}
         \PY{n}{g} \PY{o}{=} \PY{n}{sns}\PY{o}{.}\PY{n}{barplot}\PY{p}{(}\PY{n}{x}\PY{o}{=} \PY{n}{top\PYZus{}visitados\PYZus{}area}\PY{o}{.}\PY{n}{values}\PY{p}{,} \PY{n}{y}\PY{o}{=}\PY{n}{top\PYZus{}visitados\PYZus{}area}\PY{o}{.}\PY{n}{index}\PY{p}{)}
         \PY{n}{g}\PY{o}{.}\PY{n}{set\PYZus{}title}\PY{p}{(}\PY{l+s+s2}{\PYZdq{}}\PY{l+s+s2}{TOP 10 Areas con mas visitas en sus avisos}\PY{l+s+s2}{\PYZdq{}}\PY{p}{,} \PY{n}{fontsize}\PY{o}{=}\PY{l+m+mi}{18}\PY{p}{)}
         \PY{n}{g}\PY{o}{.}\PY{n}{set\PYZus{}xlabel}\PY{p}{(}\PY{l+s+s2}{\PYZdq{}}\PY{l+s+s2}{Cantidad de visitas}\PY{l+s+s2}{\PYZdq{}}\PY{p}{,} \PY{n}{fontsize}\PY{o}{=}\PY{l+m+mi}{18}\PY{p}{)}
         \PY{n}{g}\PY{o}{.}\PY{n}{set\PYZus{}ylabel}\PY{p}{(}\PY{l+s+s2}{\PYZdq{}}\PY{l+s+s2}{Area}\PY{l+s+s2}{\PYZdq{}}\PY{p}{,} \PY{n}{fontsize}\PY{o}{=}\PY{l+m+mi}{18}\PY{p}{)}
\end{Verbatim}


\begin{Verbatim}[commandchars=\\\{\}]
{\color{outcolor}Out[{\color{outcolor}24}]:} Text(0,0.5,'Area')
\end{Verbatim}
            
    \begin{center}
    \adjustimage{max size={0.9\linewidth}{0.9\paperheight}}{output_36_1.png}
    \end{center}
    { \hspace*{\fill} \\}
    
    \hypertarget{cuales-vendruxedan-a-ser-las-ingenieruxedas-con-mayor-cantidad-de-visitas}{%
\subsection{4.3 - ¿Cuales vendrían a ser las ingenierías con mayor
cantidad de
visitas?}\label{cuales-vendruxedan-a-ser-las-ingenieruxedas-con-mayor-cantidad-de-visitas}}

Podemos tratar de hacer un paralelismo sobre las áreas en Ingeniería,
reduciendo el ámbito de estudio solo a las que se relacionan con
ingenieria o ramas afines, esto nos lleva a una conclusión bastante
particular y es que la Ingeniería Industrial es la que más relevancia
parece tener a la hora de visitar los avisos.

    \begin{Verbatim}[commandchars=\\\{\}]
{\color{incolor}In [{\color{incolor}25}]:} \PY{c+c1}{\PYZsh{}\PYZsh{} GRAFICAMOS LAS DIEZ INGENIERÍAS CON MAYOR RELEVANCIA A LA HORA DE VISITAR LOS AVISOS.}
         \PY{n}{g} \PY{o}{=} \PY{n}{sns}\PY{o}{.}\PY{n}{barplot}\PY{p}{(}\PY{n}{x}\PY{o}{=} \PY{n}{diez\PYZus{}ingenierias}\PY{o}{.}\PY{n}{values}\PY{p}{,} \PY{n}{y}\PY{o}{=}\PY{n}{diez\PYZus{}ingenierias}\PY{o}{.}\PY{n}{index}\PY{p}{)}
         \PY{n}{g}\PY{o}{.}\PY{n}{set\PYZus{}title}\PY{p}{(}\PY{l+s+s2}{\PYZdq{}}\PY{l+s+s2}{TOP 10 Avisos visitados s/ rama Ingenieril}\PY{l+s+s2}{\PYZdq{}}\PY{p}{,} \PY{n}{fontsize}\PY{o}{=}\PY{l+m+mi}{18}\PY{p}{)}
         \PY{n}{g}\PY{o}{.}\PY{n}{set\PYZus{}xlabel}\PY{p}{(}\PY{l+s+s2}{\PYZdq{}}\PY{l+s+s2}{Cantidad de visitas}\PY{l+s+s2}{\PYZdq{}}\PY{p}{,} \PY{n}{fontsize}\PY{o}{=}\PY{l+m+mi}{18}\PY{p}{)}
         \PY{n}{g}\PY{o}{.}\PY{n}{set\PYZus{}ylabel}\PY{p}{(}\PY{l+s+s2}{\PYZdq{}}\PY{l+s+s2}{Ingenieria en cuestion}\PY{l+s+s2}{\PYZdq{}}\PY{p}{,} \PY{n}{fontsize}\PY{o}{=}\PY{l+m+mi}{18}\PY{p}{)}
\end{Verbatim}


\begin{Verbatim}[commandchars=\\\{\}]
{\color{outcolor}Out[{\color{outcolor}25}]:} Text(0,0.5,'Ingenieria en cuestion')
\end{Verbatim}
            
    \begin{center}
    \adjustimage{max size={0.9\linewidth}{0.9\paperheight}}{output_38_1.png}
    \end{center}
    { \hspace*{\fill} \\}
    
    \hypertarget{quuxe9-porcentaje-de-los-avisos-evaluadosvisitados-estuxe1-online}{%
\subsection{4.4 - ¿Qué porcentaje de los avisos evaluados/visitados está
online?}\label{quuxe9-porcentaje-de-los-avisos-evaluadosvisitados-estuxe1-online}}

Lo que nos sirve ver en esta gráfica es ver que porcentaje de los avisos
que se visitan estan online, entendemos que el estado online de un aviso
hace referencia a si está activo a la hora en que las personas se han
decidido postular.

    \begin{Verbatim}[commandchars=\\\{\}]
{\color{incolor}In [{\color{incolor}31}]:} \PY{c+c1}{\PYZsh{}\PYZsh{} GRAFICAMOS LA PROPORCIÓN ENTRE AVISOS VISITADOS ONLINE Y OFFLINE.}
         \PY{n}{vistas\PYZus{}online} \PY{o}{=} \PY{n}{vistas\PYZus{}avisos\PYZus{}completas}\PY{p}{[}\PY{n}{vistas\PYZus{}avisos\PYZus{}completas}\PY{p}{[}\PY{l+s+s1}{\PYZsq{}}\PY{l+s+s1}{esta\PYZus{}online}\PY{l+s+s1}{\PYZsq{}}\PY{p}{]} \PY{o}{==} \PY{l+s+s1}{\PYZsq{}}\PY{l+s+s1}{si}\PY{l+s+s1}{\PYZsq{}}\PY{p}{]}
         \PY{n}{vistas\PYZus{}offline} \PY{o}{=} \PY{n}{vistas\PYZus{}avisos\PYZus{}completas}\PY{p}{[}\PY{n}{vistas\PYZus{}avisos\PYZus{}completas}\PY{p}{[}\PY{l+s+s1}{\PYZsq{}}\PY{l+s+s1}{esta\PYZus{}online}\PY{l+s+s1}{\PYZsq{}}\PY{p}{]} \PY{o}{!=} \PY{l+s+s1}{\PYZsq{}}\PY{l+s+s1}{si}\PY{l+s+s1}{\PYZsq{}}\PY{p}{]}
         \PY{n}{sizes} \PY{o}{=} \PY{p}{[}\PY{n}{vistas\PYZus{}online}\PY{o}{.}\PY{n}{esta\PYZus{}online}\PY{o}{.}\PY{n}{count}\PY{p}{(}\PY{p}{)}\PY{p}{,} \PY{n}{vistas\PYZus{}offline}\PY{o}{.}\PY{n}{esta\PYZus{}online}\PY{o}{.}\PY{n}{count}\PY{p}{(}\PY{p}{)}\PY{p}{]}
         \PY{n}{nombres} \PY{o}{=} \PY{p}{[}\PY{l+s+s1}{\PYZsq{}}\PY{l+s+s1}{Online}\PY{l+s+s1}{\PYZsq{}}\PY{p}{,} \PY{l+s+s1}{\PYZsq{}}\PY{l+s+s1}{Offline}\PY{l+s+s1}{\PYZsq{}}\PY{p}{]}
         \PY{n}{plt}\PY{o}{.}\PY{n}{figure}\PY{p}{(}\PY{n}{figsize}\PY{o}{=}\PY{p}{(}\PY{l+m+mi}{8}\PY{p}{,} \PY{l+m+mi}{8}\PY{p}{)}\PY{p}{)}
         \PY{n}{plt}\PY{o}{.}\PY{n}{title}\PY{p}{(}\PY{l+s+s1}{\PYZsq{}}\PY{l+s+s1}{Distribucion sobre las visitas a los avisos online/offline}\PY{l+s+s1}{\PYZsq{}}\PY{p}{,} \PY{n}{fontsize}\PY{o}{=}\PY{l+m+mi}{20}\PY{p}{)}
         \PY{n}{plt}\PY{o}{.}\PY{n}{pie}\PY{p}{(}\PY{n}{sizes}\PY{p}{,} \PY{n}{labels}\PY{o}{=}\PY{n}{nombres}\PY{p}{,} \PY{n}{autopct}\PY{o}{=}\PY{l+s+s1}{\PYZsq{}}\PY{l+s+si}{\PYZpc{}1.1f}\PY{l+s+si}{\PYZpc{}\PYZpc{}}\PY{l+s+s1}{\PYZsq{}}\PY{p}{,} \PY{n}{startangle}\PY{o}{=}\PY{l+m+mi}{20}\PY{p}{,} \PY{n}{colors}\PY{o}{=}\PY{p}{[}\PY{l+s+s1}{\PYZsq{}}\PY{l+s+s1}{lightgreen}\PY{l+s+s1}{\PYZsq{}}\PY{p}{,} \PY{l+s+s1}{\PYZsq{}}\PY{l+s+s1}{lightblue}\PY{l+s+s1}{\PYZsq{}}\PY{p}{]}\PY{p}{,} \PY{n}{explode}\PY{o}{=}\PY{p}{(}\PY{l+m+mf}{0.1}\PY{p}{,} \PY{l+m+mi}{0}\PY{p}{)}\PY{p}{)}
         \PY{n}{plt}\PY{o}{.}\PY{n}{show}\PY{p}{(}\PY{p}{)}
\end{Verbatim}


    \begin{center}
    \adjustimage{max size={0.9\linewidth}{0.9\paperheight}}{output_40_0.png}
    \end{center}
    { \hspace*{\fill} \\}
    
    \hypertarget{cuales-son-las-empresas-que-reciben-mayor-cantidad-de-visitas-en-sus-avisos}{%
\subsection{4.5 - ¿Cuales son las empresas que reciben mayor cantidad de
visitas en sus
avisos?}\label{cuales-son-las-empresas-que-reciben-mayor-cantidad-de-visitas-en-sus-avisos}}

Sobre este punto hay que tener en cuenta que esto vinculará las empresas
que tengan mayor cantidad de visitas pero también se relaciona con
aquellas que tienen mayor cantidad de avisos, lo que llevará a tener por
lógica una cantidad mayor de visitas, por otro lado se trató de unificar
algunas empresas que se dividian en zonas pero que en definitiva eran la
misma empresa. Se ha decidido tomar las diez primeras como muestra de
relevancia.

    \begin{Verbatim}[commandchars=\\\{\}]
{\color{incolor}In [{\color{incolor}29}]:} \PY{c+c1}{\PYZsh{}\PYZsh{} CONTAMOS LAS DIEZ EMPRESAS CON MAYOR CANTIDAD DE VISITAS EN SUS AVISOS.}
         \PY{n}{g} \PY{o}{=} \PY{n}{sns}\PY{o}{.}\PY{n}{barplot}\PY{p}{(}\PY{n}{x}\PY{o}{=}\PY{n}{diez\PYZus{}vistas\PYZus{}empresas}\PY{o}{.}\PY{n}{values}\PY{p}{,} \PY{n}{y}\PY{o}{=}\PY{n}{diez\PYZus{}vistas\PYZus{}empresas}\PY{o}{.}\PY{n}{index}\PY{p}{)}
         \PY{n}{g}\PY{o}{.}\PY{n}{set\PYZus{}title}\PY{p}{(}\PY{l+s+s2}{\PYZdq{}}\PY{l+s+s2}{TOP 10 con mas visitas a sus avisos}\PY{l+s+s2}{\PYZdq{}}\PY{p}{,} \PY{n}{fontsize}\PY{o}{=}\PY{l+m+mi}{18}\PY{p}{)}
         \PY{n}{g}\PY{o}{.}\PY{n}{set\PYZus{}xlabel}\PY{p}{(}\PY{l+s+s2}{\PYZdq{}}\PY{l+s+s2}{Cantidad de visitas}\PY{l+s+s2}{\PYZdq{}}\PY{p}{,} \PY{n}{fontsize}\PY{o}{=}\PY{l+m+mi}{18}\PY{p}{)}
         \PY{n}{g}\PY{o}{.}\PY{n}{set\PYZus{}ylabel}\PY{p}{(}\PY{l+s+s2}{\PYZdq{}}\PY{l+s+s2}{Nombre de empresa}\PY{l+s+s2}{\PYZdq{}}\PY{p}{,} \PY{n}{fontsize}\PY{o}{=}\PY{l+m+mi}{18}\PY{p}{)}
\end{Verbatim}


\begin{Verbatim}[commandchars=\\\{\}]
{\color{outcolor}Out[{\color{outcolor}29}]:} Text(0,0.5,'Nombre de empresa')
\end{Verbatim}
            
    \begin{center}
    \adjustimage{max size={0.9\linewidth}{0.9\paperheight}}{output_42_1.png}
    \end{center}
    { \hspace*{\fill} \\}
    
    \hypertarget{quuxe9-guxe9nero-es-el-que-muxe1s-utiliza-los-servicios-de-la-empresa}{%
\subsection{4.6 - ¿Qué género es el que más utiliza los servicios de la
empresa?}\label{quuxe9-guxe9nero-es-el-que-muxe1s-utiliza-los-servicios-de-la-empresa}}

Buscamos ver que relación hay entre el género a la hora de visitar los
avisos, y ver si estos en definitiva están online u offline, pudiendo
así ver cúal es el comportamiento que se va desarrollando a entre las
personas, habiendo casos donde quizás una persona a revizado mas de 300
avisos en lo que puede parecer una actitud frenética o bien podría ser
un bot, son cosas que nos permiten analizar este tipo de situaciones o
al menos tener un dejo de duda sobre el mismo, por otro lado como es
lógico los valores donde tenemos mayor cantidad son entre 0 a 100.

    \begin{Verbatim}[commandchars=\\\{\}]
{\color{incolor}In [{\color{incolor}32}]:} \PY{c+c1}{\PYZsh{}\PYZsh{} AGRUPAMOS POR SEXO Y SEGÚN LA RELACIÓN DE SI EL AVISO ESTÁ ONLINE O NO Y COMO ES LA PROPORCIÓN DE VISITAS.}
         \PY{n}{info\PYZus{}postulantes} \PY{o}{=} \PY{n}{vistas\PYZus{}avisos\PYZus{}completas}\PY{o}{.}\PY{n}{groupby}\PY{p}{(}\PY{p}{[}\PY{l+s+s1}{\PYZsq{}}\PY{l+s+s1}{idpostulante}\PY{l+s+s1}{\PYZsq{}}\PY{p}{,} \PY{l+s+s1}{\PYZsq{}}\PY{l+s+s1}{sexo}\PY{l+s+s1}{\PYZsq{}}\PY{p}{,}\PY{l+s+s1}{\PYZsq{}}\PY{l+s+s1}{esta\PYZus{}online}\PY{l+s+s1}{\PYZsq{}}\PY{p}{]}\PY{p}{)}\PY{o}{.}\PY{n}{size}\PY{p}{(}\PY{p}{)}\PY{o}{.}\PY{n}{reset\PYZus{}index}\PY{p}{(}\PY{n}{name}\PY{o}{=}\PY{l+s+s1}{\PYZsq{}}\PY{l+s+s1}{counts}\PY{l+s+s1}{\PYZsq{}}\PY{p}{)}
         \PY{n}{info\PYZus{}postulantes} \PY{o}{=} \PY{n}{info\PYZus{}postulantes}\PY{p}{[}\PY{n}{info\PYZus{}postulantes}\PY{p}{[}\PY{l+s+s1}{\PYZsq{}}\PY{l+s+s1}{counts}\PY{l+s+s1}{\PYZsq{}}\PY{p}{]} \PY{o}{\PYZlt{}} \PY{l+m+mi}{300}\PY{p}{]}
         \PY{n}{g} \PY{o}{=} \PY{n}{sns}\PY{o}{.}\PY{n}{factorplot}\PY{p}{(}\PY{n}{x}\PY{o}{=}\PY{l+s+s2}{\PYZdq{}}\PY{l+s+s2}{sexo}\PY{l+s+s2}{\PYZdq{}}\PY{p}{,} \PY{n}{y}\PY{o}{=}\PY{l+s+s2}{\PYZdq{}}\PY{l+s+s2}{counts}\PY{l+s+s2}{\PYZdq{}}\PY{p}{,} \PY{n}{hue}\PY{o}{=}\PY{l+s+s2}{\PYZdq{}}\PY{l+s+s2}{esta\PYZus{}online}\PY{l+s+s2}{\PYZdq{}}\PY{p}{,} \PY{n}{data}\PY{o}{=}\PY{n}{info\PYZus{}postulantes}\PY{p}{,} \PY{n}{kind}\PY{o}{=}\PY{l+s+s2}{\PYZdq{}}\PY{l+s+s2}{strip}\PY{l+s+s2}{\PYZdq{}}\PY{p}{,} \PY{n}{jitter}\PY{o}{=}\PY{n+nb+bp}{True}\PY{p}{,} \PY{n}{size}\PY{o}{=}\PY{l+m+mi}{9}\PY{p}{,} \PY{n}{aspect}\PY{o}{=}\PY{o}{.}\PY{l+m+mi}{9}\PY{p}{)}\PY{p}{;}
\end{Verbatim}


    \begin{center}
    \adjustimage{max size={0.9\linewidth}{0.9\paperheight}}{output_44_0.png}
    \end{center}
    { \hspace*{\fill} \\}
    
    \hypertarget{section}{%
\subsection{======================================================================}\label{section}}

    \hypertarget{anuxe1lisis-sobre-las-postulaciones}{%
\section{5 - Análisis sobre las
postulaciones:}\label{anuxe1lisis-sobre-las-postulaciones}}

\hypertarget{quuxe9-duxedas-de-la-semana-se-postula-muxe1s-gente}{%
\subsection{5.1 - ¿Qué días de la semana se postula más
gente?}\label{quuxe9-duxedas-de-la-semana-se-postula-muxe1s-gente}}

La idea es tener ordenada la semana y ver cómo es que hay días más
influyentes a la hora de postularse en los anuncios, esto es
independiente a las visitas o incluso a los avisos, porque nos da la
pauta de que días la gente se termina postulando con mayor frecuencia
los diferentes avisos, sería interesante para las empresas el saber que
día en la semana es el más propicio para iniciar una búsqueda laboral.

    \begin{Verbatim}[commandchars=\\\{\}]
{\color{incolor}In [{\color{incolor}33}]:} \PY{c+c1}{\PYZsh{}\PYZsh{} GRAFICAMOS LOS DÍAS DE LA SEMANA DONDE SE POSTULAN EN LOS AVISOS.}
         \PY{n}{g} \PY{o}{=} \PY{n}{sns}\PY{o}{.}\PY{n}{countplot}\PY{p}{(}\PY{n}{x}\PY{o}{=}\PY{l+s+s1}{\PYZsq{}}\PY{l+s+s1}{dia\PYZus{}semana}\PY{l+s+s1}{\PYZsq{}}\PY{p}{,} \PY{n}{data}\PY{o}{=}\PY{n}{postulaciones}\PY{p}{,} \PY{n}{order}\PY{o}{=}\PY{n}{postulaciones}\PY{p}{[}\PY{l+s+s1}{\PYZsq{}}\PY{l+s+s1}{dia\PYZus{}semana}\PY{l+s+s1}{\PYZsq{}}\PY{p}{]}\PY{o}{.}\PY{n}{value\PYZus{}counts}\PY{p}{(}\PY{p}{)}\PY{o}{.}\PY{n}{index}\PY{p}{,} \PY{n}{orient}\PY{o}{=}\PY{l+s+s1}{\PYZsq{}}\PY{l+s+s1}{v}\PY{l+s+s1}{\PYZsq{}}\PY{p}{)}
         \PY{n}{g}\PY{o}{.}\PY{n}{set\PYZus{}xticklabels}\PY{p}{(}\PY{n}{g}\PY{o}{.}\PY{n}{get\PYZus{}xticklabels}\PY{p}{(}\PY{p}{)}\PY{p}{,}\PY{n}{rotation}\PY{o}{=}\PY{l+m+mi}{90}\PY{p}{)}
         \PY{n}{g}\PY{o}{.}\PY{n}{set\PYZus{}xlabel}\PY{p}{(}\PY{l+s+s2}{\PYZdq{}}\PY{l+s+s2}{Dia de la semana}\PY{l+s+s2}{\PYZdq{}}\PY{p}{,} \PY{n}{fontsize}\PY{o}{=}\PY{l+m+mi}{18}\PY{p}{)}
         \PY{n}{g}\PY{o}{.}\PY{n}{set\PYZus{}ylabel}\PY{p}{(}\PY{l+s+s2}{\PYZdq{}}\PY{l+s+s2}{Cantidad de postulaciones}\PY{l+s+s2}{\PYZdq{}}\PY{p}{,} \PY{n}{fontsize}\PY{o}{=}\PY{l+m+mi}{18}\PY{p}{)}
         \PY{n}{g}\PY{o}{.}\PY{n}{set\PYZus{}title}\PY{p}{(}\PY{l+s+s2}{\PYZdq{}}\PY{l+s+s2}{Postulaciones por dia de la semana}\PY{l+s+s2}{\PYZdq{}}\PY{p}{,} \PY{n}{fontsize}\PY{o}{=}\PY{l+m+mi}{18}\PY{p}{)}
\end{Verbatim}


\begin{Verbatim}[commandchars=\\\{\}]
{\color{outcolor}Out[{\color{outcolor}33}]:} Text(0.5,1,'Postulaciones por dia de la semana')
\end{Verbatim}
            
    \begin{center}
    \adjustimage{max size={0.9\linewidth}{0.9\paperheight}}{output_47_1.png}
    \end{center}
    { \hspace*{\fill} \\}
    
    \hypertarget{cuales-son-las-areas-con-mayor-cantidad-de-postulaciones}{%
\subsection{5.2 - ¿Cuales son las areas con mayor cantidad de
postulaciones?}\label{cuales-son-las-areas-con-mayor-cantidad-de-postulaciones}}

Buscamos ver cuales son las areas que mayor interés traen para los
usuarios, aunque esto está vinculado directamente a la cantidad de
avisos que haya para cada área, pero de todas maneras nos puede ser de
utilidad para corroborar cuales son estas áreas y que relación tienen
con los avisos y con las visitas realizadas. Como bien podremos ver las
areas de Ventas y Administración son las que mayor relevancia tienen y
las que más postulaciones reciben, siendo las que predominan en el
mercado.

    \begin{Verbatim}[commandchars=\\\{\}]
{\color{incolor}In [{\color{incolor}34}]:} \PY{c+c1}{\PYZsh{}\PYZsh{} GRAFICAMOS LAS ÁREAS CON MAYOR RELEVANCIA A LA HORA DE POSTULARSE EN LOS AVISOS.}
         \PY{n}{g} \PY{o}{=} \PY{n}{sns}\PY{o}{.}\PY{n}{barplot}\PY{p}{(}\PY{n}{x}\PY{o}{=} \PY{n}{top\PYZus{}postulaciones\PYZus{}area}\PY{o}{.}\PY{n}{values}\PY{p}{,} \PY{n}{y}\PY{o}{=}\PY{n}{top\PYZus{}postulaciones\PYZus{}area}\PY{o}{.}\PY{n}{index}\PY{p}{)}
         \PY{n}{g}\PY{o}{.}\PY{n}{set\PYZus{}title}\PY{p}{(}\PY{l+s+s2}{\PYZdq{}}\PY{l+s+s2}{TOP 10 Areas con mas postulaciones}\PY{l+s+s2}{\PYZdq{}}\PY{p}{,} \PY{n}{fontsize}\PY{o}{=}\PY{l+m+mi}{18}\PY{p}{)}
         \PY{n}{g}\PY{o}{.}\PY{n}{set\PYZus{}xlabel}\PY{p}{(}\PY{l+s+s2}{\PYZdq{}}\PY{l+s+s2}{Cantidad de visitas}\PY{l+s+s2}{\PYZdq{}}\PY{p}{,} \PY{n}{fontsize}\PY{o}{=}\PY{l+m+mi}{18}\PY{p}{)}
         \PY{n}{g}\PY{o}{.}\PY{n}{set\PYZus{}ylabel}\PY{p}{(}\PY{l+s+s2}{\PYZdq{}}\PY{l+s+s2}{Area}\PY{l+s+s2}{\PYZdq{}}\PY{p}{,} \PY{n}{fontsize}\PY{o}{=}\PY{l+m+mi}{18}\PY{p}{)}
\end{Verbatim}


\begin{Verbatim}[commandchars=\\\{\}]
{\color{outcolor}Out[{\color{outcolor}34}]:} Text(0,0.5,'Area')
\end{Verbatim}
            
    \begin{center}
    \adjustimage{max size={0.9\linewidth}{0.9\paperheight}}{output_49_1.png}
    \end{center}
    { \hspace*{\fill} \\}
    
    \hypertarget{cuales-vendruxedan-a-ser-las-ingenieruxedas-con-mayor-cantidad-de-postulaciones}{%
\subsection{5.3 - ¿Cuales vendrían a ser las ingenierías con mayor
cantidad de
postulaciones?}\label{cuales-vendruxedan-a-ser-las-ingenieruxedas-con-mayor-cantidad-de-postulaciones}}

Podemos tratar de hacer un paralelismo sobre las áreas en Ingeniería,
reduciendo el ámbito de estudio solo a las que se relacionan con
ingenieria o ramas afines, esto nos lleva a una conclusión bastante
particular y es que la Ingeniería Industrial es la que más relevancia
parece tener a la hora de postularse en los avisos.

    \begin{Verbatim}[commandchars=\\\{\}]
{\color{incolor}In [{\color{incolor}35}]:} \PY{c+c1}{\PYZsh{}\PYZsh{} GRAFICAMOS LAS DIEZ INGENIERÍAS CON MAYOR RELEVANCIA A LA HORA DE POSTULARSE LOS AVISOS.}
         \PY{n}{g} \PY{o}{=} \PY{n}{sns}\PY{o}{.}\PY{n}{barplot}\PY{p}{(}\PY{n}{x}\PY{o}{=}\PY{n}{diez\PYZus{}ingenierias}\PY{o}{.}\PY{n}{values}\PY{p}{,} \PY{n}{y}\PY{o}{=}\PY{n}{diez\PYZus{}ingenierias}\PY{o}{.}\PY{n}{index}\PY{p}{)}
         \PY{n}{g}\PY{o}{.}\PY{n}{set\PYZus{}title}\PY{p}{(}\PY{l+s+s2}{\PYZdq{}}\PY{l+s+s2}{TOP 10 Ingenierias con mas postulaciones}\PY{l+s+s2}{\PYZdq{}}\PY{p}{,} \PY{n}{fontsize}\PY{o}{=}\PY{l+m+mi}{18}\PY{p}{)}
         \PY{n}{g}\PY{o}{.}\PY{n}{set\PYZus{}xlabel}\PY{p}{(}\PY{l+s+s2}{\PYZdq{}}\PY{l+s+s2}{Cantidad de visitas}\PY{l+s+s2}{\PYZdq{}}\PY{p}{,} \PY{n}{fontsize}\PY{o}{=}\PY{l+m+mi}{18}\PY{p}{)}
         \PY{n}{g}\PY{o}{.}\PY{n}{set\PYZus{}ylabel}\PY{p}{(}\PY{l+s+s2}{\PYZdq{}}\PY{l+s+s2}{Area}\PY{l+s+s2}{\PYZdq{}}\PY{p}{,} \PY{n}{fontsize}\PY{o}{=}\PY{l+m+mi}{18}\PY{p}{)}
\end{Verbatim}


\begin{Verbatim}[commandchars=\\\{\}]
{\color{outcolor}Out[{\color{outcolor}35}]:} Text(0,0.5,'Area')
\end{Verbatim}
            
    \begin{center}
    \adjustimage{max size={0.9\linewidth}{0.9\paperheight}}{output_51_1.png}
    \end{center}
    { \hspace*{\fill} \\}
    
    \hypertarget{quuxe9-porcentaje-de-los-avisos-donde-se-ha-postulado-la-gente-estuxe1-online}{%
\subsection{5.4 - ¿Qué porcentaje de los avisos donde se ha postulado la
gente está
online?}\label{quuxe9-porcentaje-de-los-avisos-donde-se-ha-postulado-la-gente-estuxe1-online}}

Lo que nos sirve ver en esta gráfica es ver que porcentaje de los avisos
en los que se postula la gente estan online, entendemos que el estado
online de un aviso hace referencia a si está activo a la hora en que las
personas se han decidido postular.

    \begin{Verbatim}[commandchars=\\\{\}]
{\color{incolor}In [{\color{incolor}22}]:} \PY{c+c1}{\PYZsh{}\PYZsh{} GRAFICAMOS LA PROPORCIÓN ENTRE AVISOS EN LOS QUE SE HA POSTULADO LA GENTE ESTAN ONLINE Y OFFLINE.}
         \PY{n}{postulaciones\PYZus{}online} \PY{o}{=} \PY{n}{postulaciones\PYZus{}avisos\PYZus{}completos}\PY{p}{[}\PY{n}{postulaciones\PYZus{}avisos\PYZus{}completos}\PY{p}{[}\PY{l+s+s1}{\PYZsq{}}\PY{l+s+s1}{esta\PYZus{}online}\PY{l+s+s1}{\PYZsq{}}\PY{p}{]} \PY{o}{==} \PY{l+s+s1}{\PYZsq{}}\PY{l+s+s1}{si}\PY{l+s+s1}{\PYZsq{}}\PY{p}{]}
         \PY{n}{postulaciones\PYZus{}offline} \PY{o}{=} \PY{n}{postulaciones\PYZus{}avisos\PYZus{}completos}\PY{p}{[}\PY{n}{postulaciones\PYZus{}avisos\PYZus{}completos}\PY{p}{[}\PY{l+s+s1}{\PYZsq{}}\PY{l+s+s1}{esta\PYZus{}online}\PY{l+s+s1}{\PYZsq{}}\PY{p}{]} \PY{o}{!=} \PY{l+s+s1}{\PYZsq{}}\PY{l+s+s1}{si}\PY{l+s+s1}{\PYZsq{}}\PY{p}{]}
         \PY{n}{sizes} \PY{o}{=} \PY{p}{[}\PY{n}{postulaciones\PYZus{}online}\PY{o}{.}\PY{n}{esta\PYZus{}online}\PY{o}{.}\PY{n}{count}\PY{p}{(}\PY{p}{)}\PY{p}{,} \PY{n}{postulaciones\PYZus{}offline}\PY{o}{.}\PY{n}{esta\PYZus{}online}\PY{o}{.}\PY{n}{count}\PY{p}{(}\PY{p}{)}\PY{p}{]}
         \PY{n}{nombres} \PY{o}{=} \PY{p}{[}\PY{l+s+s1}{\PYZsq{}}\PY{l+s+s1}{Online}\PY{l+s+s1}{\PYZsq{}}\PY{p}{,} \PY{l+s+s1}{\PYZsq{}}\PY{l+s+s1}{Offline}\PY{l+s+s1}{\PYZsq{}}\PY{p}{]}
         \PY{n}{plt}\PY{o}{.}\PY{n}{figure}\PY{p}{(}\PY{n}{figsize}\PY{o}{=}\PY{p}{(}\PY{l+m+mi}{8}\PY{p}{,} \PY{l+m+mi}{6}\PY{p}{)}\PY{p}{)}
         \PY{n}{plt}\PY{o}{.}\PY{n}{title}\PY{p}{(}\PY{l+s+s1}{\PYZsq{}}\PY{l+s+s1}{Distribucion sobre las visitas a los avisos online/offline}\PY{l+s+s1}{\PYZsq{}}\PY{p}{,} \PY{n}{fontsize}\PY{o}{=}\PY{l+m+mi}{20}\PY{p}{)}
         \PY{n}{plt}\PY{o}{.}\PY{n}{pie}\PY{p}{(}\PY{n}{sizes}\PY{p}{,} \PY{n}{labels}\PY{o}{=}\PY{n}{nombres}\PY{p}{,} \PY{n}{autopct}\PY{o}{=}\PY{l+s+s1}{\PYZsq{}}\PY{l+s+si}{\PYZpc{}1.1f}\PY{l+s+si}{\PYZpc{}\PYZpc{}}\PY{l+s+s1}{\PYZsq{}}\PY{p}{,} \PY{n}{startangle}\PY{o}{=}\PY{l+m+mi}{20}\PY{p}{,} \PY{n}{colors}\PY{o}{=}\PY{p}{[}\PY{l+s+s1}{\PYZsq{}}\PY{l+s+s1}{lightgreen}\PY{l+s+s1}{\PYZsq{}}\PY{p}{,} \PY{l+s+s1}{\PYZsq{}}\PY{l+s+s1}{lightblue}\PY{l+s+s1}{\PYZsq{}}\PY{p}{]}\PY{p}{,} \PY{n}{explode}\PY{o}{=}\PY{p}{(}\PY{l+m+mf}{0.1}\PY{p}{,} \PY{l+m+mi}{0}\PY{p}{)}\PY{p}{)}
         \PY{n}{plt}\PY{o}{.}\PY{n}{show}\PY{p}{(}\PY{p}{)}
\end{Verbatim}


    \begin{center}
    \adjustimage{max size={0.9\linewidth}{0.9\paperheight}}{output_53_0.png}
    \end{center}
    { \hspace*{\fill} \\}
    
    \hypertarget{cuales-son-las-empresas-con-mayor-cantidad-de-postulaciones-en-sus-avisos}{%
\subsection{5.5 - ¿Cuales son las empresas con mayor cantidad de
postulaciones en sus
avisos?}\label{cuales-son-las-empresas-con-mayor-cantidad-de-postulaciones-en-sus-avisos}}

Sobre este punto hay que tener en cuenta que esto vinculará las empresas
que tengan mayor cantidad de postulaciones pero también se relaciona con
aquellas que tienen mayor cantidad de avisos, lo que llevará a tener por
lógica una cantidad mayor de postulaciones, por otro lado se trató de
unificar algunas empresas que se dividian en zonas pero que en
definitiva eran la misma empresa. Se ha decidido tomar las diez primeras
como muestra de relevancia.

    \begin{Verbatim}[commandchars=\\\{\}]
{\color{incolor}In [{\color{incolor}24}]:} \PY{c+c1}{\PYZsh{}\PYZsh{} CONTAMOS LAS DIEZ EMPRESAS CON MAYOR CANTIDAD DE POSTULACIONES EN SUS AVISOS.}
         \PY{n}{g} \PY{o}{=} \PY{n}{sns}\PY{o}{.}\PY{n}{barplot}\PY{p}{(}\PY{n}{x}\PY{o}{=}\PY{n}{diez\PYZus{}postulaciones\PYZus{}empresas}\PY{o}{.}\PY{n}{values}\PY{p}{,} \PY{n}{y}\PY{o}{=}\PY{n}{diez\PYZus{}postulaciones\PYZus{}empresas}\PY{o}{.}\PY{n}{index}\PY{p}{)}
         \PY{n}{g}\PY{o}{.}\PY{n}{set\PYZus{}title}\PY{p}{(}\PY{l+s+s2}{\PYZdq{}}\PY{l+s+s2}{TOP 10 con mas postulaciones a sus avisos}\PY{l+s+s2}{\PYZdq{}}\PY{p}{,} \PY{n}{fontsize}\PY{o}{=}\PY{l+m+mi}{18}\PY{p}{)}
         \PY{n}{g}\PY{o}{.}\PY{n}{set\PYZus{}xlabel}\PY{p}{(}\PY{l+s+s2}{\PYZdq{}}\PY{l+s+s2}{Cantidad de visitas}\PY{l+s+s2}{\PYZdq{}}\PY{p}{,} \PY{n}{fontsize}\PY{o}{=}\PY{l+m+mi}{18}\PY{p}{)}
         \PY{n}{g}\PY{o}{.}\PY{n}{set\PYZus{}ylabel}\PY{p}{(}\PY{l+s+s2}{\PYZdq{}}\PY{l+s+s2}{Nombre de empresa}\PY{l+s+s2}{\PYZdq{}}\PY{p}{,} \PY{n}{fontsize}\PY{o}{=}\PY{l+m+mi}{18}\PY{p}{)}
\end{Verbatim}


\begin{Verbatim}[commandchars=\\\{\}]
{\color{outcolor}Out[{\color{outcolor}24}]:} Text(0,0.5,'Nombre de empresa')
\end{Verbatim}
            
    \begin{center}
    \adjustimage{max size={0.9\linewidth}{0.9\paperheight}}{output_55_1.png}
    \end{center}
    { \hspace*{\fill} \\}
    
    \hypertarget{quienes-se-postulan-muxe1s-a-los-avisos-hombres-o-mujeres}{%
\subsection{5.6 - ¿Quienes se postulan más a los avisos? ¿Hombres o
mujeres?}\label{quienes-se-postulan-muxe1s-a-los-avisos-hombres-o-mujeres}}

Buscamos ver que relación hay entre el género a la hora de postularse en
los avisos, y ver si estos en definitiva están online u offline,
pudiendo así ver cúal es el comportamiento que se va desarrollando entre
las personas, habiendo casos donde quizás una persona se postuló en más
de 500 avisos en lo que puede parecer una actitud frenética aunque más
bien podría parecer ser un bot, son cosas que nos permiten analizar este
tipo de situaciones o al menos tener un dejo de duda sobre el mismo, por
otro lado como es lógico los valores donde tenemos mayor cantidad son
entre 0 a 100.

    \begin{Verbatim}[commandchars=\\\{\}]
{\color{incolor}In [{\color{incolor}8}]:} \PY{n}{postulaciones\PYZus{}avisos\PYZus{}completos} \PY{o}{=} \PY{n}{postulaciones\PYZus{}avisos\PYZus{}completos}\PY{o}{.}\PY{n}{merge}\PY{p}{(}\PY{n}{postulantes\PYZus{}unicos}\PY{p}{,}\PY{n}{left\PYZus{}on}\PY{o}{=}\PY{l+s+s1}{\PYZsq{}}\PY{l+s+s1}{idpostulante}\PY{l+s+s1}{\PYZsq{}}\PY{p}{,} \PY{n}{right\PYZus{}on}\PY{o}{=}\PY{l+s+s1}{\PYZsq{}}\PY{l+s+s1}{idpostulante}\PY{l+s+s1}{\PYZsq{}}\PY{p}{,}\PY{n}{how}\PY{o}{=}\PY{l+s+s1}{\PYZsq{}}\PY{l+s+s1}{left}\PY{l+s+s1}{\PYZsq{}}\PY{p}{)}
\end{Verbatim}


    \begin{Verbatim}[commandchars=\\\{\}]
{\color{incolor}In [{\color{incolor}9}]:} \PY{c+c1}{\PYZsh{}\PYZsh{} AGRUPAMOS POR SEXO Y SEGÚN LA RELACIÓN DE SI EL AVISO ESTÁ ONLINE O NO Y COMO ES LA PROPORCIÓN DE POSTULACIONES.}
        \PY{n}{postulaciones\PYZus{}avisos\PYZus{}completos} \PY{o}{=} \PY{n}{postulaciones\PYZus{}avisos\PYZus{}completos}\PY{o}{.}\PY{n}{merge}\PY{p}{(}\PY{n}{postulantes\PYZus{}unicos}\PY{p}{,}\PY{n}{left\PYZus{}on}\PY{o}{=}\PY{l+s+s1}{\PYZsq{}}\PY{l+s+s1}{idpostulante}\PY{l+s+s1}{\PYZsq{}}\PY{p}{,} \PY{n}{right\PYZus{}on}\PY{o}{=}\PY{l+s+s1}{\PYZsq{}}\PY{l+s+s1}{idpostulante}\PY{l+s+s1}{\PYZsq{}}\PY{p}{,}\PY{n}{how}\PY{o}{=}\PY{l+s+s1}{\PYZsq{}}\PY{l+s+s1}{left}\PY{l+s+s1}{\PYZsq{}}\PY{p}{)}
        \PY{n}{info\PYZus{}postulantes} \PY{o}{=} \PY{n}{postulaciones\PYZus{}avisos\PYZus{}completos}\PY{o}{.}\PY{n}{groupby}\PY{p}{(}\PY{p}{[}\PY{l+s+s1}{\PYZsq{}}\PY{l+s+s1}{idpostulante}\PY{l+s+s1}{\PYZsq{}}\PY{p}{,} \PY{l+s+s1}{\PYZsq{}}\PY{l+s+s1}{sexo}\PY{l+s+s1}{\PYZsq{}}\PY{p}{,}\PY{l+s+s1}{\PYZsq{}}\PY{l+s+s1}{esta\PYZus{}online}\PY{l+s+s1}{\PYZsq{}}\PY{p}{]}\PY{p}{)}\PY{o}{.}\PY{n}{size}\PY{p}{(}\PY{p}{)}\PY{o}{.}\PY{n}{reset\PYZus{}index}\PY{p}{(}\PY{n}{name}\PY{o}{=}\PY{l+s+s1}{\PYZsq{}}\PY{l+s+s1}{cantidad}\PY{l+s+s1}{\PYZsq{}}\PY{p}{)}
        \PY{n}{info\PYZus{}postulantes} \PY{o}{=} \PY{n}{info\PYZus{}postulantes}\PY{p}{[}\PY{n}{info\PYZus{}postulantes}\PY{p}{[}\PY{l+s+s1}{\PYZsq{}}\PY{l+s+s1}{cantidad}\PY{l+s+s1}{\PYZsq{}}\PY{p}{]} \PY{o}{\PYZlt{}} \PY{l+m+mi}{500}\PY{p}{]}
        \PY{n}{g} \PY{o}{=} \PY{n}{sns}\PY{o}{.}\PY{n}{factorplot}\PY{p}{(}\PY{n}{x}\PY{o}{=}\PY{l+s+s2}{\PYZdq{}}\PY{l+s+s2}{sexo}\PY{l+s+s2}{\PYZdq{}}\PY{p}{,} \PY{n}{y}\PY{o}{=}\PY{l+s+s2}{\PYZdq{}}\PY{l+s+s2}{cantidad}\PY{l+s+s2}{\PYZdq{}}\PY{p}{,} \PY{n}{hue}\PY{o}{=}\PY{l+s+s2}{\PYZdq{}}\PY{l+s+s2}{esta\PYZus{}online}\PY{l+s+s2}{\PYZdq{}}\PY{p}{,} \PY{n}{data}\PY{o}{=}\PY{n}{info\PYZus{}postulantes}\PY{p}{,} \PY{n}{kind}\PY{o}{=}\PY{l+s+s2}{\PYZdq{}}\PY{l+s+s2}{strip}\PY{l+s+s2}{\PYZdq{}}\PY{p}{,} \PY{n}{jitter}\PY{o}{=}\PY{n+nb+bp}{True}\PY{p}{,} \PY{n}{size}\PY{o}{=}\PY{l+m+mi}{8}\PY{p}{,} \PY{n}{aspect}\PY{o}{=}\PY{o}{.}\PY{l+m+mi}{9}\PY{p}{)}\PY{p}{;}
\end{Verbatim}


    \begin{center}
    \adjustimage{max size={0.9\linewidth}{0.9\paperheight}}{output_58_0.png}
    \end{center}
    { \hspace*{\fill} \\}
    
    \hypertarget{section}{%
\subsection{======================================================================}\label{section}}

    \hypertarget{conclusiones-generales}{%
\section{6 - Conclusiones generales:}\label{conclusiones-generales}}

\hypertarget{conclusiones-sobre-lo-que-hemos-obtenido-del-informe.}{%
\subsection{6.1 - Conclusiones sobre lo que hemos obtenido del
informe.}\label{conclusiones-sobre-lo-que-hemos-obtenido-del-informe.}}

En principio el set de datos que obtuvimos estaba bastante prolijo. A
comparación con lo visto en los ejemplos de otros cuatrimestres, los
duplicados se dieron en cuanto empezamos a mergear tablas para cruzar
datos.

En esta primera instancia no encontramos la forma de utilizar estos
datos en pos del negocio de bienes raíces al que apunta la empresa, pero
encontramos algunos datos interesantes sobre los rangos etáreos que
maneja y el nivel de educación de los postulantes.

Esto puede ser muy útil para campañas de marketing y para vender el
servicio de consultoría en recursos humanos a las empresas que busquen
personal, uno de los principales problemas de las empresas hoy en día es
``Conocer a su cliente'' y en este análisis vimos los dos tipos de
clientes que tiene Navent: los candidatos y las empresas que buscan
recursos. Con estos datos sabemos qué tipo de candidatos tenemos y que
empresas buscan.

\hypertarget{repositorio-github.}{%
\subsection{6.2 - Repositorio GitHub.}\label{repositorio-github.}}

Nuestro repositorio de github es el siguiente:
https://github.com/shizus/orga\_datos\_2018\_1c

\hypertarget{section}{%
\subsection{======================================================================}\label{section}}


    % Add a bibliography block to the postdoc
    
    
    
    \end{document}
